\documentclass[runningheads,a4paper]{llncs}

\usepackage{amssymb}
\usepackage{amsmath}

\usepackage{mathrsfs}
\usepackage{stmaryrd}

\usepackage{enumitem}
%\usepackage{enumerate}

\usepackage{color}
\usepackage{graphicx}
\usepackage{rotating}
%\usepackage{xparse}
%\usepackage{latex8}
\usepackage{upgreek} 
\usepackage{cmll}
\usepackage{url}
\usepackage{xifthen}% provides \isempty test
\usepackage{multirow}

\setcounter{tocdepth}{3}

\urldef{\mailsa}\path|{acarraro}@dsi.unive.it|
%\urldef{\mailsb}\path||
%\urldef{\mailsc}\path|
\newcommand{\keywords}[1]{\par\addvspace\baselineskip
\noindent\keywordname\enspace\ignorespaces#1}

\input prooftree.sty
\input xy
\xyoption{all}

\makeindex

\begin{document}

\mainmatter  % start of an individual contribution

% first the title is needed
\title{Note del corso di Calcolabilit\`{a} e Linguaggi Formali - Lezione 2}

% a short form should be given in case it is too long for the running head
\titlerunning{Note del corso di Calcolabilit\`{a} e Linguaggi Formali - Lezione 2}

% the name(s) of the author(s) follow(s) next
%
% NB: Chinese authors should write their first names(s) in front of
% their surnames. This ensures that the names appear correctly in
% the running heads and the author index.
%
\author{Alberto Carraro \\ 12 ottobre 2011}
%
\authorrunning{A. Carraro}
% (feature abused for this document to repeat the title also on left hand pages)

% the affiliations are given next; don't give your e-mail address
% unless you accept that it will be published
\institute{DAIS, Universit\`{a} Ca' Foscari Venezia
%\mailsa\\
%\mailsb\\
%\mailsc\\
\url{http://www.dsi.unive.it/~acarraro}
}

%
% NB: a more complex sample for affiliations and the mapping to the
% corresponding authors can be found in the file "llncs.dem"
% (search for the string "\mainmatter" where a contribution starts).
% "llncs.dem" accompanies the document class "llncs.cls".
%

\toctitle{Note del corso di Calcolabilit\`{a} e Linguaggi Formali - Lezione 2}
\tocauthor{A. Carraro}

\newcommand{\scA}{\mathscr{A}}
\newcommand{\scB}{\mathscr{B}}
\newcommand{\scC}{\mathscr{C}}
\newcommand{\scD}{\mathscr{D}}
\newcommand{\scE}{\mathscr{E}}
\newcommand{\scF}{\mathscr{F}}
\newcommand{\scG}{\mathscr{G}}
\newcommand{\scH}{\mathscr{H}}
\newcommand{\scI}{\mathscr{I}}
\newcommand{\scJ}{\mathscr{J}}
\newcommand{\scK}{\mathscr{K}}
\newcommand{\scL}{\mathscr{L}}
\newcommand{\scM}{\mathscr{M}}
\newcommand{\scN}{\mathscr{N}}
\newcommand{\scO}{\mathscr{O}}
\newcommand{\scP}{\mathscr{P}}
\newcommand{\scQ}{\mathscr{Q}}
\newcommand{\scR}{\mathscr{R}}
\newcommand{\scS}{\mathscr{S}}
\newcommand{\scT}{\mathscr{T}}
\newcommand{\scU}{\mathscr{U}}
\newcommand{\scV}{\mathscr{V}}
\newcommand{\scW}{\mathscr{W}}
\newcommand{\scX}{\mathscr{X}}
\newcommand{\scY}{\mathscr{Y}}
\newcommand{\scZ}{\mathscr{Z}}

\newcommand{\fA}{\mathfrak{A}}
\newcommand{\fB}{\mathfrak{B}}
\newcommand{\fC}{\mathfrak{C}}
\newcommand{\fD}{\mathfrak{D}}
\newcommand{\fE}{\mathfrak{E}}
\newcommand{\fF}{\mathfrak{F}}
\newcommand{\fG}{\mathfrak{G}}
\newcommand{\fH}{\mathfrak{H}}
\newcommand{\fI}{\mathfrak{I}}
\newcommand{\fJ}{\mathfrak{J}}
\newcommand{\fK}{\mathfrak{K}}
\newcommand{\fL}{\mathfrak{L}}
\newcommand{\fM}{\mathfrak{M}}
\newcommand{\fN}{\mathfrak{N}}
\newcommand{\fO}{\mathfrak{O}}
\newcommand{\fP}{\mathfrak{P}}
\newcommand{\fQ}{\mathfrak{Q}}
\newcommand{\fR}{\mathfrak{R}}
\newcommand{\fS}{\mathfrak{S}}
\newcommand{\fT}{\mathfrak{T}}
\newcommand{\fU}{\mathfrak{U}}
\newcommand{\fV}{\mathfrak{V}}
\newcommand{\fW}{\mathfrak{W}}
\newcommand{\fX}{\mathfrak{X}}
\newcommand{\fY}{\mathfrak{Y}}
\newcommand{\fZ}{\mathfrak{Z}}

\newcommand\tA{{\mathsf{A}}}
\newcommand\tB{{\mathsf{B}}}
\newcommand\tC{{\mathsf{C}}}
\newcommand\tD{{\mathsf{D}}}
\newcommand\tE{{\mathsf{E}}}
\newcommand{\tF}{\mathsf{F}}
\newcommand\tG{{\mathsf{G}}}
\newcommand\tH{{\mathsf{H}}}
\newcommand\tI{{\mathsf{I}}}
\newcommand\tJ{{\mathsf{J}}}
\newcommand\tK{{\mathsf{K}}}
\newcommand\tL{{\mathsf{L}}}
\newcommand\tM{{\mathsf{M}}}
\newcommand\tN{{\mathsf{N}}}
\newcommand\tO{{\mathsf{O}}}
\newcommand\tP{{\mathsf{P}}}
\newcommand\tQ{{\mathsf{Q}}}
\newcommand\tR{{\mathsf{R}}}
\newcommand\tS{{\mathsf{S}}}
\newcommand\tT{{\mathsf{T}}}
\newcommand\tU{{\mathsf{U}}}
\newcommand\tV{{\mathsf{V}}}
\newcommand\tW{{\mathsf{W}}}
\newcommand\tX{{\mathsf{X}}}
\newcommand\tY{{\mathsf{Y}}}
\newcommand\tZ{{\mathsf{Z}}}

%Sums
\newcommand{\sM}{\mathbb{M}}
\newcommand{\sN}{\mathbb{N}}
\newcommand{\sL}{\mathbb{L}}
\newcommand{\sH}{\mathbb{H}}
\newcommand{\sP}{\mathbb{P}}
\newcommand{\sQ}{\mathbb{Q}}
\newcommand{\sR}{\mathbb{R}}
\newcommand{\sA}{\mathbb{A}}
\newcommand{\sB}{\mathbb{B}}
\newcommand{\sC}{\mathbb{C}}
\newcommand{\sD}{\mathbb{D}}

%overlined letters
\newcommand{\ova}{\bar{a}}
\newcommand{\ovb}{\bar{b}}
\newcommand{\ovc}{\bar{c}}
\newcommand{\ovd}{\bar{d}}
\newcommand{\ove}{\bar{e}}
\newcommand{\ovf}{\bar{f}}
\newcommand{\ovg}{\bar{g}}
\newcommand{\ovh}{\bar{h}}
\newcommand{\ovi}{\bar{i}}
\newcommand{\ovj}{\bar{j}}
\newcommand{\ovk}{\bar{k}}
\newcommand{\ovl}{\bar{l}}
\newcommand{\ovm}{\bar{m}}
\newcommand{\ovn}{\bar{n}}
\newcommand{\ovo}{\bar{o}}
\newcommand{\ovp}{\bar{p}}
\newcommand{\ovq}{\bar{q}}
\newcommand{\ovr}{\bar{r}}
\newcommand{\ovs}{\bar{s}}
\newcommand{\ovt}{\bar{t}}
\newcommand{\ovu}{\bar{u}}
\newcommand{\ovv}{\bar{v}}
\newcommand{\ovw}{\bar{w}}
\newcommand{\ovx}{\bar{x}}
\newcommand{\ovy}{\bar{y}}
\newcommand{\ovz}{\bar{z}}

%overlined capital letters
\newcommand{\ovA}{\overline{A}}
\newcommand{\ovB}{\overline{B}}
\newcommand{\ovC}{\overline{C}}
\newcommand{\ovD}{\overline{D}}
\newcommand{\ovE}{\overline{E}}
\newcommand{\ovF}{\overline{F}}
\newcommand{\ovG}{\overline{G}}
\newcommand{\ovH}{\overline{H}}
\newcommand{\ovI}{\overline{I}}
\newcommand{\ovJ}{\overline{J}}
\newcommand{\ovK}{\overline{K}}
\newcommand{\ovL}{\overline{L}}
\newcommand{\ovM}{\overline{M}}
\newcommand{\ovN}{\overline{N}}
\newcommand{\ovO}{\overline{O}}
\newcommand{\ovP}{\overline{P}}
\newcommand{\ovQ}{\overline{Q}}
\newcommand{\ovR}{\overline{R}}
\newcommand{\ovS}{\overline{S}}
\newcommand{\ovT}{\overline{T}}
\newcommand{\ovU}{\overline{U}}
\newcommand{\ovV}{\overline{V}}
\newcommand{\ovW}{\overline{W}}
\newcommand{\ovX}{\overline{X}}
\newcommand{\ovY}{\overline{Y}}
\newcommand{\ovZ}{\overline{Z}}

%vec capital letters
\newcommand{\veA}{\vec{A}}
\newcommand{\veB}{\vec{B}}
\newcommand{\veC}{\vec{C}}
\newcommand{\veD}{\vec{D}}
\newcommand{\veE}{\vec{E}}
\newcommand{\veF}{\vec{F}}
\newcommand{\veG}{\vec{G}}
\newcommand{\veH}{\vec{H}}
\newcommand{\veI}{\vec{I}}
\newcommand{\veJ}{\vec{J}}
\newcommand{\veK}{\vec{K}}
\newcommand{\veL}{\vec{L}}
\newcommand{\veM}{\vec{M}}
\newcommand{\veN}{\vec{N}}
\newcommand{\veO}{\vec{O}}
\newcommand{\veP}{\vec{P}}
\newcommand{\veQ}{\vec{Q}}
\newcommand{\veR}{\vec{R}}
\newcommand{\veS}{\vec{S}}
\newcommand{\veT}{\vec{T}}
\newcommand{\veU}{\vec{U}}
\newcommand{\veV}{\vec{V}}
\newcommand{\veW}{\vec{W}}
\newcommand{\veX}{\vec{X}}
\newcommand{\veY}{\vec{Y}}
\newcommand{\veZ}{\vec{Z}}

%bold capital letters
\newcommand{\bA}{\mathbf{A}}
\newcommand{\bB}{\mathbf{B}}
\newcommand{\bC}{\mathbf{C}}
\newcommand{\bD}{\mathbf{D}}
\newcommand{\bE}{\mathbf{E}}
\newcommand{\bF}{\mathbf{F}}
\newcommand{\bG}{\mathbf{G}}
\newcommand{\bH}{\mathbf{H}}
\newcommand{\bI}{\mathbf{I}}
\newcommand{\bJ}{\mathbf{J}}
\newcommand{\bK}{\mathbf{K}}
\newcommand{\bL}{\mathbf{L}}
\newcommand{\bM}{\mathbf{M}}
\newcommand{\bN}{\mathbf{N}}
\newcommand{\bO}{\mathbf{O}}
\newcommand{\bP}{\mathbf{P}}
\newcommand{\bQ}{\mathbf{Q}}
\newcommand{\bR}{\mathbf{R}}
\newcommand{\bS}{\mathbf{S}}
\newcommand{\bT}{\mathbf{T}}
\newcommand{\bU}{\mathbf{U}}
\newcommand{\bV}{\mathbf{V}}
\newcommand{\bW}{\mathbf{W}}
\newcommand{\bX}{\mathbf{X}}
\newcommand{\bY}{\mathbf{Y}}
\newcommand{\bZ}{\mathbf{Z}}

\newcommand{\mbbA}{\mathbb{A}}
\newcommand{\mbbB}{\mathbb{B}}
\newcommand{\mbbC}{\mathbb{C}}
\newcommand{\mbbD}{\mathbb{D}}
\newcommand{\mbbE}{\mathbb{E}}
\newcommand{\mbbF}{\mathbb{F}}
\newcommand{\mbbG}{\mathbb{G}}
\newcommand{\mbbH}{\mathbb{H}}
\newcommand{\mbbI}{\mathbb{I}}
\newcommand{\mbbL}{\mathbb{L}}
\newcommand{\mbbM}{\mathbb{M}}
\newcommand{\mbbN}{\mathbb{N}}
\newcommand{\mbbW}{\mathbb{W}}
\newcommand{\mbbY}{\mathbb{Y}}
\newcommand{\mbbX}{\mathbb{X}}
\newcommand{\mbbZ}{\mathbb{Z}}

%lower case greek letters
\newcommand{\ga}{\alpha}
\newcommand{\gb}{\beta}
\newcommand{\gc}{\gamma}
\newcommand{\gd}{\delta}
\newcommand{\gep}{\varepsilon}
\newcommand{\gz}{\zeta}
\newcommand{\geta}{\eta}
\newcommand{\gth}{\theta}
\newcommand{\gi}{\iota}
\newcommand{\gv}{\nu}
\newcommand{\gk}{\kappa}
\newcommand{\gl}{\lambda}
\newcommand{\gm}{\mu}
\newcommand{\gn}{\nu}
\newcommand{\gx}{\xi}
\newcommand{\gp}{\pi}
\newcommand{\gr}{\rho}
\newcommand{\gs}{\sigma}
\newcommand{\gt}{\ensuremath{\tau}}
\newcommand{\gu}{\upsilon}
% \newcommand{\gph}{\varphi}
\newcommand{\gch}{\chi}
\newcommand{\gps}{\psi}
\newcommand{\go}{\omega}
\newcommand{\gto}{\ensuremath{\bar\tau}}

%bold lower case greek letters
%\newcommand\ssn{\mbox{\boldmath $\eta$}}
\newcommand{\bga}{\mbox{\boldmath $\alpha$}}
\newcommand{\bgb}{\mbox{\boldmath $\beta$}}
\newcommand{\bgc}{\mbox{\boldmath $\gamma$}}
\newcommand{\bgp}{\mbox{\boldmath $\pi$}}
\newcommand{\bgd}{\mbox{\boldmath $\delta$}}
\newcommand{\bge}{\mbox{\boldmath $\epsilon$}}
\newcommand{\bgs}{\mbox{\boldmath $\sigma$}}
\newcommand{\bgt}{\mbox{\boldmath $\tau$}}
\newcommand{\bgr}{\mbox{\boldmath $\rho$}}
\newcommand{\bgch}{\mbox{\boldmath $\chi$}}
\newcommand{\bgo}{\mbox{\boldmath $\omega$}}

%upper case greek letters
\newcommand{\gG}{\Gamma}
\newcommand{\gF}{\Phi}
\newcommand{\gD}{\Delta}
\newcommand{\gT}{\Theta}
\newcommand{\gP}{\Pi}
\newcommand{\gX}{\Xi}
\newcommand{\gS}{\Sigma}
\newcommand{\gO}{\Omega}
\newcommand{\gL}{\Lambda}

\newcommand\rA{{\mathrm{A}}}
\newcommand\rB{{\mathrm{B}}}
\newcommand\rC{{\mathrm{C}}}
\newcommand\rD{{\mathrm{D}}}
\newcommand\rE{{\mathrm{E}}}
\newcommand{\rF}{\mathrm{F}}
\newcommand\rG{{\mathrm{G}}}
\newcommand\rH{{\mathrm{H}}}
\newcommand\rI{{\mathrm{I}}}
\newcommand\rL{{\mathrm{L}}}
 
%\newcommand\ra{{\mathrm{a}}}
\newcommand\rb{{\mathrm{b}}}
\newcommand\rc{{\mathrm{c}}}
\newcommand\rd{{\mathrm{d}}}
\newcommand\re{{\mathrm{e}}}
\newcommand{\rf}{\mathrm{f}}
\newcommand\rg{{\mathrm{g}}}
\newcommand\rh{{\mathrm{h}}}
\newcommand\ri{{\mathrm{i}}}
\newcommand\rl{{\mathrm{l}}}
\newcommand\mrm{{\mathrm{m}}}
\newcommand\rn{{\mathrm{n}}}
\newcommand\ro{{\mathrm{o}}}
\newcommand\rp{{\mathrm{p}}}
%\newcommand\rq{{\mathrm{q}}}
\newcommand\rr{{\mathrm{r}}}
\newcommand\rs{{\mathrm{s}}}
\newcommand\rt{{\mathrm{t}}}

\newcommand{\cA}{\mathcal{A}}
\newcommand{\cB}{\mathcal{B}}
\newcommand{\cC}{\mathcal{C}}
\newcommand{\cD}{\mathcal{D}}
\newcommand{\cE}{\mathcal{E}}
\newcommand{\cF}{\mathcal{F}}
\newcommand{\cG}{\mathcal{G}}
\newcommand{\cH}{\mathcal{H}}
\newcommand{\cI}{\mathcal{I}}
\newcommand{\cJ}{\mathcal{J}}
\newcommand{\cK}{\mathcal{K}}
\newcommand{\cL}{\mathcal{L}}
\newcommand{\cM}{\mathcal{M}}
\newcommand{\cN}{\mathcal{N}}
\newcommand{\cO}{\mathcal{O}}
\newcommand{\cP}{\mathcal{P}}
\newcommand{\cQ}{\mathcal{Q}}
\newcommand{\cR}{\mathcal{R}}
\newcommand{\cS}{\mathcal{S}}
\newcommand{\cT}{\mathcal{T}}
\newcommand{\cU}{\mathcal{U}}
\newcommand{\cV}{\mathcal{V}}
\newcommand{\cW}{\mathcal{W}}
\newcommand{\cX}{\mathcal{X}}
\newcommand{\cY}{\mathcal{Y}}
\newcommand{\cZ}{\mathcal{Z}}

\newenvironment{myitem}%
{\begin{list}%
       {-}%
       {\setlength{\itemsep}{0pt}
     \setlength{\parsep}{3pt}
     \setlength{\topsep}{3pt}
     \setlength{\partopsep}{0pt}
     \setlength{\leftmargin}{0.7em}
     \setlength{\labelwidth}{1em}
     \setlength{\labelsep}{0.3em}}}%
{\end{list}}

\newenvironment{myitemize}%
{\begin{list}%
       {-}%
       {\setlength{\itemsep}{0pt}
     \setlength{\parsep}{2pt}
     \setlength{\topsep}{2pt}
     \setlength{\partopsep}{0pt}
     \setlength{\leftmargin}{2em}
     \setlength{\labelwidth}{1em}
     \setlength{\labelsep}{0.3em}}}%
{\end{list}}

%Alberto's macros
\newcommand{\ls}[2]{\langle #2 / #1\rangle} % linear substitution
\newcommand{\cs}[2]{\{ #2 / #1\}} % classical substitution
%\newcommand{\ls}[2]{\langle #1:=#2\rangle} % linear substitution
%\newcommand{\cs}[2]{\{ #1:=#2\}} % classical substitution
\newcommand{\Bag}[1]{[#1]} % bag formation
\renewcommand{\smallsetminus}{-}
%Giulio's macros
%Sets:
%\newcommand{\nat}{\mathcal{N}}
\newcommand{\mbz}{\mathbf{0}}
\newcommand{\mbo}{\mathbf{1}}
\newcommand{\mbt}{\mathbf{2}}
\newcommand{\rea}[1]{\mathsf{rea}(#1)} % set of realizers of #1
\newcommand{\realize}{\Vdash} % realizability relation
\newcommand{\natp}{\nat^+}
\newcommand{\one}{\mathbf{1}}
\newcommand{\bool}{\mathbf{2}}
\newcommand{\perm}[1]{\fS_{#1}}
\newcommand{\card}[1]{\# #1} % cardinality of a set
%Boh
\newcommand{\Omegatuple}[1]{\Mfin{#1}^{(\omega)}}
\newcommand{\Pow}[1]{\cP(#1)}
\newcommand{\Powf}[1]{\cP_{\mathrm{f}}(#1)}
\newcommand{\Id}[1]{\mathrm{Id}_{#1}}
\newcommand{\comp}{\circ}
\newcommand{\With}[2]{{#1}\with{#2}}
\newcommand{\Termobj}{1}
\newcommand{\App}{\mathrm{Ap}}
\newcommand{\Abs}{\uplambda}
\newcommand{\Funint}[2]{[{#1}\!\!\imp\!\!{#2}]}

%Lambda calculus:
%\newcommand{\full}{\gto{\bang}}
%\newcommand{\dlam}{\ensuremath{\partial\lambda}}
%\newcommand{\dzlam}{\ensuremath{\partial_0\lambda}}
%\newcommand{\lam}{\ensuremath{\lambda}}
%\newcommand{\bang}{\oc}
%\newcommand{\hole}[1]{\llparenthesis #1\rrparenthesis}
\newcommand{\paral}{\vert}
\newcommand{\FSet}[1]{\Lambda^{#1}_{\bang}}
\newcommand{\supp}[1]{\mathsf{su}(#1)} % support of multises

%\newcommand{\tContSet}{\Set{\gt}\hole{\cdot}} % bang-free test contexts
%\newcommand{\tFContSet}{\FSet{\gt}\hole{\cdot}} % all test contexts

%\newcommand{\ContSet}{\Set{\gt}\hole{\cdot}} % bang-free term contexts
%\newcommand{\FContSet}{\FSet{\gt}\hole{\cdot}} % all term contexts

\newcommand{\sums}[1]{\bool\langle\Set{#1}\rangle}
\newcommand{\Fsums}[1]{\bool\langle\FSet{#1}\rangle}
\newcommand{\la}{\leftarrow}
\newcommand{\ot}{\leftarrow}
\newcommand{\labelot}[1]{\ _{#1}\!\leftarrow} % left arrow with label
\newcommand{\labelto}[1]{\rightarrow_{#1}} % right arrow with label
\newcommand{\mslabelot}[1]{\ _{#1}\!\twoheadleftarrow} % left two head arrow with label
\newcommand{\mslabelto}[1]{\twoheadrightarrow_{#1}} % right two head arrow with label
\newcommand{\msla}{\twoheadleftarrow} 
\newcommand{\msto}{\twoheadrightarrow}
\newcommand{\toh}{\to_{h}} % head reduction
\newcommand{\mstoh}{\msto_{h}} % transitive head reduction
\newcommand{\etoh}{\to_{h\eta}} % extensional head reduction
\newcommand{\msetoh}{\msto_{h\eta}} % extensional transitive head reduction
\newcommand{\too}{\to_{o}} % outer-reduction
\newcommand{\mstoo}{\msto_{o}} % transitive outer-reduction
\newcommand{\etoo}{\to_{o\eta}} % extensional outer-reduction
\newcommand{\msetoo}{\msto_{o\eta}} % extensional transitive outer-reduction
\newcommand{\eqt}{=_{\theta}} % weakly extensional conversion
\newcommand{\eqte}{=_{\theta\eta}} % extensional conversion
\newcommand{\eq}{=} % basic conversion

\newcommand{\dg}[2]{\mathrm{deg}_{#1}(#2)} % degree of a variable #1 in a term #2

\newcommand{\obsle}{\sqsubseteq_{\mathcal{O}}} % observational preorder
\newcommand{\obseq}{\approx_{\mathcal{O}}} % observational equivalence

\newcommand{\tesle}{\sqsubseteq_{\mathcal{C}}} % convergence preorder
\newcommand{\teseq}{\approx_{\mathcal{C}}} % convergence equivalence

\newcommand{\Fobsle}{\sqsubseteq^{\bang}_{\mathcal{O}}} % full observational preorder
\newcommand{\Fobseq}{\approx^{\bang}_{\mathcal{O}}} % full observational equivalence

\newcommand{\Ftesle}{\sqsubseteq^{\bang}_{\mathcal{C}}} % full convergence preorder
\newcommand{\Fteseq}{\approx^{\bang}_{\mathcal{C}}} % full convergence equivalence

%Semantics:
\newcommand{\rank}[1]{\mathsf{rk}(#1)} % rank of something
\newcommand{\rrank}[1]{\mathsf{rrk}(#1)} % right rank of an implicative formula
\newcommand{\lrank}[1]{\mathsf{lrk}(#1)} % left rank of an implicative formula
%\newcommand{\termin}[1]{\mathsf{t}(#1)} % set of terminals of a set of formulas
\newcommand{\termin}[3]{\mathsf{tmn}_{#1}^{#2}(#3)} % set of terminals of a set of formulas. The first argument is a tuple of terms to be substituted for the tuple of variables given in the second argument. The third argument is the formula of which we take the terminals 
\newcommand{\conc}[1]{\mathsf{cn}(#1)} % set of premisses of a set of formulas
\newcommand{\prem}[3]{\mathsf{pr}_{#1}^{#2}(#3)} % set of premisses of a set of formulas
\newcommand{\premp}[1]{\mathsf{pp}(#1)} % special premisses of premisses of a set of formulas
\newcommand{\premterm}[3]{\mathsf{prt}_{#1}^{#2}(#3)} % set of premisses having terminals in common with set of formulas #1
\newcommand{\spnex}[1]{\overline{#1}} % semi-prenex form of the formula #1
\newcommand{\ospnex}[1]{\overline{\overline{#1}}} % semi-prenex form of the formula #1 deprived of all universal quantifiers at the front
\newcommand{\forant}{\mathsf{uqa}} % one step semi-prenex form of the formula #1
\newcommand{\wrap}[1]{\bar{#1}} % wrapping of a term
\newcommand{\len}{\ell}
\newcommand{\trm}[1]{#1^{\textrm{--}}}
\newcommand{\cont}[2]{#1^{+}\hole{#2}}
\newcommand{\Mfin}[1]{\mathcal{M}_{\mathrm{f}}(#1)}
\newcommand{\mcup}{\uplus}
\newcommand{\mmcup}{\bar{\mcup}}
% \newcommand{\Pair}[2]{\langle{#1},{#2}\rangle}
\newcommand{\Rel}{\mathbf{REL}} %category of sets and relations
\newcommand{\MRel}{\mathbf{REL}_{\bang}} %Kleisli category of sets and relations
\newcommand{\Inf}{\mathbf{Inf}} %category of information system and approx rels
\newcommand{\SD}{\mathbf{SD}} %category of Scott domains and continuous functions
\newcommand{\CPO}{\mathbf{CPO}} %category of CPOs and continuous functions
\newcommand{\SL}{\mathbf{ScottL}} %category of preorders
\newcommand{\SLb}{\mathbf{ScottL}_{\bang}} %Kleisli category of \SL
\newcommand{\Coh}{\mathbf{Coh}} %category of coherent spaces
\newcommand{\Cohb}{\mathbf{Coh}_{\bang}} %Kleisli category of \Coh


\newcommand{\otspam}{
\mathrel{\vcenter{\offinterlineskip
\vskip-.130ex\hbox{\begin{turn}{180}$\mapsto$\end{turn}}}}} % reversed mapsto

\newcommand{\envup}[3]{#1[#2 \otspam #3]} % environment update

\newcommand{\try}[2]{\mathsf{try}_{#1}\{#2\}} % execute the second argument first argument until the second one is found
\newcommand{\catch}[2]{\mathsf{catch}_{#1}\{#2\}} % when the first argument is found, execute the second one

\newcommand{\Lamex}{\Lambda_{\mathsf{ex}}} % lambda calculus with try and catch

\renewcommand{\iff}{\Leftrightarrow}
\newcommand{\imp}{\Rightarrow}
\newcommand{\Apex}[1]{^{\: #1}}

\newcommand{\compl}[1]{{#1}^c} % complement of a set
\newcommand{\pts}{.\,.\,} % points abbreviated
%\newcommand{\conv}[1]{{#1}\!\downarrow} % covergence
\newcommand{\convh}[1]{{#1}\!\downarrow_h} % head covergence
\newcommand{\solv}[1]{#1\lightning} % solvance
\newcommand{\solvo}[1]{#1\lightning_o} % outer solvance
\newcommand{\module}[1]{\bool\langle #1 \rangle}

\newcommand{\Ide}[1]{Ide(#1)} % set of all ideals of a preorder

\newcommand{\Bstk}{\bB_{\mathsf{s}}} % quasi-boolean algebra of saturated sets of stacks
\newcommand{\fsubseteq}{\subseteq_\mathrm{f}} % finite subset
\newcommand{\Ps}[1]{\cP(#1)} % powerset
\newcommand{\Pss}[1]{\cP_\mathrm{s}(#1)} % set of all saturated subsets
\newcommand{\Psc}[1]{\cP_\mathrm{c}(#1)} % set of all closed subsets
\newcommand{\Psg}[1]{\cP_\mathrm{g}(#1)} % set of all good subsets
\newcommand{\Psf}[1]{\cP_\mathrm{f}(#1)} % set of all finite subsets
\newcommand{\Ms}[1]{\cM(#1)} % set of all multisets
\newcommand{\Msf}[1]{\cM_\mathrm{f}(#1)} % set of all finite multisets
\newcommand{\fst}{\mathsf{fst}} % reduction proper to the \Lambda\mu-calculus
\newcommand{\cons}{::} % stack constructor
\newcommand{\at}{\!\centerdot} % stack constructor (cons)
\newcommand{\ats}{\at\ldots\at} % stack constructor (cons) with lower suspension dots 
%\newcommand{\at}{\!::\!} % stack constructor

%\newcommand{\meet}{\} % inf operator
%\newcommand{\join}{\!\centerdot} % inf operator

\newcommand{\sps}[3]{\bgp^{(#1,#2,#3)}} % special stack defined as \overbrace{\cadr{#1}{0}\at\ldots\at\cadr{#1}{0}}^{#3 \mbox{ times}}\at #1
\newcommand{\spt}[1]{\bA^{(#1)}} % special term defined as \bd\epsilon.\cadr{\gd}{0}\ap(\cadr{\epsilon}{0}\at\ldots\at\cadr{\epsilon}{q-1}}\at\cddr{\epsilon}{q})

% \newcommand{\cdr}[1]{\mathsf{cdr}(#1)} % tail of stack
% \newcommand{\car}[1]{\mathsf{car}(#1)} % head of stack
% \newcommand{\itcdr}[2]{#1[#2)} % iterated tail of stack
% \newcommand{\cddr}[2]{#1[#2)} % iterated tail of stack
% \newcommand{\cadr}[2]{#1[#2]} % head of an iterated tail of stack

\newcommand{\op}{\mathsf{op}} % generic binary infix operator
\newcommand{\fun}[1]{\mathsf{f}(#1)} % generic unary function symbol
\newcommand{\nil}{\mathsf{nil}} % empty stack
\newcommand{\mcddr}[2]{\mathsf{cdr}^{#1}(#2)} % modified iterated tail of stack
\newcommand{\mitcar}[2]{\mathsf{car}^{#1}(#2)} % modified iterated head of stack
\newcommand{\mitcdr}[2]{\mathsf{cdr}^{#1}(#2)} % modified iterated tail of stack

\newcommand{\callcc}{\mathsf{cc}} % Felleisen's call/cc
\newcommand{\kpi}[1]{\mathsf{k}_{#1}} % Krivine's term that restores the stack
\newcommand{\nf}[1]{\mathsf{Nf}(#1)} % partial function returning the normal form 
\newcommand{\onf}[1]{\mathsf{Onf}(#1)} % partial function returning the outer normal form
\newcommand{\eonf}[1]{\eta\mathsf{Onf}(#1)} % partial function returning the extensional outer normal form
\newcommand{\hnf}[1]{\mathsf{Hnf}(#1)} % partial function returning the beta-head normal form
\newcommand{\ehnf}[1]{\eta\mathsf{Hnf}(#1)} % partial function returning the beta-eta head normal form
\newcommand{\Sol}{\mathsf{Sol}^{\mathsf{t}}} % set of all solvable terms
%\newcommand{\USol}{\mathsf{Sol}^{\mathsf{t}} % set of all solvable terms
\newcommand{\SetBT}{\mathfrak{B}} % set of all Bohm trees
\newcommand{\SetBTt}{\mathfrak{B}^{\mathsf{t}}} % set of all Bohm trees of \stk-terms
\newcommand{\BT}[1]{\mathsf{BT}(#1)} % Bohm tree of an expression
\newcommand{\tBT}[2]{\mathsf{BT}_{#2}(#1)} % truncated Bohm tree of an expression
\newcommand{\eBT}[1]{\eta\mathsf{BT}(#1)} % extensional Bohm tree of an expression
\newcommand{\teBT}[2]{\eta\mathsf{BT}_{#2}(#1)} % truncated extensional Bohm tree of an expression
\newcommand{\bdom}[2]{\mathsf{dom}(#1,#2)} % bounded domain of a term seen as a function over sequences of natural numbers
%\newcommand{\virt}[2]{\langle #1 \mid #2 \rangle} % virtual extension of the map corresponding to a term 
\newcommand{\bout}[3]{\mbox{\boldmath{$\langle$}} #1 \!\mid\! #2 \!\mid\! #3 \mbox{\boldmath{$\rangle$}}} % Bohm out term corresponding to a term #1, the sequence #2 , the bound #3 and the width #4
\newcommand{\vbout}[3]{\mbox{\boldmath{$\langle$}} #1 \!\mid\! #2 \!\mid\! #3 \mbox{\boldmath{$\rangle$}}} % virtual Bohm out term corresponding to a term #1, the sequence #2 and the bound #3 
\newcommand{\virt}[1]{\mathsf{vir}(#1)} % set of sequences that belong virtually to the map corresponding to a term
\newcommand{\bvirt}[2]{\mathsf{vir}(#1,#2)} % set of sequences that belong virtually to the map corresponding to a term, with a bound on their length
\newcommand{\extr}[1]{\mathsf{extr}(#1)} % extensionally reachable sequences
\newcommand{\uns}[1]{\mathsf{uns}(#1)} % unsolvable sequences
\newcommand{\unr}[1]{\mathsf{unr}(#1)} % unreachable sequences
\newcommand{\eqty}{\stackrel{\infty}{=}} % equality of Bohm trees up to infinite eta-expansion
\newcommand{\simty}{\stackrel{\infty}{\sim}} % similarity at all sequences of natural numbers
\newcommand{\pexp}[2]{\mbox{\boldmath{$\langle$}} #1 \lVert #2 \mbox{\boldmath{$\rangle$}}} % path expansion of a term
\newcommand{\Seq}{Seq} % the set of finite sequences of strictly positive natural numbers
\newcommand{\tSeq}[1]{Seq_{\leq #1}} % the set of finite sequences of length less or equal to a specified bound
\newcommand{\Lab}{Lab} % the set of labels of Bohm trees

\newcommand{\NT}[1]{\mathsf{NT}(#1)} % Nakajima tree of an expression
\newcommand{\tNT}[2]{\mathsf{NT}_{#2}(#1)} % truncated Nakajima tree of an expression

\newcommand{\sub}[2]{\{#1/#2\}} % classical substitution of #1 for #2
\newcommand{\ab}[1]{\mathcal{A}(#1)} % abort of a term
\newcommand{\ctrl}[1]{\mathcal{C}(#1)} % control of a term
\newcommand{\cmd}[2]{\langle #1 \lVert #2\rangle} % command constructor for lambda mu-mu-tilde
\newcommand{\ap}{\star} % application symbol of a term to a process
\newcommand{\bd}{\kappa} % binder for stack variables
\newcommand{\lambdab}{\bar{\lambda}} % lambda bar of mu-mu tilde calculus
\newcommand{\mut}{\tilde{\mu}} % binder for mu tilde calculus
\newcommand{\tcbn}[1]{#1^{\circ}} % translation of the cbn lambdamumu expressions into stack expressions 
\newcommand{\tcbv}[1]{#1^{\bullet}} % translation of the cbv lambdamumu expressions into stack expressions 
\newcommand{\texp}[1]{#1^{\circ}} % translation of the lambdamu expressions into stack expressions
\newcommand{\ttyp}[1]{#1^{\circ}} % translation of the lambdamu types into stack types
\newcommand{\Tp}[1]{#1^{\circ}} % translation
\newcommand{\Te}[1]{#1^{\circ}} % translation of the lambdamu expressions into stack expressions
\newcommand{\Neg}[1]{#1^{-}} % negative translation of formulas
\newcommand{\Pos}[1]{#1^{+}} % positive translation of formulas
\newcommand{\Tt}[1]{#1^{-}} % translation of stack expressions into lambda calculus with pairing
\newcommand{\Ts}[1]{#1^{+}} % translation of the lambda mu calculus into the lambda calculus
\newcommand{\dev}[1]{#1^{\baro}} % inner-outer development of the redexes of #1
\newcommand{\AtForm}{\mathrm{AtFm}} % set of atomic formulas of first-order logic
\newcommand{\UqAtForm}{\mathrm{UqAtFm}} % the set of universally quantified atomic formulas of first-order logic
\newcommand{\UqBot}{\mathrm{UqBot}} % the set of universally quantified atomic formulas of first-order logic in which the atomic formula is $\bot$
\newcommand{\Form}{\mathrm{Fm}} % set of formulas of second-order logic
\newcommand{\cForm}{\mathrm{Fm}^\mathsf{o}} % set of closed formulas of second-order logic
\newcommand{\Val}[1]{\mathrm{Val}_{#1}} % set of valuations into the structure #1
\newcommand{\At}{\mathrm{At}} % atomic formulas
\newcommand{\cAt}{\mathrm{At}^\mathsf{o}} % closed atomic formulas
\newcommand{\Var}{\mathrm{Var}} % set of variables
\newcommand{\Nam}{\mathrm{Nam}} % set of names
\newcommand{\FV}{\mathrm{FV}} % free variables
\newcommand{\FN}{\mathrm{FN}} % free names
\newcommand{\LTer}[1]{\Lambda^{\mathsf{#1}}} % set of terms of the lambda-mu calculus
\newcommand{\LTyp}[1]{\cT_{\lambda\mu}^{\mathsf{#1}}} % set of types of the lambda-mu calculus
\newcommand{\ITer}[1]{\Sigma_{\mathsf{in}}^{\mathsf{#1}}} % set of intuitionistic terms of the stack calculus
\newcommand{\BTer}[1]{\Sigma_{\mathsf{b}}^{\mathsf{#1}}} % finite Bohm trees of the stack calculus
\newcommand{\KTer}[1]{\Sigma^{\mathsf{#1}}} % set of terms of the stack calculus
\newcommand{\KTyp}[1]{\cT_\bd^{\mathsf{#1}}} % set of types of the stack calculus

%\newcommand{\Kstate}[4]{\langle({#1},{#2}),({#3},{#4})\rangle} % a state of the Krivine Abstract Machine involving a term 
\newcommand{\transition}{\longrightarrow} % transition symbol from one state of the Krivine Abstract Machine to another
\newcommand{\Kstate}[3]{\mbox{\boldmath{$\langle$}} \ {#1},{#2},{#3} \ \mbox{\boldmath{$\rangle$}}} % a state of the Krivine Abstract Machine involving a term 

\newcommand{\Kproc}[2]{\langle{#1},{#2}\rangle} % a state of the Krivine Abstract Machine involving a process
\newcommand{\Kclos}[2]{({#1},{#2})} % a closure of the Krivine Abstract Machine

\newcommand{\SN}[1]{\mathrm{SN}^{\mathsf{#1}}} % set of strongly normalizing expressions of the stack calculus

%\newcommand{\deg}[2]{\mathsf{deg}_{#1}(#2)} % degree of a variable in a n expression

\newcommand{\dgr}[2]{\mathsf{deg}_{#1}(#2)} % degree of a variable in a n expression

\newcommand{\bbot}{
\mathrel{\vcenter{\offinterlineskip
\vskip-.130ex\hbox{\begin{turn}{90}$\models$\end{turn}}}}} % Krivine's double bottom

\newcommand{\ttop}{
\mathrel{\vcenter{\offinterlineskip
\vskip-.130ex\hbox{\begin{turn}{270}$\models$\end{turn}}}}} % double top

\newcommand{\sepa}{
\mathrel{\vcenter{\offinterlineskip
\vskip-.130ex\hbox{\begin{turn}{90}$\succ$\end{turn}}}}} % separability

\newcommand{\asm}{\! : \!} % separator for type assumptions in contexts
\newcommand{\tass}{:} % separator type assignment in judgements

\newcommand{\tval}[1]{\vert #1\vert} % truth value interpretation of types into sets of terms
\newcommand{\fval}[1]{\lVert #1 \rVert} % falsehood value interpretation of types into sets of stacks
\newcommand{\tInt}[1]{\vert #1\vert} % truth value-like interpretation of term types into set of terms
\newcommand{\sInt}[1]{\vert #1\vert} % falsehood value-like interpretation of stack types into set of stacks
\newcommand{\pInt}[1]{\vert #1 \vert} % interpretation of the process type into a set of processes
\newcommand{\eInt}[1]{\vert #1 \vert} % interpretation of the expression type into a set of expressions
\newcommand{\Int}[1]{\llbracket #1\rrbracket} % interpretation of expressions in a mathematical domain
\newcommand{\id}{\mathsf{id}} % identity morphsism in a category
\newcommand{\pr}[1]{\mathsf{pr}_{#1}} % i-th projection of a cartesian product
\newcommand{\ev}{\mathsf{ev}} % evaluation morphism of a ccc

%\newcommand{\list}[1]{\langle #1 \rangle} % list constructor write inside the arguments separated by commas
\newcommand{\lis}[1]{\prec #1 \succ} % list constructor
\newcommand{\copair}[2]{[ #1, #2 ]} % copair constructor

\newcommand{\cur}[1]{\Lambda(#1)} % currying natural isomorphism
\newcommand{\invcur}[1]{\Lambda^{-1}(#1)} % inverse of currying natural isomorphism
\newcommand{\adbmaL}{
\mathrel{\vcenter{\offinterlineskip
\vskip-.100ex\hbox{\begin{turn}{180}$\Lambda$\end{turn}}}}}

\newcommand{\ctrliso}[1]{\phi(#1)} % natural isomorphism proper to control categories
\newcommand{\invctrliso}[1]{\phi^{-1}(#1)} % inverse of the natural isomorphism proper to control categories

\newcommand{\cocur}[1]{\adbmaL\!\!(#1)} % co-currying natural isomorphism
\newcommand{\invcocur}[1]{\adbmaL^{-1}(#1)} % inverse of co-currying natural isomorphism

\newcommand{\cord}{\sqsubseteq_c} % computational order on Bohm trees
\newcommand{\lord}{\sqsubseteq_l} % logical order on Bohm trees

\newcommand{\coher}{\stackrel{\frown}{\smile}} % Girard's coherence relation
\newcommand{\scoher}{\frown} % Girard's strict coherence relation

\newcommand{\Cl}[1]{Cl(#1)} % the set of cliques of a set 

\newcommand{\ccl}{\ensuremath{CCL}} % name of classical combinatory logic
\newcommand{\lmuo}{\ensuremath{\lambda\mu\mathbf{1}}} % name of Andou's lambda-mu calculus 
\newcommand{\lmc}{\ensuremath{\lambda C}} % name of Herbelin-De Groote's lambda-C calculus 
\newcommand{\lamb}{\ensuremath{\lambda}} % name of Church's lambda calculus 
\newcommand{\lmu}{\ensuremath{\lambda\mu}} % name of Parigot's lambda-mu calculus 
\newcommand{\stk}{\ensuremath{\bd}} % name of the stack calculus
\newcommand{\stke}{\ensuremath{\bd\eta}} % name of the extensional stack calculus
\newcommand{\stkw}{\ensuremath{\bd w}} % name of the stack calculus + weta
\newcommand{\lsp}{\ensuremath{\lambda\mathsf{sp}}} % name of the lambda calculus with surjective pairing
\newcommand{\lesp}{\ensuremath{\lambda\eta\mathsf{sp}}} % name of the extensional lambda calculus with surjective pairing
\newcommand{\ort}[1]{#1^{\bot}} % orthogonal object

\newcommand{\wi}{\binampersand} % with connective
\newcommand{\pa}{\bindnasrepma} % par connective

\newcommand{\te}{\mathsf{ten}} % tensor morphism
\newcommand{\parm}{\mathsf{par}} % par morphism

\newcommand{\mon}{\mathsf{m}} % monoidality morphism
\newcommand{\see}{\mathsf{s}} % seely isomorphism
\newcommand{\ut}{\mathsf{t}} % terminal morphism in a Cartesian category

\newcommand{\assoc}{\mathsf{ass}} % generalized associativity morphism

\newcommand{\der}{\mathsf{der}} % dereliction morphism
\newcommand{\coder}{\mathsf{cod}} % codereliction morphism

\newcommand{\coa}{\mathsf{h}} % coalgebra for the functor \ort{(\cdot)} \xrightarrow{\cdot}

\newcommand{\con}{\mathsf{con}} % contraction morphism
\newcommand{\wkn}{\mathsf{wkn}} % weakening morphism
\newcommand{\cowkn}{\mathsf{cow}} % coweakening morphism

\newcommand{\nco}{\overline{\mathsf{con}}} % negative contraction morphism
\newcommand{\nwk}{\overline{\mathsf{wkn}}} % negative weakening morphism

\newcommand{\dig}{\mathsf{dig}} % digging morphism

%\newcommand{\codig}{\mathsf{cod}} % codigging morphism

\newcommand{\dual}{\partial} % dualizing morphism
\newcommand{\ddual}{\partial^{-1}} % inverse of the dualizing morphism

\newcommand{\teid}{\mathbf{1}} % identity of the tensor product
\newcommand{\bon}{\mathbf{1}} % identity of the tensor product 

\newcommand{\DEC}{\mathsf{DEC}} % the class of decidable languages
\newcommand{\SDEC}{\mathsf{SDEC}} % the class of semi-decidable languages
\newcommand{\REG}{\mathsf{REG}} % the class of regular languages
\newcommand{\CFL}{\mathsf{CFL}} % the class of context-free languages
\newcommand{\dCFL}{\mathsf{dCFL}} % the class of deterministically context-free languages

\newcommand{\sqb}[1]{[#1]} % square brackets

\newcommand{\cnt}[1]{#1^\bullet} % center of a control category
\newcommand{\foc}[1]{#1^\sharp} % focus of a control category

\newcommand{\com}{\mathsf{comp}} % composition proof
\newcommand{\exc}{\mathsf{exc}} % exchange proof

\newcommand{\ax}{\mathsf{ax}} % axiom rule
\newcommand{\dne}{\mathsf{dne}} % double negation elimination rule
\newcommand{\raa}{\mathsf{raa}} % reductio ad absurdum rule
\newcommand{\efq}{\mathsf{efq}} % ex flaso quodlibet rule
\newcommand{\cut}{\mathsf{cut}} % cut rule
\newcommand{\dni}{\mathsf{dni}} % double negation introduction proof
\newcommand{\mpo}{\mathsf{mp}} % modus ponens rule
\newcommand{\dt}{\mathsf{dt}} % deduction theorem
\newcommand{\idem}{\mathsf{id}} % identity proof
\newcommand{\contp}{\mathsf{contp}} % contraposition proof (positive)
\newcommand{\contn}{\mathsf{contn}} % contraposition proof (negative)
\newcommand{\sded}{\mathsf{sded}} % symmetric deduction proof

\newcommand{\varrule}{\mathsf{ax}} % inference rule for variables
\newcommand{\carrule}{\to e_r} % inference rule for \car
\newcommand{\cdrrule}{\to e_l} % inference rule for \cdr
\newcommand{\atrule}{\to i} % inference rule for \at
\newcommand{\aprule}{\mathsf{cut}} % inference rule for \app
\newcommand{\nilrule}{\bot i} % inference rule for \nil
\newcommand{\bdrule}[1]{\bd,{#1}} % inference rule for \bd with reference to the bound variable 
\newcommand{\orrule}{\vee i} % inference rule for or
\newcommand{\rsallrule}{2\forall r} % inference rule for right introduction of the second order universal quantifier
\newcommand{\rfallrule}{\forall r} % inference rule for right introduction of the first order universal quantifier
\newcommand{\lsallrule}{2\forall l} % inference rule for left introduction of the second order universal quantifier
\newcommand{\lfallrule}{\forall l} % inference rule for left introduction of the first order universal quantifier

\newcommand{\leng}[1]{\sharp #1} % length of a sequence

%\newcommand{\wid}[1]{\mathsf{w}(#1)} % width of a term
%\newcommand{\bwid}[2]{\mathsf{w}(#1,#2)} % bounded width of a term
\newcommand{\wei}[1]{\mathsf{w}(#1)} % weight of a term
\newcommand{\bwei}[2]{\mathsf{w}(#1,#2)} % bounded weight of a term
\newcommand{\brea}[1]{\mathsf{b}(#1)} % breadth of a term
\newcommand{\bbrea}[2]{\mathsf{b}(#1,#2)} % bounded breadth of a term
\newcommand{\gap}[1]{\mathsf{g}(#1)} % gap of a term
\newcommand{\bgap}[2]{\mathsf{g}(#1,#2)} % bounded gap of a term

\newcommand{\wnot}{?} % why not modality
\newcommand{\bang}{!} % bang modality
\newcommand{\bbang}{!!} % double bang functor
\newcommand{\app}{F} % morphism from $U \to U \Rightarrow U$
\newcommand{\lam}{G} % morphism from $U \Rightarrow U \to U$
%\newcommand{\cur}{\Lambda} % currying
\newcommand{\cld}{\downarrow\!} % down arrow closure operator
\newcommand{\clu}{\uparrow\!} % up arrow closure operator
\newcommand{\clo}[1]{\overline{#1}} % overline closure operator
\newcommand{\clde}{\downarrow_\eta\!} % closure operator for the extensionality preorder
\newcommand{\parcl}[1]{\uparrow_{#1}\!} % parameterized closure operator
\newcommand{\cldn}[2]{\downarrow_{#1}\!{#2}} % downwards closure operator
\newcommand{\clup}[2]{\uparrow_{#1}\!{#2}} % upwards closure operator
\newcommand{\opp}[1]{{#1}^{\mathsf{op}}} % opposite

%Macro for stack sequents. The forms of annotated sequents are 
%\tystk{s}{stack}{stack-type}{context}
%\tystk{t}{term}{term-type}{context}
%\tystk{p}{process}{process-type}{context}
%Sequents without annotations
%\tystk{s}{}{stack-type}{context}
%\tystk{t}{}{term-type}{context}
%\tystk{p}{}{}{context}
\newcommand{\tystk}[4]{%
\ifthenelse{\equal{#1}{s}\OR\equal{#1}{t}}{
	\ifthenelse{\equal{#1}{s}}{%\equal{#1}{s}
		\ifthenelse{\isempty{#2}}{#3 \vdash #4}{\textcolor{blue}{#2} \textcolor{blue}{:} #3 \vdash #4}
		}{%\equal{#1}{t}
		\ifthenelse{\isempty{#2}}{\vdash #3, #4}{\vdash \textcolor{blue}{#2} \textcolor{blue}{:} #3 \ \textcolor{blue}{\mid} \ #4}
		}
	}{%\equal{#1}{p}
	\ifthenelse{\isempty{#2}}{\vdash #4}{\vdash \textcolor{blue}{#2} \textcolor{blue}{\mid} #4}
	}
}
\newcommand{\ntystk}[4]{%
\ifthenelse{\equal{#1}{s}\OR\equal{#1}{t}}{
	\ifthenelse{\equal{#1}{s}}{%\equal{#1}{s}
		\ifthenelse{\isempty{#2}}{#3 \nvdash #4}{\textcolor{blue}{#2} \textcolor{blue}{:} #3 \nvdash #4}
		}{%\equal{#1}{t}
		\ifthenelse{\isempty{#2}}{\nvdash #3, #4}{\nvdash \textcolor{blue}{#2} \textcolor{blue}{:} #3 \ \textcolor{blue}{\mid} \ #4}
		}
	}{%\equal{#1}{p}
	\ifthenelse{\isempty{#2}}{\nvdash #4}{\nvdash \textcolor{blue}{#2} \textcolor{blue}{\mid} #4}
	}
}

%Macro for lambda-mu sequents. The form is 
%\tylmu{left_context}{expression}{expression-type}{right_context}
\newcommand{\tylmu}[4]{
	\ifthenelse{\isempty{#3}}
	{%if
	\ifthenelse{\isempty{#4}}
		{#1 \vdash_{\lmu} \textcolor{blue}{#2}}
		{#1 \vdash_{\lmu} \textcolor{blue}{#2} \mid #4}
	}
	{%else
        \ifthenelse{\isempty{#4}}
        	{#1 \vdash_{\lmu} \textcolor{blue}{#2} \textcolor{blue}{:} #3}
        	{#1 \vdash_{\lmu} \textcolor{blue}{#2} \textcolor{blue}{:} #3 \textcolor{blue}{\mid} #4}
	}
}

%Macros for lambda-mu-mu-tilde sequents.
 
%\tylmmcom{command}{left_context}{right_context}
\newcommand{\tylmmcom}[3]{\textcolor{blue}{#1}\ \textcolor{blue}{\triangleright} #2 \vdash #3}
%\tylmmter{left_context}{term}{right_active_formula}{right_context}
\newcommand{\tylmmter}[4]{#1 \vdash \textcolor{blue}{#2} \textcolor{blue}{:} #3 \mid #4}
%\tylmmenv{left_context}{environment}{left_active_formula}{right_context}
\newcommand{\tylmmenv}[4]{#1 \mid \textcolor{blue}{#2} \textcolor{blue}{:} #3 \vdash #4}

%Macro for lambda-mu-one sequents. The form is 
%\tylmuo{left_context}{expression}{expression-type}
%\newcommand{\tylmuo}[3]{#1 \vdash_{\lmuo} \textcolor{blue}{#2} \textcolor{blue}{:} #3}
\newcommand{\tylmuo}[3]{#1 \vdash \textcolor{blue}{#2} \textcolor{blue}{:} #3}

%Macro for lambda sequents, i.e. typed lambda terms. The form is 
%\tylamb{left_context}{expression}{expression-type}
\newcommand{\tylamb}[3]{#1 \vdash_{\lamb} \textcolor{blue}{#2} \textcolor{blue}{:} #3}

%Macro for lambda-c sequents. The form is 
%\tylmc{left_context}{expression}{expression-type}
\newcommand{\tylmc}[3]{#1 \vdash_{\lmc} \textcolor{blue}{#2} \textcolor{blue}{:} #3}

%Macro for ccl sequents. The form is 
%\tyccl{left_context}{expression}{expression-type}
\newcommand{\tyccl}[3]{#1 \vdash_{\ccl} \textcolor{blue}{#2} \textcolor{blue}{:} #3}

\newcommand{\prov}[2]{#1 \vdash #2} % provability symbol
\newcommand{\refu}[2]{#1 \dashv #2} % refutation symbol


%%%%%%%% MACRO PER LE NOTE DEL CORSO DI CALCOLABILITA'

\newcommand{\eclose}[1]{\mathsf{ecl}(#1)} % operatore di epsilon-chiusura
\newcommand{\zr}{\mathsf{Z}} % the constantly zero function
\newcommand{\suc}{\mathsf{S}} % the successor function
\newcommand{\pred}{\mathsf{P}} % the predecessor function
\newcommand{\prj}[2]{I_{#1}^{#2}} % the projection function
\newcommand{\ca}[1]{\mathsf{c}_{#1}} % the characteristic function of a predicate
\newcommand{\minus}{\stackrel{\centerdot}{-}} % the minus function on natural numbers
\newcommand{\conv}[1]{{#1}\!\downarrow} % convergence of a function
\newcommand{\dive}[1]{{#1}\!\uparrow} % divergence of a function
\newcommand{\PR}{\mathbf{PR}} % partial recursive functions
\newcommand{\REC}{\mathbf{REC}} % total recursive functions
\newcommand{\PRIMREC}{\mathbf{PrimREC}} % primitive recursive functions
\newcommand{\RESET}{\Sigma} % set of all recursively enumerable sets
\newcommand{\RECSET}{\Delta} % set of all recursive sets
\newcommand{\sse}{\iff}
\newcommand{\bforall}[2]{\forall{#1}\!<\!{#2}} % bounded universal quantification
\newcommand{\bexists}[2]{\exists{#1}\!<\!{#2}} % bounded existential quantification
\newcommand{\bmu}[2]{\mu{#1}\!<\!{#2}} % bounded mu-recursion
\newcommand{\fprim}[1]{\mathsf{p}(#1)} % function returning the n-th prime number
\newcommand{\pprim}[1]{\mathsf{prim}(#1)} % predicate testing primality of a number
\newcommand{\expn}[2]{\mathsf{exp}(#1,#2)} % function returning the exponent of #1 in the unique prime decomposition of #2

%\newcommand{\nat}{\mathbb{N}}
%\newcommand{\pair}[2]{\langle #1,#2 \rangle}
\newcommand{\fset}[1]{\sharp(#1)}
%\newcommand{\Pf}[1]{\mathcal{P}_{\mathrm{f}}(#1)}
%\newcommand{\st}{:}
% \newcommand{\seq}[1]{\vec{#1}}
\newcommand{\gramm}{\mathrel{::=}} % EBNF grammar definition
\newcommand{\ass}{\mathrel{:=}} % syntactical definition 
\newcommand{\nat}{\mathbb{N}} % set of natural numbers
\newcommand{\st}{:} % set constructor
% \newcommand{\ass}{:=} % assignment
\newcommand{\car}[1]{\mathsf{c}_{#1}} % characteristic function
\newcommand{\ran}[1]{\mathsf{ran}(#1)} % range of a function
\newcommand{\dom}[1]{\mathsf{dom}(#1)} % domain of a function
\newcommand{\secod}[1]{\prec\! #1 \!\succ} % code of a sequence
\newcommand{\seq}[1]{\prec\! #1 \!\succ} % code of a sequence
\newcommand{\sedecod}[2]{(#1)_{#2}} % extract the #2-th element of the sequence with code #1
\newcommand{\pair}[2]{\langle #1,#2 \rangle} % coding of pairs
\newcommand{\gph}[1]{\mathsf{gr}(#1)} % graph of a function

\maketitle
%\tableofcontents

%%%%%%%%%%%%%%%%%%%%%%%%%%%%%%%%%%%%%%%%%%%%
\section{Automi a stati finiti deterministici e non}
%%%%%%%%%%%%%%%%%%%%%%%%%%%%%%%%%%%%%%%%%%%%

Tra gli automi a stati finiti si distinguono due categorie: quelli \emph{deterministici} (DFA) e quelli \emph{non-deterministici} (NFA). I primi sono un caso particolare dei secondi.

\begin{definition}[DFA]\label{def:DFA}
Un automa a stati finiti deterministico (DFA, in breve) \`{e} una quintupla $\cA = (Q,\Sigma,\delta,q_0,F)$ dove 
\begin{itemize}
\item $Q$ \`{e} l'insieme finito degli stati,
\item $\Sigma$ \`{e} l'alfabeto di input,
\item $\delta: Q \times \Sigma \hookrightarrow Q$ \`{e} la funzione (parziale) di transizione,
\item $q_0 \in Q$ \`{e} lo stato iniziale,
\item $F \subseteq Q$ \`{e} l'insieme degli stati finali.
\end{itemize}
\end{definition}

La prossima tappa \`{e} definire il linguaggio accettato da un automa a stati finiti deterministico. A tal fine dobbiamo precisare cosa significa dire che una stringa \`{e} accettata da un tale automa.

\begin{definition}[Funzione di transizione estesa]\label{def:trans-est}
Dato un DFA $\cA = (Q,\Sigma,\delta,q_0,F)$ definiamo la sua funzione di transizione estesa $\hat{\delta}: Q \times \Sigma^* \hookrightarrow Q$ per induzione sulla lunghezza delle stringhe in input nella maniera seguente:
$$ \hat{\delta}(q,\epsilon) = q ; \qquad \qquad \hat{\delta}(q,xa) = \delta(\hat{\delta}(q,x),a) $$
\end{definition}

\begin{definition}[Linguaggio di un DFA]\label{def:lang-acc}
Sia $\cA = (Q,\Sigma,\delta,q_0,F)$ un DFA. Diciamo che una stringa $w \in \Sigma^*$ \`{e} accettata da $\cA$ sse $\hat{\delta}(q_0,w) \in F$. Il linguaggio accettato da $\cA$, indicato con $\cL(\cA)$, \`{e} l'insieme delle stringhe accettate da $\cA$, ovvero $\cL(\cA) = \{w \in \Sigma^* \st \hat{\delta}(q_0,w) \in F\}$.
\end{definition}

\textbf{Notazione.} Indichiamo con $\Ps{Q}$ l'insieme delle parti di $Q$, ovvero l'insieme di tutti i sottoinsiemi di $Q$. Inoltre indichiamo con $\Psf{Q}$ l'insieme di tutti i sottoinsiemi finiti di $Q$.

\begin{definition}[NFA]\label{def:NFA}
Un automa a stati finiti non-deterministico (NFA, in breve) \`{e} una quintupla $\cA = (Q,\Sigma,\delta,q_0,F)$ dove 
\begin{itemize}
\item $Q$ \`{e} l'insieme finito degli stati,
\item $\Sigma$ \`{e} l'alfabeto di input,
\item $\delta: Q \times \Sigma \to \Psf{Q}$ \`{e} la funzione di transizione,
\item $q_0 \in Q$ \`{e} lo stato iniziale,
\item $F \subseteq Q$ \`{e} l'insieme degli stati finali.
\end{itemize}
\end{definition}

Notiamo che per gli NFA la funzione di transizione $\delta$ \`{e} pensata come una funzione totale: infatti per uno stato $q$ non avente archi uscenti etichettati con input $a$ si pu\`{o} equivalentemente dire che $\delta(q,a) = \emptyset$.

Si pu\`{o} constatare direttamente che ogni DFA \`{e} un NFA in cui ogni valore $\delta(q,a)$ della funzione di transizione \`{e} un insieme di cardinalit\`{a} uno. Viceversa vi sono degli NFA che non sono dei DFA. Andiamo ora a generalizzare le definizioni \ref{def:trans-est} e \ref{def:lang-acc} al caso degli NFA.

\begin{definition}[Funzione di transizione estesa]\label{def:trans-est2}
Dato un NFA $\cA = (Q,\Sigma,\delta,q_0,F)$ definiamo la sua funzione di transizione estesa $\hat{\delta}: Q \times \Sigma^* \to \Psf{Q}$ per induzione sulla lunghezza delle stringhe nella maniera seguente:
$$ \hat{\delta}(q,\epsilon) = \{q\} ; \qquad \qquad \hat{\delta}(q,xa) = \bigcup_{p \in \hat{\delta}(q,x)} \delta(p,a) $$
\end{definition}

\begin{definition}[Linguaggio di un NFA]\label{def:lang-acc2}
Sia $\cA = (Q,\Sigma,\delta,q_0,F)$ un NFA. Diciamo che una stringa $w \in \Sigma^*$ \`{e} accettata da $\cA$ sse $\hat{\delta}(q_0,w) \cap F \neq \emptyset$. Il linguaggio accettato da $\cA$, indicato con $\cL(\cA)$, \`{e} l'insieme delle stringhe accettate da $\cA$, ovvero $\cL(\cA) = \{w \in \Sigma^* \st \hat{\delta}(q_0,w) \cap F \neq \emptyset\}$.
\end{definition}

%%%%%%%%%%%%%%%%%%%%%%%%%%%%%%%%%%%%%%%%%%%%
\section{Equivalenza di automi finiti deterministici e non}
%%%%%%%%%%%%%%%%%%%%%%%%%%%%%%%%%%%%%%%%%%%%

Abbiamo osservato che la classe degli NFA contiene propriamente quella dei DFA. Tuttavia in questa sezione dimostreremo che le due classi hanno la stessa potenza, ovvero che la classe dei linguaggi accettati dagli NFA coincide con quella dei linguaggi accettati dai DFA. Un verso di questa doppia inclusione \`{e} banale. Se $L$ \`{e} un linguaggio tale che esiste un DFA $\cA$ per cui $L = \cL(\cA)$, allora esiste anche un NFA $\cB$ tale che $L = \cL(\cB)$.

L'altra inclusione invece non \`{e} triviale e richiede una dimostrazione. A tale scopo si utilizza una particolare costruzione, chiamata la ``costruzione per sottoinsiemi" (o determinizzazione), che trasforma un NFA in un DFA preservando il linguaggio accettato dall'automa di partenza.

\begin{definition}[Determinizzazione]\label{def:determiniz}
Sia $\cA = (Q_\cA,\Sigma,\delta_\cA,q_0,F_\cA)$ un NFA. Definiamo un DFA $\cB = (Q_\cB,\Sigma,\delta_\cB,\{q_0\},F_\cB)$ come segue:
\begin{itemize}
\item $Q_\cB = \Ps{Q_\cA}$;
\item $F_\cB = \{S \subseteq Q_\cA \st S \cap F_\cA \neq \emptyset\}$;
\item $\delta_\cB(S,a) = \bigcup_{p \in S} \delta_\cA(p,a)$
\end{itemize}
\end{definition}

\begin{theorem}[Equivalenza]\label{thm:det-non-det}
Sia $\cA$ un NFA e sia $\cB$ il DFA della Definizione \ref{def:determiniz}. Allora $\cL(\cA) = \cL(\cB)$.
\end{theorem}

\begin{proof}
Sia $\cA = (Q_\cA,\Sigma,\delta_\cA,q_0,F_\cA)$ un NFA e sia $\cB = (Q_\cB,\Sigma,\delta_\cB,\{q_0\},F_\cB)$ definito come nella Definizione \ref{def:determiniz}. Dimostriamo per induzione sulla lunghezza di $w \in \Sigma^*$ che $\hat{\delta}_\cA(q_0,w)= \hat{\delta}_\cB(\{q_0\},w)$.

\noindent\textbf{Base.} La sola stringa di lunghezza $0$ \`{e} $\epsilon$ e per definizione abbiamo $\hat{\delta}_\cA(q_0,\epsilon) = \{q_0\} = \hat{\delta}_\cB(\{q_0\},\epsilon)$.

\noindent\textbf{Passo induttivo.} Sia $w$ una stringa di lunghezza $\geq 1$, ovvero $w = xa$. Per ipotesi induttiva abbiamo che $\hat{\delta}_\cA(q_0,x) = \hat{\delta}_\cB(\{q_0\},x)$. Quindi
\begin{eqnarray*}
\hat{\delta}_\cA(q_0,xa) & = & \bigcup_{p \in \hat{\delta}_\cA(q_0,x)} \delta_\cA(p,a) \qquad \mbox{, per la Def. \ref{def:trans-est2}.} \\
 & = & \delta_\cB(\hat{\delta}_\cA(q_0,x),a) \qquad \mbox{, per la Def. \ref{def:determiniz},}\\
 & = & \delta_\cB(\hat{\delta}_\cB(\{q_0\},x),a) \qquad \mbox{, per l'ipotesi induttiva,} \\
 & = & \hat{\delta}_\cB(\{q_0\},xa) \qquad \mbox{, per la Def. \ref{def:trans-est}.}
\end{eqnarray*}
Per concludere osserviamo che $\hat{\delta}_\cA(q_0,w) \cap F_\cA \neq \emptyset \iff \hat{\delta}_\cB(\{q_0\},w) \in F_\cB$, da cui risulta che $\cL(\cA) = \cL(\cB)$.
\qed\end{proof}

%%%%%%%%%%%%%%%%%%%%%%%%%%%%%%%%%%%%%%%%%%%%
\section{Automi a stati finiti non-deterministici con $\epsilon$-transizioni}
%%%%%%%%%%%%%%%%%%%%%%%%%%%%%%%%%%%%%%%%%%%%

\begin{definition}[$\epsilon$-NFA]\label{def:eps-NFA}
Un automa a stati finiti non-deterministico con $\epsilon$-transizioni ($\epsilon$-NFA, in breve) \`{e} una quintupla $\cA = (Q,\Sigma,\delta,q_0,F)$ dove 
\begin{itemize}
\item $Q$ \`{e} l'insieme finito degli stati,
\item $\Sigma$ \`{e} l'alfabeto di input (non contenente $\epsilon$),
\item $\delta: Q \times (\Sigma\cup \{\epsilon\}) \hookrightarrow \Psf{Q}$ \`{e} la funzione di transizione,
\item $q_0 \in Q$ \`{e} lo stato iniziale,
\item $F \subseteq Q$ \`{e} l'insieme degli stati finali.
\end{itemize}
\end{definition}

Si pu\`{o} constatare direttamente che ogni NFA \`{e} un $\epsilon$-NFA in cui ogni $\epsilon$ non compare mai come secondo argomento della funzione di transizione. Viceversa vi sono degli $\epsilon$-NFA che non sono degli NFA. Come fatto in precedenza, definiamo la funzione di transizione estesa di un $\epsilon$-NFA per poi definirne il linguaggio accettato.

Stavolta per\`{o} \`{e} richiesto un passo preliminare: la determinazione delle cosiddette $\epsilon$-chiusure. 

\begin{definition}[$\epsilon$-chiusura]\label{def:eps-chius}
Dato uno stato $q \in Q$, la sua $\epsilon$-chiusura $n$-esima, indicata con $\textsf{ecl}^n(q)$, \`{e} definita per induzione su $n$ come segue:
$$ \textsf{ecl}^0(q) = \{q\}; \qquad \textsf{ecl}^{n+1}(q) = \bigcup_{p \in \textsf{ecl}^{n}(q)} \delta(p,\epsilon) $$
Infine definiamo la $\epsilon$-chiusura $\eclose q$ di $q$ ponendo $\eclose q = \bigcup_{n \geq 0} \textsf{ecl}^{n}(q)$.
\end{definition}

Per comodit\`{a} estendiamo la definizione della $\epsilon$-chiusura ad un insieme di stati $S \subseteq Q$ ponendo $\eclose S = \bigcup_{p \in S} \eclose p$.

\begin{definition}[Funzione di transizione estesa]\label{def:trans-est3}
Dato un $\epsilon$-NFA $\cA = (Q,\Sigma,\delta,q_0,F)$ definiamo la sua funzione di transizione estesa $\hat{\delta}: Q \times \Sigma^* \hookrightarrow \Psf{Q}$ per induzione sulla lunghezza delle stringhe nella maniera seguente:
$$ \hat{\delta}(q,\epsilon) = \eclose q ; \qquad \qquad \hat{\delta}(q,xa) = \eclose{\bigcup_{p \in \hat{\delta}(q,x)} \delta(p,a)} $$
\end{definition}

A questo punto diciamo che una stringa $w \in \Sigma^*$ \`{e} accettata da $\cA$ sse $\hat{\delta}(q_0,w) \cap F \neq \emptyset$. Quindi $\cL(\cA)$ \`{e} definito esattamente come nella Definizione \ref{def:lang-acc2}.

Abbiamo osservato che la classe degli $\epsilon$-NFA contiene propriamente quella degli NFA. Tuttavia dimostreremo che le due classi hanno la stessa potenza, ovvero che la classe dei linguaggi accettati dagli $\epsilon$-NFA coincide con quella dei linguaggi accettati dagli NFA. Un verso di questa doppia inclusione \`{e} banale. Se $L$ \`{e} un linguaggio tale che esiste un DFA $\cA$ per cui $L = \cL(\cA)$, esiste anche un NFA (e quindi un $\epsilon$-NFA) $\cB$ tale che $L = \cL(\cB)$.

L'altra inclusione invece non \`{e} triviale e richiede una dimostrazione. A tale scopo si utilizza una particolare costruzione, chiamata la ``eliminazione delle $\epsilon$-transizioni", che trasforma un $\epsilon$-NFA in un DFA preservando il linguaggio accettato dall'automa di partenza. Questa costruzione ci d\`{a} il risultato che volevamo, poich\'{e} ogni DFA \`{e} anche un NFA.

\begin{definition}[Eliminazione delle $\epsilon$-transizioni]\label{def:epselim}
Sia $\cA = (Q_\cA,\Sigma,\delta_\cA,q_0,F_\cA)$ un $\epsilon$-NFA. Definiamo un DFA $\cB = (Q_\cB,\Sigma,\delta_\cB,\eclose{q_0},F_\cB)$ come segue:
\begin{itemize}
\item $Q_\cB = \{S \subseteq Q_\cA \st S = \eclose S \}$;
\item $F_\cB = \{S \in Q_\cB \st S \cap F_\cA \neq \emptyset\}$;
\item $\delta_\cB(S,a) = \eclose{ \bigcup_{p \in S} \delta_\cA(p,a) }$
\end{itemize}
\end{definition}

\begin{theorem}[Equivalenza]\label{thm:epselim}
Sia $\cA$ un $\epsilon$-NFA e sia $\cB$ il DFA della Definizione \ref{def:epselim}. Allora $\cL(\cA) = \cL(\cB)$.
\end{theorem}

\begin{proof}
Sia $\cA = (Q_\cA,\Sigma,\delta_\cA,q_0,F_\cA)$ un $\epsilon$-NFA e sia $\cB = (Q_\cB,\Sigma,\delta_\cB,\eclose{q_0},F_\cB)$ dato come nella Definizione \ref{def:epselim}. 

Dimostriamo per induzione sulla lunghezza di $w \in \Sigma^*$ che $\hat{\delta}_\cA(q_0,w) = \hat{\delta}_\cB(\eclose{q_0},w)$.

\noindent\textbf{Base.} La sola stringa di lunghezza $0$ \`{e} $\epsilon$ e per definizione abbiamo $\hat{\delta}_\cA(q_0,\epsilon) = \eclose{q_0} = \hat{\delta}_\cB(\eclose{q_0},\epsilon)$.

\noindent\textbf{Passo induttivo.} Sia $w$ una stringa di lunghezza $\geq 1$, ovvero $w = xa$. Per ipotesi induttiva abbiamo che $\hat{\delta}_\cA(q_0,x) = \hat{\delta}_\cB(\eclose{q_0},x)$. Quindi
\begin{eqnarray*}
\hat{\delta}_\cA(q_0,xa) & = & \eclose{\bigcup_{p \in \hat{\delta}_\cA(q_0,x)} \delta_\cA(p,a)} \qquad \mbox{, per la Def. \ref{def:trans-est3}.} \\
& = & \eclose{ \bigcup_{p \in \hat{\delta}_\cB(\eclose{q_0},x)} \delta_\cA(p,a) } \qquad \mbox{, per l'ipotesi induttiva,} \\
& = & \delta_\cB(\hat{\delta}_\cB(\eclose{q_0},x),a) \qquad \mbox{, per la Def. \ref{def:epselim},} \\
& = & \hat{\delta}_\cB(\eclose{q_0},xa) \qquad \mbox{, per la Def. \ref{def:trans-est}.}
\end{eqnarray*}
Per concludere osserviamo che $\hat{\delta}_\cA(q_0,w) \cap F_\cA \neq \emptyset \iff \hat{\delta}_\cB(\eclose{q_0},w) \in F_\cB$, da cui risulta che $\cL(\cA) = \cL(\cB)$.
\qed\end{proof}

Notiamo che non abbiamo trasformato l'$\epsilon$-NFA in un NFA che accetta lo stesso linguaggio. Perch\'{e}? Provate a farlo da voi.

%%%%%%%%%%%%%%%%%%%%%%%%%%%%%%%%%%%%%%%%%%%%
\section{Linguaggi regolari e loro propriet\`{a}}
%%%%%%%%%%%%%%%%%%%%%%%%%%%%%%%%%%%%%%%%%%%%

\begin{definition}[Linguaggio regolare]\label{def:ling-reg}
Un linguaggio $L$ \`{e} \emph{regolare} sse esiste un DFA $\cA$ tale che $L = \cL(\cA)$.
\end{definition}

Indichiamo con $\REG$ la classe dei linguaggi regolari.

Notiamo che, in virt\`{u} dei Teoremi \ref{thm:det-non-det} e \ref{thm:epselim}, nella Definizione \ref{def:ling-reg} si pu\`{o} equivalentemente richiedere l'esistenza di un NFA (o di un $\epsilon$-NFA) $\cA$ tale che $L = \cL(\cA)$. Faremo uso di questo fatto in seguito.

\begin{theorem}\label{thm:fin-reg}
Ogni linguaggio finito \`{e} regolare.
\end{theorem}

\begin{proof}
Sia $L = \{w_1,\ldots,w_n\}$ un linguaggio su $\Sigma$. Definiamo un opportuno DFA $\cA = (Q_\cA,\Sigma,\delta_\cA,q_0,F_\cA)$ come seque:
\begin{itemize}
\item $Q_\cA = \{x \in \Sigma^* \st \exists w \in L.\ x \mbox{ \`{e} prefisso di } w \}$
\item $q_0 = \epsilon$
\item $F_\cA = L$
\item $\delta_\cA(x,a) = 
\begin{cases}
xa & \mbox{ se } xa \in Q_\cA \\
\mbox{indefinito } & \mbox{ altrimenti } \\
\end{cases}$
\end{itemize}
Ora \`{e} facile dimostrare per induzione sulla lunghezza delle parole in $Q_\cA$ che $\hat{\delta}_\cA(x,a) = xa$ per cui $\cL(\cA) = \{w \in \Sigma^* \st \hat{\delta}_\cA(\epsilon,w) \in F_\cA\} = L$. Ci\`{o} dimostra che $L$ \`{e} regolare.
\qed\end{proof}

In questa sezione ci occupiamo delle propriet\`{a} di chiusura della classe dei linguaggi regolari. Dimostreremo che:
\begin{itemize}
\item l'intersezione di due linguaggi regolari \`{e} un linguaggio regolare,
\item il complemento di un linguaggio regolare \`{e} ancora regolare,
\item l'unione di due linguaggi regolari \`{e} un linguaggio regolare,
\item la differenza di due linguaggi regolari \`{e} un linguaggio regolare,
\item la concatenazione di due linguaggi regolari \`{e} un linguaggio regolare,
\item la potenza di un linguaggio regolare \`{e} ancora regolare,
\item la chiusura di Kleene di linguaggio regolare \`{e} ancora regolare,
\item il rovesciamento di linguaggio regolare \`{e} ancora regolare.
\end{itemize}

Data la Definizione \ref{def:ling-reg}, bisogna di volta in volta mostrare l'esistenza di un opportuno DFA che riconosca il linguaggio unione, intersezione ecc... Questo si pu\`{o} fare mediante delle costruzioni che andremo a definire in questa sezione.

\begin{definition}[Prodotto di DFA]\label{def:prod-DFA}
Siano $\cA = (Q_\cA,\Sigma,\delta_\cA,q_0,F_\cA)$ e \\ $\cB = (Q_\cB,\Sigma,\delta_\cB,q_0',F_\cB)$ due DFA. Definiamo il DFA $\cA \times \cB= (Q_{\cA \times \cB},\Sigma,\delta_{\cA \times \cB},(q_0,q_0'),F_{\cA \times \cB})$ come segue:
\begin{itemize}
\item $Q_{\cA \times \cB} = Q_\cA \times Q_\cB$,
\item $\delta_{\cA \times \cB}((p,q),a) = (\delta_\cA(p,a),\delta_\cB(q,a))$,
\item $F_{\cA \times \cB} = F_\cA \times F_\cB$.
\end{itemize}
\end{definition}

\begin{theorem}[Intersezione]\label{thm:prod-DFA}
L'intersezione di due linguaggi regolari \`{e} regolare.
\end{theorem}

\begin{proof}
Siano $L,L'$ due linguaggi regolari. Allora esistono due DFA $\cA,\cB$ tali che $L = \cL(\cA)$ e $L' = \cL(\cB)$. Ora abbiamo che $\cL(\cA) \cap \cL(\cB)= \{w \in \Sigma^* \st \hat{\delta}_{\cA}(q_0,w) \in F_\cA,\ \hat{\delta}_{\cB}(q_0',w) \in F_\cB\} = \{w \in \Sigma^* \st \hat{\delta}_{\cA \times \cB}((q_0,q_0'),w) \in F_{\cA\times \cB}\} = \cL(\cA \times \cB)$. Quindi $\cA\times\cB$ \`{e} un DFA che accetta $L \cap L'$, dimostrando che $L \cap L'$ \`{e} regolare.
\qed\end{proof}

\begin{definition}[Complemento di DFA]\label{def:compl-DFA}
Sia $\cA = (Q_\cA,\Sigma,\delta_\cA,q_0,F_\cA)$ un DFA. Definiamo suo DFA complementare $\cA^c$ ponendo $\cA^c = (Q_{\cA},\Sigma,\delta_\cA,q_0,Q_\cA-F_\cA)$.
\end{definition}

\begin{theorem}[Complemento]\label{thm:compl-DFA}
Il complemento di un linguaggio regolare \`{e} regolare.
\end{theorem}

\begin{proof}
Sia $L$ un linguaggio regolare. Allora esiste un DFA $\cA$ tale che $L = \cL(\cA)$. Ora abbiamo che $(\cL(\cA))^c = \{w \in \Sigma^* \st \hat{\delta}_\cA(q_0,w) \in Q_\cA-F_\cA\} = \cL(\cA^c)$. Quindi $\cA^c$ \`{e} un DFA che accetta $L^c$, dimostrando che $L^c$ \`{e} regolare.
\qed\end{proof}

\begin{theorem}[Unione e differenza]\label{thm:unione-diff-DFA}
Il l'unione e la differenza di due linguaggi regolari sono regolari.
\end{theorem}

\begin{proof}
Siano $L,L'$ due linguaggi regolari. Abbiamo che $L \cup L' = (L^c \cap (L')^c)^c$ e $L-L' = L \cap (L')^c$, quindi per i Teoremi \ref{thm:prod-DFA} e \ref{thm:compl-DFA} i linguaggi $L \cup L'$ e $L-L'$ sono entrambi regolari.
\qed\end{proof}

\begin{remark}
Sia $\Sigma$ un alfabeto. Notiamo che i linguaggi regolari su $\Sigma$ sono chiusi per unione finita ma non infinita. Infatti se cos\`{i} fosse siccome ogni insieme del tipo $\{w\}$ con $w \in \Sigma^*$ \`{e} regolare, allora anche ogni sottoinsieme di $\Sigma^*$ sarebbe regolare. Vedremo pi\`{u} avanti che ci\`{o} \`{e} falso: esistono dei linguaggi che non sono regolari.
\end{remark}

\begin{definition}[Concatenazione di $\epsilon$-NFA]\label{def:conc-eNFA}
Siano $\cA = (Q_\cA,\Sigma,\delta_\cA,q_0,F_\cA)$ e $\cB = (Q_\cB,\Sigma,\delta_\cB,q_0',F_\cB)$ due $\epsilon$-NFA. Definiamo l'$\epsilon$-NFA $\cA \cdot \cB= (Q_{\cA \cdot\cB},\Sigma,\delta_{\cA\cdot\cB},q_0,F_{\cB})$ come segue:
\begin{itemize}
\item $Q_{\cA\cdot\cB} = Q_\cA \cup Q_\cB$,
\item $\delta_{\cA\cdot\cB}(p,a) =
\begin{cases}
\delta_{\cA}(p,a) & \mbox{ se } p \in Q_\cA \\
\delta_{\cB}(p,a) & \mbox{ se } p \in Q_\cB \\
\end{cases}$
\qquad
$\delta_{\cA\cdot\cB}(p,\epsilon) =
\begin{cases}
\delta_{\cA}(p,\epsilon) & \mbox{ se } p \in Q_\cA - F_\cA \\
\delta_{\cA}(p,\epsilon) \cup \{q_0'\} & \mbox{ se } p \in F_\cA \\
\delta_{\cB}(p,\epsilon) & \mbox{ se } p \in Q_\cB \\
\end{cases}$
\end{itemize}
\end{definition}

\begin{theorem}[Concatenazione]\label{thm:conc-eNFA}
La concatenazione di due linguaggi regolari \`{e} regolare.
\end{theorem}

\begin{proof}
Siano $L,L'$ due linguaggi regolari. Allora esistono due $\epsilon$-NFA $\cA,\cB$ tali che $L = \cL(\cA)$ e $L' = \cL(\cB)$. Ora abbiamo che $\cL(\cA)\cL(\cB) = \{xy \in \Sigma^* \st \hat{\delta}_{\cA}(q_0,x) \in F_\cA,\ \hat{\delta}_{\cB}(q_0',y) \in F_\cB\} = \{w \in \Sigma^* \st \hat{\delta}_{\cA\cdot\cB}(q_0,w) \in F_\cB\} = \cL(\cA\cdot\cB)$. Quindi $\cA\cdot\cB$ \`{e} un $\epsilon$-NFA che accetta $LL'$, dimostrando che $LL'$ \`{e} regolare.
\qed\end{proof}

\begin{theorem}[Potenza]\label{thm:pot-eNFA}
Ogni potenza di un linguaggio regolare \`{e} regolare.
\end{theorem}

\begin{proof}
Sia $L$ un linguaggio regolare. Dimostriamo per induzione su $n$ che $L^n$ \`{e} regolare. 

\noindent\textbf{Base.} Per $n=0$ abbiamo $L^0 = \{\epsilon\}$, che \`{e} un linguaggio regolare. 

\noindent\textbf{Passo induttivo.} Per $n > 0$ abbiamo $L^n = L^{n-1}L$. Ora $L^{n-1}$ \`{e} regolare per ipotesi induttiva, ed $L$ \`{e} regolare per ipotesi, quindi per il Teorema \ref{thm:conc-eNFA} anche $L^n$ \`{e} regolare.
\qed\end{proof}

\begin{definition}[Chiusura di Kleene di $\epsilon$-NFA]\label{def:Kleene-eNFA}
Sia $\cA = (Q_\cA,\Sigma,\delta_\cA,q_0,F_\cA)$ un $\epsilon$-NFA. Definiamo l'$\epsilon$-NFA $\cA^*= (Q_\cA,\Sigma,\delta_{\cA^*},q_0,F_\cA)$ come segue:
\begin{itemize}
\item $\delta_{\cA^*}(p,a) = \delta_{\cA}(p,a)$; \qquad 
$\delta_{\cA^*}(p,\epsilon) =
\begin{cases}
\delta_{\cA}(p,\epsilon) & \mbox{ se } p \in Q_\cA - F_\cA \\
\delta_{\cA}(p,\epsilon) \cup \{q_0\} & \mbox{ se } p \in F_\cA \\
\end{cases}$
\end{itemize}
\end{definition}

\begin{theorem}[Chiusura di Kleene]\label{thm:Kleene-eNFA}
La chiusura di Kleene di un linguaggio regolare \`{e} regolare.
\end{theorem}

\begin{proof}
Sia $L$ un linguaggio regolare. Allora esiste un $\epsilon$-NFA $\cA$ tale che $L = \cL(\cA)$. Ora abbiamo che $(\cL(\cA))^* = \bigcup_{n\geq 0} \{x_1\cdots x_n \in \Sigma^* \st \forall i = 1,\ldots,n.\ \hat{\delta}_\cA(q_0,x_i) \in F_\cA\} = \{w \in \Sigma^* \st \hat{\delta}_{\cA^*} (q_0,w) \in F_\cA\} = \cL(\cA^*)$. Quindi $\cA^*$ \`{e} un $\epsilon$-NFA che accetta $L^*$, dimostrando che $L^*$ \`{e} regolare.
\qed\end{proof}

\begin{definition}[Rovesciamento di $\epsilon$-NFA]\label{def:rovesc-eNFA}
Sia $\cA = (Q_{\cA},\Sigma,\delta,q_0,F)$ un $\epsilon$-NFA. Definiamo il DFA $\cA^R = (_{\cA^R},\Sigma,\delta_{\cA^R},q_0',F_{\cA^R})$ come segue:
\begin{itemize}
\item $Q_{\cA^R} = Q_{\cA} \cup \{q_0'\}\in \delta_{\cA^R}(q,a)$ sse $q \in \delta_{\cA}(p,a)$,
\item $p \in \delta_{\cA^R}(q,a)$ sse $q \in \delta_{\cA}(p,a)$, per ogni $p,q \in Q_{\cA}$
\item $p \in \delta_{\cA^R}(q,\epsilon)$ sse $q \in \delta_{\cA}(p,\epsilon)$, per ogni $p,q \in Q_{\cA}$
\item $\delta_{\cA^R}(q_0',\epsilon) = F_{\cA}$
\item $F_{\cA^R} =\{q_0\}$.
\end{itemize}
\end{definition}

Questa costruzione, seppure pu\`{o} sembrare complessa, dice che l'automa $\cA^R$ \`{e} ottenuto rovesciando tutti gli archi presenti in $\cA$. Lo stato finale di $\cA^R$ \`{e} quello iniziale di $\cA$ e lo stato iniziale di $\cA^R$ \`{e} un nuovo stato, collegato tramite $\epsilon$-mosse a quelli che erano gli stati finali di $\cA$.

\begin{theorem}[Rovesciamento]\label{thm:rovesc-eNFA}
Il rovesciamento di un linguaggio regolare \`{e} regolare.
\end{theorem}

\begin{proof}
Sia $L$ un linguaggio regolare. Allora esiste un $\epsilon$-NFA $\cA$ tale che $L = \cL(\cA)$. Ora abbiamo che $(\cL(\cA))^R = \{w^R \st \hat{\delta}_\cA(q_0,w) \in F_\cA\} = \{x \st \hat{\delta}_{\cA^R} (q_0',x) \in F_{\cA^R}\} = \cL(\cA^R)$. Quindi $\cA^R$ \`{e} un $\epsilon$-NFA che accetta $L^R$, dimostrando che $L^R$ \`{e} regolare.
\qed\end{proof}

%%%%%%%%%%%%%%%%%%%%%%%%%%%%%%%%%%%%%%%%%%%%
\section{Caratterizzazioni alternative dei linguaggi regolari}
%%%%%%%%%%%%%%%%%%%%%%%%%%%%%%%%%%%%%%%%%%%%

%%%%%%%%%%%%%%%%%%%%%%%%%%%%%%%%%%%%%%%%%%%%
\subsection{Espressioni regolari}
%%%%%%%%%%%%%%%%%%%%%%%%%%%%%%%%%%%%%%%%%%%%

In questa sezione presentiamo le cosiddette ``espressioni regolari", un tipo di notazione per definire linguaggi. Da una descrizione dei linguaggi regolari fondata su modelli di automa (DFA, NFA, $\epsilon$-NFA) volgiamo ora l'attenzione ad una descrizione algebrica. 

\begin{definition}[Espressioni regolari]\label{def:regexp}
Sia $\Sigma$ un alfabeto. Definiamo per induzione le \emph{espressione regolari} su $\Sigma$ come segue:
\begin{itemize}
\item le costanti $\epsilon$ e $\emptyset$ sono espressioni regolari ed ogni elemento di $\Sigma$ \`{e} un'espressione regolare;
\item se $r,s$ sono espressioni regolari, allora $r \cdot s$ \`{e} un'espressione regolare;
\item se $r,s$ sono espressioni regolari, allora $r + s$ \`{e} un'espressione regolare;
\item se $r$ \`{e} un'espressione regolare, allora $r^*$ \`{e} un'espressione regolare.
\end{itemize} 
\end{definition}

\begin{definition}[Linguaggio di un'espressione regolare]\label{def:lin-regexp}
Definiamo per induzione il \emph{linguaggio $\cL(r)$ associato} ad un' espressione regolare $r$ come segue:
\begin{itemize}
\item $\cL(\epsilon) = \{\epsilon\}$, $\cL(\emptyset) = \emptyset$ e $\cL(a) = \{a\}$, per ogni $a \in \Sigma$;
\item $\cL(r\cdot s) = \cL(r)\cdot \cL(s)$;
\item $\cL(r+ s) = \cL(r)\cup \cL(s)$;
\item $\cL(r^*) = (\cL(r))^*$.
\end{itemize} 
\end{definition}

\begin{theorem}\label{thm:regexp-NFA}
Sia $r$ un'espressione regolare. Allora esiste un $\epsilon$-NFA $\cA$ tale che $\cL(r) = \cL(\cA)$.
\end{theorem}

\begin{proof}
Procediamo per induzione sulla costruzione delle espressioni regolari.

\noindent\textbf{Base.} Riportiamo qui sotto gli automi finiti che riconoscono, rispettivamente, i linguaggi $\{\epsilon\}$, $\emptyset$ e $\{a\}$. \\

%\entrymodifiers={++[o][F-]} 
%\SelectTips{cm}{} 
%$\xymatrix{ 
%*\txt{start} \ar[r] & *++[o][{0} \ar@(r,u)[]^b \\ 
%}$
\entrymodifiers={++[o][F-]}
\noindent$\xymatrix{
*\txt{ } \ar[r]
& 0 \ar[r]^{\epsilon} 
& *++[o][F=]{1} \\ 
}
\qquad\qquad\qquad
\xymatrix{
*\txt{ } \ar[r]
& 0 
& *++[o][F=]{1} \\ 
}$ \\ \\
$\xymatrix{
*\txt{ } \ar[r]
& 0 \ar[r]^{a} 
& *++[o][F=]{1} \\ 
}$ \\

\noindent\textbf{Passo induttivo.} Siano $r,s$ due espressioni regolari e supponiamo di aver costruito due $\epsilon$-NFA $\cA$, $\cB$ tali che $\cL(\cA) = \cL(r)$ e $\cL(\cB) = \cL(s)$. Allora 
\begin{itemize}
\item[\ ] Avendo due automi $\cA,\cB$ che riconoscono $\cL(r)$ e $\cL(s)$, rispettivamente, il seguente \`{e} un automa che riconosce $\cL(r + s)$.
$$\xymatrix{
*\txt{ }                  &             *\txt{ }            & \cA \ar[dr]^{\epsilon}  \\
*\txt{ } \ar[r] & 0 \ar[ur]^{\epsilon}\ar[dr]_{\epsilon} & *\txt{ } & *++[o][F=]{1} \\ 
*\txt{ }                  &            *\txt{ }              & \cB \ar[ur]_{\epsilon} \\
}$$
La lettera $\cA$ cerchiata simboleggia l'intero automa $\cA$ e la $\epsilon$-transizione entrante indica un arco etichettato da $\epsilon$ che si collega allo stato iniziale di $\cA$; inoltre la $\epsilon$-transizione che esce indica un insieme di archi etichettati da $\epsilon$, ciascuno uscente da uno stato finale di $\cA$.
\item[\ ] Avendo due automi $\cA,\cB$ che riconoscono $\cL(r)$ e $\cL(s)$, rispettivamente, il seguente \`{e} un automa che riconosce $\cL(r\cdot s)$.
$$\xymatrix{
*\txt{ } \ar[r] & \cA \ar[r]^{\epsilon} & \cB \ar[r]^{\epsilon} & *++[o][F=]{1} \\
}$$
\item[\ ] Avendo un automa $\cA$ che riconosce $\cL(r)$, il seguente \`{e} un automa che riconosce $\cL(r^*)$.
$$\xymatrix{
*\txt{ } \ar[r] & \cA \ar@(r,d)[]^{\epsilon} \ar[r]^{\epsilon} & *++[o][F=]{0} \\ 
}$$
\end{itemize}
\qed\end{proof}

\begin{theorem}\label{thm:NFA-regexp}
Sia $\cA$ un DFA. Allora esiste un'espressione regolare $r$ tale che $\cL(\cA) = \cL(r)$.
\end{theorem}

\begin{proof}
Possiamo assumere senza perdita di generalit\`{a} che l'insieme degli stati di $\cA$ sia $\{1,\ldots,n\}$, per un certo $n\geq 1$. Definiamo ora $L_{ij}^{(k)}$ come il linguaggio formato da tutte e sole le stringhe $w$ tali che:
\begin{enumerate}[(1)]
\item $\hat{\delta}_{\cA}(i,w) = j$ e  
\item per ogni prefisso non banale $x$ di $w$ abbiamo $\hat{\delta}_{\cA}(i,x) \leq k$.
\end{enumerate}

Andiamo quindi a costruire, per induzione su $k$, un'espressione regolare $R_{ij}^{(k)}$ avente la propriet\`{a} che $\cL(R_{ij}^{(k)}) = L_{ij}^{(k)}$.

\noindent\textbf{Base.} Supponiamo $k=0$. Distinguiamo due casi:
\begin{itemize}
\item Se $i=j$ allora poniamo $X = \{a \in \Sigma \st \delta_\cA(i,a) = i\}$ e $R_{ii}^{(0)} = \epsilon + \sum_{a \in X} a$.
\item Se $i\neq j$ allora poniamo $X = \{a \in \Sigma \st \delta_\cA(i,a) = j\}$ e $R_{ij}^{(0)} = \sum_{a \in X} a$, con la convenzione che $\sum_{a \in \emptyset} a = \emptyset$.
\end{itemize}
\`{E} immediato verificare che $\cL(R_{ij}^{(0)}) = L_{ij}^{(0)}$.

\noindent\textbf{Passo induttivo.} Supponiamo di aver determinato $R_{ij}^{(k-1)}$, per ogni coppia $(i,j)$ di stati in $Q_\cA$. Allora poniamo $R_{ij}^{(k)} = R_{ij}^{(k-1)} + R_{ik}^{(k-1)}\cdot (R_{kk}^{(k-1)})^*\cdot R_{kj}^{(k-1)}$.

Ancora una volta non \`{e} difficile vedere che $\cL(R_{ij}^{(k)}) = L_{ij}^{(k)}$, poich\'{e}\\ $L_{ij}^{(k)} = L_{ij}^{(k-1)} + L_{ik}^{(k-1)}\cdot (L_{kk}^{(k-1)})^*\cdot L_{kj}^{(k-1)}$.

In questo modo arriviamo a calcolare $R_{ij}^{(n)}$, per ogni coppia $(i,j)$ di stati. Sia $i$ lo stato iniziale di $\cA$; a questo punto definiamo l'espressione regolare\\ $r = \sum_{j \in F_\cA} R_{ij}^{(n)}$ e per costruzione otteniamo che $\cL(r)= \bigcup_{j \in F_\cA} L_{ij}^{(n)} =\cL(\cA)$.
\qed\end{proof}

\begin{corollary}\label{cor:lin-regexp}
Un linguaggio $L$ \`{e} regolare sse esiste un'espressione regolare $r$ tale che $L = \cL(r)$.
\end{corollary}

\begin{proof}
Segue immediatamente dai Teoremi \ref{thm:regexp-NFA} e \ref{thm:NFA-regexp}.
\qed\end{proof}

\begin{theorem}[Teorema di Kleene per i linguaggi regolari]\label{thm:Kleene-reglin}
L'insieme $\REG$ dei linguaggi regolari su un dato alfabeto $\Sigma$ \`{e} la pi\`{u} piccola famiglia $\cF$ di sottoinsiemi di $\Sigma^*$ tale che 
\begin{itemize}
\item $\cF$ contiene tutti i sottoinsiemi finiti di $\Sigma^*$,
\item $\cF$ \`{e} chiusa per unione e concatenazione finita,
\item $\cF$ \`{e} chiusa per stella di Kleene.
\end{itemize}
\end{theorem}

\begin{proof}
Segue dal Corollario \ref{cor:lin-regexp}, poich\'{e} $\REG$ coincide con l'insieme dei linguaggi generati da espressioni regolari, che a loro volta sono per definizione introdotti tramite le propriet\`{a} algebriche sopraccitate.
\qed\end{proof}

La famiglia dei linguaggi regolari su un dato alfabeto $\Sigma$ \`{e} un insieme, ma di solito si parla di $\REG$ come della \emph{classe} dei linguaggi regolari. Questa terminologia va usata quando non si fa riferimento ad un alfabeto fissato, perch\'{e} allora la famiglia diventa ``troppo grande" per essere un insieme.

Notiamo inoltre come non sia immediato dimostrare, partendo dalla teoria delle espressioni regolari e dal Corollario \ref{cor:lin-regexp}, che $\REG$ \`{e} chiusa per complementazione.

%%%%%%%%%%%%%%%%%%%%%%%%%%%%%%%%%%%%%%%%%%%%
\subsection{Grammatiche regolari}
%%%%%%%%%%%%%%%%%%%%%%%%%%%%%%%%%%%%%%%%%%%%

\begin{definition}[Grammatica regolare]\label{def:reg-gramm}
Una \emph{grammatica regolare} su un alfabeto $\Sigma$ \`{e} una quadrupla $\cG = (V_T,V_N,P,S)$ dove
\begin{itemize}
\item $V_T$ \`{e} un sottoinsieme di $\Sigma$, detto l'insieme dei terminali;
\item $V_N$ \`{e} l'insieme dei non-terminali (si assume $V_N \cap V_T = \emptyset$);
\item $S \in V_N$ \`{e} il simbolo iniziale;
\item $P$, detto l'insieme delle produzioni, \`{e} un insieme finito di regole che definiscono una relazione $\to \subseteq (V_N \times V_TV_N \cup V_T \cup \{\epsilon\})$. In altre parole le regole possono assumere la forma $A \to aB$, $A \to a$, oppure $A \to \epsilon$ dove le lettere maiuscole $A,B,\ldots$ variano su $V_N$ e le lettere minuscole $a,b,\ldots$ variano su $V_T$
\end{itemize}
\end{definition}

Le grammatiche regolari sono anche dette \emph{tipo 3}.

Per abbreviare le notazioni nella scrittura delle produzioni grammaticali scriveremo $A \to aB \mid aC$ ove vi siano due produzioni $A \to aB$ e $A \to aC$.

Definiamo la concatenazione $\to^*$ di produzioni informalmente, con un esempio, come segue: se $A \to aB$ e $B \to bC$ sono due produzioni, allora $A \to^* abC$.

\begin{definition}[Linguaggio generato da una grammatica]\label{def:ling-reg-gramm}
Il linguaggio generato da una grammatica $\cG$ \`{e} l'insieme $\cL(\cG)$ definito come segue:\\ $\cL(\cG) = \{w \in V_T^* \st S \to^* w\}$.
\end{definition}

\begin{theorem}\label{thm:ling-reg-gramm1}
Se un linguaggio $L$ \`{e} regolare, allora esiste una grammatica regolare $\cG$ tale che $L = \cL(\cG)$.
\end{theorem}

\begin{proof}
Sia $\cA=(Q,\Sigma,\delta,q_0,F)$ un DFA tale che $\cL(\cA) = L$. Costruiamo una grammatica $\cG = (V_T,V_N,P,S)$ tale che $\cL(\cA) = \cL(\cG)$. Per evitare confusione supponiamo di avere, per ogni stato $q$, un non-terminale corrispondente $A_q$. Poniamo:
\begin{itemize}
\item $V_N = \{A_q \st q \in Q\}$;
\item $V_T = \Sigma$;
\item $S = A_{q_0}$;
\item infine definiamo l'insieme $P$ delle produzioni come il pi\`{u} piccolo che soddisfa le seguenti propriet\`{a}:
\begin{itemize}
\item se $\delta(q,b) = p$, allora $A_q \to bA_p$ appartiene a $P$;
\item se $\delta(q,b) \in F$, allora $A_q \to b$ appartiene a $P$.
\end{itemize}
\end{itemize}
Non \`{e}  complicato ora mostrare che $\hat{\delta}(q_0,w) \in F$ sse $S \to^* w$, da cui segue il risultato che volevamo.
\qed\end{proof}

\begin{theorem}\label{thm:ling-reg-gramm2}
Sia $\cG$ una grammatica regolare. Allora $\cL(\cG)$ \`{e} un linguaggio regolare.
\end{theorem}

\begin{proof}
Sia $\cG = (V_T,V_N,P,S)$ una grammatica regolare su $\Sigma$. Dimostriamo che esiste un $\epsilon$-NFA $\cA=(Q,\Sigma,\delta,q_0,F)$ tale che $\cL(\cG) = \cL(\cA)$. Per evitare confusione supponiamo di avere, per ogni terminale $a$, uno stato corrispondente $q_a$ e per ogni non-terminale $A$ uno stato $q_A$; inoltre prendiamo uno stato $q_\epsilon$. Poniamo
\begin{itemize}
\item $Q = \{q_A \st A \in V_N\} \cup \{q_a \st a \in V_T\} \cup \{q_\epsilon\}$;
\item $F = \{q_a \st a \in V_T\} \cup \{q_\epsilon\}$;
\item $q_0 = q_S$;
\item $\delta$ \`{e} la pi\`{u} piccola funzione parziale (in termini del suo grafico) che soddisfa le seguenti propriet\`{a}:
\begin{itemize}
\item se $A \to bB$ \`{e} una produzione di $P$, allora $q_B \in \delta(q_A,b)$;
\item se $A \to b$ \`{e} una produzione di $P$, allora $q_b \in \delta(q_A,\epsilon)$;
\item se $A \to \epsilon$ \`{e} una produzione di $P$, allora $q_\epsilon \in \delta(q_A,\epsilon)$.
\end{itemize}
\end{itemize}
Non \`{e}  complicato ora mostrare che $S \to^* w$ sse $\hat{\delta}(q_S,w) \in F$, da cui segue il risultato che volevamo.
\qed\end{proof}

%%%%%%%%%%%%%%%%%%%%%%%%%%%%%%%%%%%%%%%%%%%%
\subsection{Pumping lemma tipo 3}
%%%%%%%%%%%%%%%%%%%%%%%%%%%%%%%%%%%%%%%%%%%%

Concludiamo la sezione dedicata alle descrizioni alternative per il linguaggi regolari con il cosiddetto approccio descrittivo. Esso non risulta in una caratterizzazione dei linguaggi regolari, ma in una condizione necessaria alla regolarit\`{a}, che si concretizza nel cosiddetto \emph{Pumping Lemma}. 

\begin{lemma}[Pumping Lemma tipo 3]\label{lem:pumping3}
Se $L$ \`{e} un linguaggio regolare, allora esiste una costante $n \geq 1$, che dipende solo da $L$, tale che per ogni $w \in L$, con $|w| \geq n$, esistono stringhe $u,v,z$ tali che 
\begin{itemize}
  \item $w = uvz$,
  \item $|uv| \leq n$,
  \item $|v| \geq 1$,
  \item per ogni $k \geq 0$ abbiamo $uv^kz \in L$.
\end{itemize}
\end{lemma}

\begin{proof}
Supponiamo $L$ regolare. Allora esiste un DFA $\cA$ tale che $L = \cL(\cA)$. Sia $n$ il numero di stati di $\cA$ e sia $w$ una stringa di lunghezza $\geq n$. Poniamo $X = \{\hat{\delta}_\cA(q_0,x) \st x \mbox{ prefisso di } w\}$. Poich\'{e} $|Q_\cA| = n$ e $w$ ha almeno $n+1$ prefissi, esistono necessariamente due prefissi $u,u'$ di $w$ tali per cui $\hat{\delta}_\cA(q_0,u) = \hat{\delta}_\cA(q_0,u')$. Possiamo assumere, senza perdita di generalit\`{a}, che $u$ sia prefisso di $u'$. Riassumendo quindi esistono due stringhe $v,z$ tali che $uv = u'$ e $u'z = w$; quindi $uvz = w$. Notiamo inoltre che $\hat{\delta}_\cA(\hat{\delta}_\cA(q_0,u),v) = \hat{\delta}_\cA(q_0,u)$, ovvero $v$ porta l'automa da $\hat{\delta}_\cA(q_0,u)$ in se stesso, e quindi la sequenza di input che compongono $v$ pu\`{o} essere ripetuta un numero arbitrario di volte. Infine, poich\`{e} $w \in L$, abbiamo che $\hat{\delta}_\cA(\hat{\delta}_\cA(q_0,u'),z) \in F$; quindi per il ragionamento fatto qui sopra $\hat{\delta}_\cA(\hat{\delta}_\cA(q_0,uv^k),z) \in F$ per ogni $k \geq 0$. Questo dimostra che $uv^kz \in L$ per ogni $k \geq 0$.
\qed\end{proof}

Il Lemma \ref{lem:pumping3} trova la sua migliore applicazione nel provare quando un linguaggio non \`{e} regolare. Difatti se, dato $L$, non si verifica che non esiste una costante $n$ che soddisfa le propriet\`{a} dell'enunciato, allora $L$ non pu\`{o} essere regolare.

\begin{remark}
Notiamo che il reciproco del Pumping Lemma non vale. In altre parole vi \`{e} un linguaggio non regolare $L$ (in verit\`{a} ve ne sono molti), tale per cui esiste una costante $n$, che dipende solo da $L$, tale che per ogni $w \in L$, con $|z| \geq n$, esistono stringhe $u,v,w$ tali che 
\begin{itemize}
  \item $w = uvz$,
  \item $|uv| \leq n$,
  \item $|v| \geq 1$,
  \item per ogni $k \geq 0$ abbiamo $uv^kz \in L$.
\end{itemize}
\end{remark}

%%%%%%%%%%%%%%%%%%%%%%%%%%%%%%%%%%%%%%%%%%%%
\section{Esercizi}
%%%%%%%%%%%%%%%%%%%%%%%%%%%%%%%%%%%%%%%%%%%%

\begin{example}
Dimostrare se i seguenti linguaggi sono o meno regolari:
\begin{itemize}
\item $L_1 = \{0^n1^n \st n \geq 1\}$
\item $L_2 = \{0^n \st n \mbox{ \`{e} potenza di } 2\}$
\item $L_3 = \{w \in \{0,1\}^* \st |w| \leq 1000\}$
\item $L_4 = \{w \in \{0,1\}^* \st |w| > 1000\}$
\item $L_5 = \{ww^R \st w \in \{0,1\}^*\}$
\item $L_6 = \bigcup_{i=1}^{100} \{w \in \{0,1\}^* \st i \leq |w| \leq 1000\}$
\item $L_7 = \{w \in \{0,1\}^* \st |w| \mbox{ \`{e} multiplo di } 5 \}$
\end{itemize}
\end{example}

\begin{example}
Sia $L$ un linguaggio regolare. \`{E} vero che ogni linguaggio $L' \subseteq L$ \`{e} anch'esso regolare?
\end{example}

%%%%%%%%%%%%%%%%%%%%%%%%%%%%%%%%%%%%%%%%%%%%
%\bibliographystyle{abbrv}%splncs
%\bibliography{bibliography}
%%%%%%%%%%%%%%%%%%%%%%%%%%%%%%%%%%%%%%%%%%%%
\end{document}

%%%%%%%%%%%%%%%%%%%%%%%%%%%%%%%%%%%%%%%%%%%%
%%%%%%%%%%%%%%%%%%%%%%%%%%%%%%%%%%%%%%%%%%%%
%%%%%%%%%%%%%%%%%%%%%%%%%%%%%%%%%%%%%%%%%%%%