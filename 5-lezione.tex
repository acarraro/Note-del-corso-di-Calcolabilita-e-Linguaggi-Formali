\documentclass[runningheads,a4paper]{llncs}

\usepackage{amssymb}
\usepackage{amsmath}

\usepackage{mathrsfs}
\usepackage{stmaryrd}

\usepackage{enumitem}
%\usepackage{enumerate}

\usepackage{color}
\usepackage{graphicx}
\usepackage{rotating}
%\usepackage{xparse}
%\usepackage{latex8}
\usepackage{upgreek} 
\usepackage{cmll}
\usepackage{url}
\usepackage{xifthen}% provides \isempty test

\setcounter{tocdepth}{3}

\urldef{\mailsa}\path|{acarraro}@dsi.unive.it|
%\urldef{\mailsb}\path||
%\urldef{\mailsc}\path|
\newcommand{\keywords}[1]{\par\addvspace\baselineskip
\noindent\keywordname\enspace\ignorespaces#1}

\input prooftree.sty
\input xy
\xyoption{all}

\makeindex

\begin{document}

\mainmatter  % start of an individual contribution

% first the title is needed
\title{Note del corso di Calcolabilit\`{a} e Linguaggi Formali - Lezione 5}

% a short form should be given in case it is too long for the running head
\titlerunning{Note del corso di Calcolabilit\`{a} e Linguaggi Formali - Lezione 5}

% the name(s) of the author(s) follow(s) next
%
% NB: Chinese authors should write their first names(s) in front of
% their surnames. This ensures that the names appear correctly in
% the running heads and the author index.
%
\author{Alberto Carraro \\ 23 novembre 2011}
%
\authorrunning{A. Carraro}
% (feature abused for this document to repeat the title also on left hand pages)

% the affiliations are given next; don't give your e-mail address
% unless you accept that it will be published
\institute{DAIS, Universit\'{a} Ca' Foscari Venezia
%\mailsa\\
%\mailsb\\
%\mailsc\\
\url{http://www.dsi.unive.it/~acarraro}
}

%
% NB: a more complex sample for affiliations and the mapping to the
% corresponding authors can be found in the file "llncs.dem"
% (search for the string "\mainmatter" where a contribution starts).
% "llncs.dem" accompanies the document class "llncs.cls".
%

\toctitle{Note del corso di Calcolabilit\`{a} e Linguaggi Formali - Lezione 5}
\tocauthor{A. Carraro}

\newcommand{\scA}{\mathscr{A}}
\newcommand{\scB}{\mathscr{B}}
\newcommand{\scC}{\mathscr{C}}
\newcommand{\scD}{\mathscr{D}}
\newcommand{\scE}{\mathscr{E}}
\newcommand{\scF}{\mathscr{F}}
\newcommand{\scG}{\mathscr{G}}
\newcommand{\scH}{\mathscr{H}}
\newcommand{\scI}{\mathscr{I}}
\newcommand{\scJ}{\mathscr{J}}
\newcommand{\scK}{\mathscr{K}}
\newcommand{\scL}{\mathscr{L}}
\newcommand{\scM}{\mathscr{M}}
\newcommand{\scN}{\mathscr{N}}
\newcommand{\scO}{\mathscr{O}}
\newcommand{\scP}{\mathscr{P}}
\newcommand{\scQ}{\mathscr{Q}}
\newcommand{\scR}{\mathscr{R}}
\newcommand{\scS}{\mathscr{S}}
\newcommand{\scT}{\mathscr{T}}
\newcommand{\scU}{\mathscr{U}}
\newcommand{\scV}{\mathscr{V}}
\newcommand{\scW}{\mathscr{W}}
\newcommand{\scX}{\mathscr{X}}
\newcommand{\scY}{\mathscr{Y}}
\newcommand{\scZ}{\mathscr{Z}}

\newcommand{\fA}{\mathfrak{A}}
\newcommand{\fB}{\mathfrak{B}}
\newcommand{\fC}{\mathfrak{C}}
\newcommand{\fD}{\mathfrak{D}}
\newcommand{\fE}{\mathfrak{E}}
\newcommand{\fF}{\mathfrak{F}}
\newcommand{\fG}{\mathfrak{G}}
\newcommand{\fH}{\mathfrak{H}}
\newcommand{\fI}{\mathfrak{I}}
\newcommand{\fJ}{\mathfrak{J}}
\newcommand{\fK}{\mathfrak{K}}
\newcommand{\fL}{\mathfrak{L}}
\newcommand{\fM}{\mathfrak{M}}
\newcommand{\fN}{\mathfrak{N}}
\newcommand{\fO}{\mathfrak{O}}
\newcommand{\fP}{\mathfrak{P}}
\newcommand{\fQ}{\mathfrak{Q}}
\newcommand{\fR}{\mathfrak{R}}
\newcommand{\fS}{\mathfrak{S}}
\newcommand{\fT}{\mathfrak{T}}
\newcommand{\fU}{\mathfrak{U}}
\newcommand{\fV}{\mathfrak{V}}
\newcommand{\fW}{\mathfrak{W}}
\newcommand{\fX}{\mathfrak{X}}
\newcommand{\fY}{\mathfrak{Y}}
\newcommand{\fZ}{\mathfrak{Z}}

\newcommand\tA{{\mathsf{A}}}
\newcommand\tB{{\mathsf{B}}}
\newcommand\tC{{\mathsf{C}}}
\newcommand\tD{{\mathsf{D}}}
\newcommand\tE{{\mathsf{E}}}
\newcommand{\tF}{\mathsf{F}}
\newcommand\tG{{\mathsf{G}}}
\newcommand\tH{{\mathsf{H}}}
\newcommand\tI{{\mathsf{I}}}
\newcommand\tJ{{\mathsf{J}}}
\newcommand\tK{{\mathsf{K}}}
\newcommand\tL{{\mathsf{L}}}
\newcommand\tM{{\mathsf{M}}}
\newcommand\tN{{\mathsf{N}}}
\newcommand\tO{{\mathsf{O}}}
\newcommand\tP{{\mathsf{P}}}
\newcommand\tQ{{\mathsf{Q}}}
\newcommand\tR{{\mathsf{R}}}
\newcommand\tS{{\mathsf{S}}}
\newcommand\tT{{\mathsf{T}}}
\newcommand\tU{{\mathsf{U}}}
\newcommand\tV{{\mathsf{V}}}
\newcommand\tW{{\mathsf{W}}}
\newcommand\tX{{\mathsf{X}}}
\newcommand\tY{{\mathsf{Y}}}
\newcommand\tZ{{\mathsf{Z}}}

%Sums
\newcommand{\sM}{\mathbb{M}}
\newcommand{\sN}{\mathbb{N}}
\newcommand{\sL}{\mathbb{L}}
\newcommand{\sH}{\mathbb{H}}
\newcommand{\sP}{\mathbb{P}}
\newcommand{\sQ}{\mathbb{Q}}
\newcommand{\sR}{\mathbb{R}}
\newcommand{\sA}{\mathbb{A}}
\newcommand{\sB}{\mathbb{B}}
\newcommand{\sC}{\mathbb{C}}
\newcommand{\sD}{\mathbb{D}}

%overlined letters
\newcommand{\ova}{\bar{a}}
\newcommand{\ovb}{\bar{b}}
\newcommand{\ovc}{\bar{c}}
\newcommand{\ovd}{\bar{d}}
\newcommand{\ove}{\bar{e}}
\newcommand{\ovf}{\bar{f}}
\newcommand{\ovg}{\bar{g}}
\newcommand{\ovh}{\bar{h}}
\newcommand{\ovi}{\bar{i}}
\newcommand{\ovj}{\bar{j}}
\newcommand{\ovk}{\bar{k}}
\newcommand{\ovl}{\bar{l}}
\newcommand{\ovm}{\bar{m}}
\newcommand{\ovn}{\bar{n}}
\newcommand{\ovo}{\bar{o}}
\newcommand{\ovp}{\bar{p}}
\newcommand{\ovq}{\bar{q}}
\newcommand{\ovr}{\bar{r}}
\newcommand{\ovs}{\bar{s}}
\newcommand{\ovt}{\bar{t}}
\newcommand{\ovu}{\bar{u}}
\newcommand{\ovv}{\bar{v}}
\newcommand{\ovw}{\bar{w}}
\newcommand{\ovx}{\bar{x}}
\newcommand{\ovy}{\bar{y}}
\newcommand{\ovz}{\bar{z}}

%overlined capital letters
\newcommand{\ovA}{\overline{A}}
\newcommand{\ovB}{\overline{B}}
\newcommand{\ovC}{\overline{C}}
\newcommand{\ovD}{\overline{D}}
\newcommand{\ovE}{\overline{E}}
\newcommand{\ovF}{\overline{F}}
\newcommand{\ovG}{\overline{G}}
\newcommand{\ovH}{\overline{H}}
\newcommand{\ovI}{\overline{I}}
\newcommand{\ovJ}{\overline{J}}
\newcommand{\ovK}{\overline{K}}
\newcommand{\ovL}{\overline{L}}
\newcommand{\ovM}{\overline{M}}
\newcommand{\ovN}{\overline{N}}
\newcommand{\ovO}{\overline{O}}
\newcommand{\ovP}{\overline{P}}
\newcommand{\ovQ}{\overline{Q}}
\newcommand{\ovR}{\overline{R}}
\newcommand{\ovS}{\overline{S}}
\newcommand{\ovT}{\overline{T}}
\newcommand{\ovU}{\overline{U}}
\newcommand{\ovV}{\overline{V}}
\newcommand{\ovW}{\overline{W}}
\newcommand{\ovX}{\overline{X}}
\newcommand{\ovY}{\overline{Y}}
\newcommand{\ovZ}{\overline{Z}}

%vec capital letters
\newcommand{\veA}{\vec{A}}
\newcommand{\veB}{\vec{B}}
\newcommand{\veC}{\vec{C}}
\newcommand{\veD}{\vec{D}}
\newcommand{\veE}{\vec{E}}
\newcommand{\veF}{\vec{F}}
\newcommand{\veG}{\vec{G}}
\newcommand{\veH}{\vec{H}}
\newcommand{\veI}{\vec{I}}
\newcommand{\veJ}{\vec{J}}
\newcommand{\veK}{\vec{K}}
\newcommand{\veL}{\vec{L}}
\newcommand{\veM}{\vec{M}}
\newcommand{\veN}{\vec{N}}
\newcommand{\veO}{\vec{O}}
\newcommand{\veP}{\vec{P}}
\newcommand{\veQ}{\vec{Q}}
\newcommand{\veR}{\vec{R}}
\newcommand{\veS}{\vec{S}}
\newcommand{\veT}{\vec{T}}
\newcommand{\veU}{\vec{U}}
\newcommand{\veV}{\vec{V}}
\newcommand{\veW}{\vec{W}}
\newcommand{\veX}{\vec{X}}
\newcommand{\veY}{\vec{Y}}
\newcommand{\veZ}{\vec{Z}}

%bold capital letters
\newcommand{\bA}{\mathbf{A}}
\newcommand{\bB}{\mathbf{B}}
\newcommand{\bC}{\mathbf{C}}
\newcommand{\bD}{\mathbf{D}}
\newcommand{\bE}{\mathbf{E}}
\newcommand{\bF}{\mathbf{F}}
\newcommand{\bG}{\mathbf{G}}
\newcommand{\bH}{\mathbf{H}}
\newcommand{\bI}{\mathbf{I}}
\newcommand{\bJ}{\mathbf{J}}
\newcommand{\bK}{\mathbf{K}}
\newcommand{\bL}{\mathbf{L}}
\newcommand{\bM}{\mathbf{M}}
\newcommand{\bN}{\mathbf{N}}
\newcommand{\bO}{\mathbf{O}}
\newcommand{\bP}{\mathbf{P}}
\newcommand{\bQ}{\mathbf{Q}}
\newcommand{\bR}{\mathbf{R}}
\newcommand{\bS}{\mathbf{S}}
\newcommand{\bT}{\mathbf{T}}
\newcommand{\bU}{\mathbf{U}}
\newcommand{\bV}{\mathbf{V}}
\newcommand{\bW}{\mathbf{W}}
\newcommand{\bX}{\mathbf{X}}
\newcommand{\bY}{\mathbf{Y}}
\newcommand{\bZ}{\mathbf{Z}}

\newcommand{\mbbA}{\mathbb{A}}
\newcommand{\mbbB}{\mathbb{B}}
\newcommand{\mbbC}{\mathbb{C}}
\newcommand{\mbbD}{\mathbb{D}}
\newcommand{\mbbE}{\mathbb{E}}
\newcommand{\mbbF}{\mathbb{F}}
\newcommand{\mbbG}{\mathbb{G}}
\newcommand{\mbbH}{\mathbb{H}}
\newcommand{\mbbI}{\mathbb{I}}
\newcommand{\mbbL}{\mathbb{L}}
\newcommand{\mbbM}{\mathbb{M}}
\newcommand{\mbbN}{\mathbb{N}}
\newcommand{\mbbW}{\mathbb{W}}
\newcommand{\mbbY}{\mathbb{Y}}
\newcommand{\mbbX}{\mathbb{X}}
\newcommand{\mbbZ}{\mathbb{Z}}

%lower case greek letters
\newcommand{\ga}{\alpha}
\newcommand{\gb}{\beta}
\newcommand{\gc}{\gamma}
\newcommand{\gd}{\delta}
\newcommand{\gep}{\varepsilon}
\newcommand{\gz}{\zeta}
\newcommand{\geta}{\eta}
\newcommand{\gth}{\theta}
\newcommand{\gi}{\iota}
\newcommand{\gv}{\nu}
\newcommand{\gk}{\kappa}
\newcommand{\gl}{\lambda}
\newcommand{\gm}{\mu}
\newcommand{\gn}{\nu}
\newcommand{\gx}{\xi}
\newcommand{\gp}{\pi}
\newcommand{\gr}{\rho}
\newcommand{\gs}{\sigma}
\newcommand{\gt}{\ensuremath{\tau}}
\newcommand{\gu}{\upsilon}
% \newcommand{\gph}{\varphi}
\newcommand{\gch}{\chi}
\newcommand{\gps}{\psi}
\newcommand{\go}{\omega}
\newcommand{\gto}{\ensuremath{\bar\tau}}

%bold lower case greek letters
%\newcommand\ssn{\mbox{\boldmath $\eta$}}
\newcommand{\bga}{\mbox{\boldmath $\alpha$}}
\newcommand{\bgb}{\mbox{\boldmath $\beta$}}
\newcommand{\bgc}{\mbox{\boldmath $\gamma$}}
\newcommand{\bgp}{\mbox{\boldmath $\pi$}}
\newcommand{\bgd}{\mbox{\boldmath $\delta$}}
\newcommand{\bge}{\mbox{\boldmath $\epsilon$}}
\newcommand{\bgs}{\mbox{\boldmath $\sigma$}}
\newcommand{\bgt}{\mbox{\boldmath $\tau$}}
\newcommand{\bgr}{\mbox{\boldmath $\rho$}}
\newcommand{\bgch}{\mbox{\boldmath $\chi$}}
\newcommand{\bgo}{\mbox{\boldmath $\omega$}}

%upper case greek letters
\newcommand{\gG}{\Gamma}
\newcommand{\gF}{\Phi}
\newcommand{\gD}{\Delta}
\newcommand{\gT}{\Theta}
\newcommand{\gP}{\Pi}
\newcommand{\gX}{\Xi}
\newcommand{\gS}{\Sigma}
\newcommand{\gO}{\Omega}
\newcommand{\gL}{\Lambda}

\newcommand\rA{{\mathrm{A}}}
\newcommand\rB{{\mathrm{B}}}
\newcommand\rC{{\mathrm{C}}}
\newcommand\rD{{\mathrm{D}}}
\newcommand\rE{{\mathrm{E}}}
\newcommand{\rF}{\mathrm{F}}
\newcommand\rG{{\mathrm{G}}}
\newcommand\rH{{\mathrm{H}}}
\newcommand\rI{{\mathrm{I}}}
\newcommand\rL{{\mathrm{L}}}
 
%\newcommand\ra{{\mathrm{a}}}
\newcommand\rb{{\mathrm{b}}}
\newcommand\rc{{\mathrm{c}}}
\newcommand\rd{{\mathrm{d}}}
\newcommand\re{{\mathrm{e}}}
\newcommand{\rf}{\mathrm{f}}
\newcommand\rg{{\mathrm{g}}}
\newcommand\rh{{\mathrm{h}}}
\newcommand\ri{{\mathrm{i}}}
\newcommand\rl{{\mathrm{l}}}
\newcommand\mrm{{\mathrm{m}}}
\newcommand\rn{{\mathrm{n}}}
\newcommand\ro{{\mathrm{o}}}
\newcommand\rp{{\mathrm{p}}}
%\newcommand\rq{{\mathrm{q}}}
\newcommand\rr{{\mathrm{r}}}
\newcommand\rs{{\mathrm{s}}}
\newcommand\rt{{\mathrm{t}}}

\newcommand{\cA}{\mathcal{A}}
\newcommand{\cB}{\mathcal{B}}
\newcommand{\cC}{\mathcal{C}}
\newcommand{\cD}{\mathcal{D}}
\newcommand{\cE}{\mathcal{E}}
\newcommand{\cF}{\mathcal{F}}
\newcommand{\cG}{\mathcal{G}}
\newcommand{\cH}{\mathcal{H}}
\newcommand{\cI}{\mathcal{I}}
\newcommand{\cJ}{\mathcal{J}}
\newcommand{\cK}{\mathcal{K}}
\newcommand{\cL}{\mathcal{L}}
\newcommand{\cM}{\mathcal{M}}
\newcommand{\cN}{\mathcal{N}}
\newcommand{\cO}{\mathcal{O}}
\newcommand{\cP}{\mathcal{P}}
\newcommand{\cQ}{\mathcal{Q}}
\newcommand{\cR}{\mathcal{R}}
\newcommand{\cS}{\mathcal{S}}
\newcommand{\cT}{\mathcal{T}}
\newcommand{\cU}{\mathcal{U}}
\newcommand{\cV}{\mathcal{V}}
\newcommand{\cW}{\mathcal{W}}
\newcommand{\cX}{\mathcal{X}}
\newcommand{\cY}{\mathcal{Y}}
\newcommand{\cZ}{\mathcal{Z}}

\newenvironment{myitem}%
{\begin{list}%
       {-}%
       {\setlength{\itemsep}{0pt}
     \setlength{\parsep}{3pt}
     \setlength{\topsep}{3pt}
     \setlength{\partopsep}{0pt}
     \setlength{\leftmargin}{0.7em}
     \setlength{\labelwidth}{1em}
     \setlength{\labelsep}{0.3em}}}%
{\end{list}}

\newenvironment{myitemize}%
{\begin{list}%
       {-}%
       {\setlength{\itemsep}{0pt}
     \setlength{\parsep}{2pt}
     \setlength{\topsep}{2pt}
     \setlength{\partopsep}{0pt}
     \setlength{\leftmargin}{2em}
     \setlength{\labelwidth}{1em}
     \setlength{\labelsep}{0.3em}}}%
{\end{list}}

%Alberto's macros
\newcommand{\ls}[2]{\langle #2 / #1\rangle} % linear substitution
\newcommand{\cs}[2]{\{ #2 / #1\}} % classical substitution
%\newcommand{\ls}[2]{\langle #1:=#2\rangle} % linear substitution
%\newcommand{\cs}[2]{\{ #1:=#2\}} % classical substitution
\newcommand{\Bag}[1]{[#1]} % bag formation
\renewcommand{\smallsetminus}{-}
%Giulio's macros
%Sets:
%\newcommand{\nat}{\mathcal{N}}
\newcommand{\mbz}{\mathbf{0}}
\newcommand{\mbo}{\mathbf{1}}
\newcommand{\mbt}{\mathbf{2}}
\newcommand{\rea}[1]{\mathsf{rea}(#1)} % set of realizers of #1
\newcommand{\realize}{\Vdash} % realizability relation
\newcommand{\natp}{\nat^+}
\newcommand{\one}{\mathbf{1}}
\newcommand{\bool}{\mathbf{2}}
\newcommand{\perm}[1]{\fS_{#1}}
\newcommand{\card}[1]{\# #1} % cardinality of a set
%Boh
\newcommand{\Omegatuple}[1]{\Mfin{#1}^{(\omega)}}
\newcommand{\Pow}[1]{\cP(#1)}
\newcommand{\Powf}[1]{\cP_{\mathrm{f}}(#1)}
\newcommand{\Id}[1]{\mathrm{Id}_{#1}}
\newcommand{\comp}{\circ}
\newcommand{\With}[2]{{#1}\with{#2}}
\newcommand{\Termobj}{1}
\newcommand{\App}{\mathrm{Ap}}
\newcommand{\Abs}{\uplambda}
\newcommand{\Funint}[2]{[{#1}\!\!\imp\!\!{#2}]}

%Lambda calculus:
%\newcommand{\full}{\gto{\bang}}
%\newcommand{\dlam}{\ensuremath{\partial\lambda}}
%\newcommand{\dzlam}{\ensuremath{\partial_0\lambda}}
%\newcommand{\lam}{\ensuremath{\lambda}}
%\newcommand{\bang}{\oc}
%\newcommand{\hole}[1]{\llparenthesis #1\rrparenthesis}
\newcommand{\paral}{\vert}
\newcommand{\FSet}[1]{\Lambda^{#1}_{\bang}}
\newcommand{\supp}[1]{\mathsf{su}(#1)} % support of multises

%\newcommand{\tContSet}{\Set{\gt}\hole{\cdot}} % bang-free test contexts
%\newcommand{\tFContSet}{\FSet{\gt}\hole{\cdot}} % all test contexts

%\newcommand{\ContSet}{\Set{\gt}\hole{\cdot}} % bang-free term contexts
%\newcommand{\FContSet}{\FSet{\gt}\hole{\cdot}} % all term contexts

\newcommand{\sums}[1]{\bool\langle\Set{#1}\rangle}
\newcommand{\Fsums}[1]{\bool\langle\FSet{#1}\rangle}
\newcommand{\la}{\leftarrow}
\newcommand{\ot}{\leftarrow}
\newcommand{\labelot}[1]{\ _{#1}\!\leftarrow} % left arrow with label
\newcommand{\labelto}[1]{\rightarrow_{#1}} % right arrow with label
\newcommand{\mslabelot}[1]{\ _{#1}\!\twoheadleftarrow} % left two head arrow with label
\newcommand{\mslabelto}[1]{\twoheadrightarrow_{#1}} % right two head arrow with label
\newcommand{\msla}{\twoheadleftarrow} 
\newcommand{\msto}{\twoheadrightarrow}
\newcommand{\toh}{\to_{h}} % head reduction
\newcommand{\mstoh}{\msto_{h}} % transitive head reduction
\newcommand{\etoh}{\to_{h\eta}} % extensional head reduction
\newcommand{\msetoh}{\msto_{h\eta}} % extensional transitive head reduction
\newcommand{\too}{\to_{o}} % outer-reduction
\newcommand{\mstoo}{\msto_{o}} % transitive outer-reduction
\newcommand{\etoo}{\to_{o\eta}} % extensional outer-reduction
\newcommand{\msetoo}{\msto_{o\eta}} % extensional transitive outer-reduction
\newcommand{\eqt}{=_{\theta}} % weakly extensional conversion
\newcommand{\eqte}{=_{\theta\eta}} % extensional conversion
\newcommand{\eq}{=} % basic conversion

\newcommand{\dg}[2]{\mathrm{deg}_{#1}(#2)} % degree of a variable #1 in a term #2

\newcommand{\obsle}{\sqsubseteq_{\mathcal{O}}} % observational preorder
\newcommand{\obseq}{\approx_{\mathcal{O}}} % observational equivalence

\newcommand{\tesle}{\sqsubseteq_{\mathcal{C}}} % convergence preorder
\newcommand{\teseq}{\approx_{\mathcal{C}}} % convergence equivalence

\newcommand{\Fobsle}{\sqsubseteq^{\bang}_{\mathcal{O}}} % full observational preorder
\newcommand{\Fobseq}{\approx^{\bang}_{\mathcal{O}}} % full observational equivalence

\newcommand{\Ftesle}{\sqsubseteq^{\bang}_{\mathcal{C}}} % full convergence preorder
\newcommand{\Fteseq}{\approx^{\bang}_{\mathcal{C}}} % full convergence equivalence

%Semantics:
\newcommand{\rank}[1]{\mathsf{rk}(#1)} % rank of something
\newcommand{\rrank}[1]{\mathsf{rrk}(#1)} % right rank of an implicative formula
\newcommand{\lrank}[1]{\mathsf{lrk}(#1)} % left rank of an implicative formula
%\newcommand{\termin}[1]{\mathsf{t}(#1)} % set of terminals of a set of formulas
\newcommand{\termin}[3]{\mathsf{tmn}_{#1}^{#2}(#3)} % set of terminals of a set of formulas. The first argument is a tuple of terms to be substituted for the tuple of variables given in the second argument. The third argument is the formula of which we take the terminals 
\newcommand{\conc}[1]{\mathsf{cn}(#1)} % set of premisses of a set of formulas
\newcommand{\prem}[3]{\mathsf{pr}_{#1}^{#2}(#3)} % set of premisses of a set of formulas
\newcommand{\premp}[1]{\mathsf{pp}(#1)} % special premisses of premisses of a set of formulas
\newcommand{\premterm}[3]{\mathsf{prt}_{#1}^{#2}(#3)} % set of premisses having terminals in common with set of formulas #1
\newcommand{\spnex}[1]{\overline{#1}} % semi-prenex form of the formula #1
\newcommand{\ospnex}[1]{\overline{\overline{#1}}} % semi-prenex form of the formula #1 deprived of all universal quantifiers at the front
\newcommand{\forant}{\mathsf{uqa}} % one step semi-prenex form of the formula #1
\newcommand{\wrap}[1]{\bar{#1}} % wrapping of a term
\newcommand{\len}{\ell}
\newcommand{\trm}[1]{#1^{\textrm{--}}}
\newcommand{\cont}[2]{#1^{+}\hole{#2}}
\newcommand{\Mfin}[1]{\mathcal{M}_{\mathrm{f}}(#1)}
\newcommand{\mcup}{\uplus}
\newcommand{\mmcup}{\bar{\mcup}}
% \newcommand{\Pair}[2]{\langle{#1},{#2}\rangle}
\newcommand{\Rel}{\mathbf{REL}} %category of sets and relations
\newcommand{\MRel}{\mathbf{REL}_{\bang}} %Kleisli category of sets and relations
\newcommand{\Inf}{\mathbf{Inf}} %category of information system and approx rels
\newcommand{\SD}{\mathbf{SD}} %category of Scott domains and continuous functions
\newcommand{\CPO}{\mathbf{CPO}} %category of CPOs and continuous functions
\newcommand{\SL}{\mathbf{ScottL}} %category of preorders
\newcommand{\SLb}{\mathbf{ScottL}_{\bang}} %Kleisli category of \SL
\newcommand{\Coh}{\mathbf{Coh}} %category of coherent spaces
\newcommand{\Cohb}{\mathbf{Coh}_{\bang}} %Kleisli category of \Coh


\newcommand{\otspam}{
\mathrel{\vcenter{\offinterlineskip
\vskip-.130ex\hbox{\begin{turn}{180}$\mapsto$\end{turn}}}}} % reversed mapsto

\newcommand{\envup}[3]{#1[#2 \otspam #3]} % environment update

\newcommand{\try}[2]{\mathsf{try}_{#1}\{#2\}} % execute the second argument first argument until the second one is found
\newcommand{\catch}[2]{\mathsf{catch}_{#1}\{#2\}} % when the first argument is found, execute the second one

\newcommand{\Lamex}{\Lambda_{\mathsf{ex}}} % lambda calculus with try and catch

\renewcommand{\iff}{\Leftrightarrow}
\newcommand{\imp}{\Rightarrow}
\newcommand{\Apex}[1]{^{\: #1}}

\newcommand{\compl}[1]{{#1}^c} % complement of a set
\newcommand{\pts}{.\,.\,} % points abbreviated
%\newcommand{\conv}[1]{{#1}\!\downarrow} % covergence
\newcommand{\convh}[1]{{#1}\!\downarrow_h} % head covergence
\newcommand{\solv}[1]{#1\lightning} % solvance
\newcommand{\solvo}[1]{#1\lightning_o} % outer solvance
\newcommand{\module}[1]{\bool\langle #1 \rangle}

\newcommand{\Ide}[1]{Ide(#1)} % set of all ideals of a preorder

\newcommand{\Bstk}{\bB_{\mathsf{s}}} % quasi-boolean algebra of saturated sets of stacks
\newcommand{\fsubseteq}{\subseteq_\mathrm{f}} % finite subset
\newcommand{\Ps}[1]{\cP(#1)} % powerset
\newcommand{\Pss}[1]{\cP_\mathrm{s}(#1)} % set of all saturated subsets
\newcommand{\Psc}[1]{\cP_\mathrm{c}(#1)} % set of all closed subsets
\newcommand{\Psg}[1]{\cP_\mathrm{g}(#1)} % set of all good subsets
\newcommand{\Psf}[1]{\cP_\mathrm{f}(#1)} % set of all finite subsets
\newcommand{\Ms}[1]{\cM(#1)} % set of all multisets
\newcommand{\Msf}[1]{\cM_\mathrm{f}(#1)} % set of all finite multisets
\newcommand{\fst}{\mathsf{fst}} % reduction proper to the \Lambda\mu-calculus
\newcommand{\cons}{::} % stack constructor
\newcommand{\at}{\!\centerdot} % stack constructor (cons)
\newcommand{\ats}{\at\ldots\at} % stack constructor (cons) with lower suspension dots 
%\newcommand{\at}{\!::\!} % stack constructor

%\newcommand{\meet}{\} % inf operator
%\newcommand{\join}{\!\centerdot} % inf operator

\newcommand{\sps}[3]{\bgp^{(#1,#2,#3)}} % special stack defined as \overbrace{\cadr{#1}{0}\at\ldots\at\cadr{#1}{0}}^{#3 \mbox{ times}}\at #1
\newcommand{\spt}[1]{\bA^{(#1)}} % special term defined as \bd\epsilon.\cadr{\gd}{0}\ap(\cadr{\epsilon}{0}\at\ldots\at\cadr{\epsilon}{q-1}}\at\cddr{\epsilon}{q})

% \newcommand{\cdr}[1]{\mathsf{cdr}(#1)} % tail of stack
% \newcommand{\car}[1]{\mathsf{car}(#1)} % head of stack
% \newcommand{\itcdr}[2]{#1[#2)} % iterated tail of stack
% \newcommand{\cddr}[2]{#1[#2)} % iterated tail of stack
% \newcommand{\cadr}[2]{#1[#2]} % head of an iterated tail of stack

\newcommand{\op}{\mathsf{op}} % generic binary infix operator
\newcommand{\fun}[1]{\mathsf{f}(#1)} % generic unary function symbol
\newcommand{\nil}{\mathsf{nil}} % empty stack
\newcommand{\mcddr}[2]{\mathsf{cdr}^{#1}(#2)} % modified iterated tail of stack
\newcommand{\mitcar}[2]{\mathsf{car}^{#1}(#2)} % modified iterated head of stack
\newcommand{\mitcdr}[2]{\mathsf{cdr}^{#1}(#2)} % modified iterated tail of stack

\newcommand{\callcc}{\mathsf{cc}} % Felleisen's call/cc
\newcommand{\kpi}[1]{\mathsf{k}_{#1}} % Krivine's term that restores the stack
\newcommand{\nf}[1]{\mathsf{Nf}(#1)} % partial function returning the normal form 
\newcommand{\onf}[1]{\mathsf{Onf}(#1)} % partial function returning the outer normal form
\newcommand{\eonf}[1]{\eta\mathsf{Onf}(#1)} % partial function returning the extensional outer normal form
\newcommand{\hnf}[1]{\mathsf{Hnf}(#1)} % partial function returning the beta-head normal form
\newcommand{\ehnf}[1]{\eta\mathsf{Hnf}(#1)} % partial function returning the beta-eta head normal form
\newcommand{\Sol}{\mathsf{Sol}^{\mathsf{t}}} % set of all solvable terms
%\newcommand{\USol}{\mathsf{Sol}^{\mathsf{t}} % set of all solvable terms
\newcommand{\SetBT}{\mathfrak{B}} % set of all Bohm trees
\newcommand{\SetBTt}{\mathfrak{B}^{\mathsf{t}}} % set of all Bohm trees of \stk-terms
\newcommand{\BT}[1]{\mathsf{BT}(#1)} % Bohm tree of an expression
\newcommand{\tBT}[2]{\mathsf{BT}_{#2}(#1)} % truncated Bohm tree of an expression
\newcommand{\eBT}[1]{\eta\mathsf{BT}(#1)} % extensional Bohm tree of an expression
\newcommand{\teBT}[2]{\eta\mathsf{BT}_{#2}(#1)} % truncated extensional Bohm tree of an expression
\newcommand{\bdom}[2]{\mathsf{dom}(#1,#2)} % bounded domain of a term seen as a function over sequences of natural numbers
%\newcommand{\virt}[2]{\langle #1 \mid #2 \rangle} % virtual extension of the map corresponding to a term 
\newcommand{\bout}[3]{\mbox{\boldmath{$\langle$}} #1 \!\mid\! #2 \!\mid\! #3 \mbox{\boldmath{$\rangle$}}} % Bohm out term corresponding to a term #1, the sequence #2 , the bound #3 and the width #4
\newcommand{\vbout}[3]{\mbox{\boldmath{$\langle$}} #1 \!\mid\! #2 \!\mid\! #3 \mbox{\boldmath{$\rangle$}}} % virtual Bohm out term corresponding to a term #1, the sequence #2 and the bound #3 
\newcommand{\virt}[1]{\mathsf{vir}(#1)} % set of sequences that belong virtually to the map corresponding to a term
\newcommand{\bvirt}[2]{\mathsf{vir}(#1,#2)} % set of sequences that belong virtually to the map corresponding to a term, with a bound on their length
\newcommand{\extr}[1]{\mathsf{extr}(#1)} % extensionally reachable sequences
\newcommand{\uns}[1]{\mathsf{uns}(#1)} % unsolvable sequences
\newcommand{\unr}[1]{\mathsf{unr}(#1)} % unreachable sequences
\newcommand{\eqty}{\stackrel{\infty}{=}} % equality of Bohm trees up to infinite eta-expansion
\newcommand{\simty}{\stackrel{\infty}{\sim}} % similarity at all sequences of natural numbers
\newcommand{\pexp}[2]{\mbox{\boldmath{$\langle$}} #1 \lVert #2 \mbox{\boldmath{$\rangle$}}} % path expansion of a term
\newcommand{\Seq}{Seq} % the set of finite sequences of strictly positive natural numbers
\newcommand{\tSeq}[1]{Seq_{\leq #1}} % the set of finite sequences of length less or equal to a specified bound
\newcommand{\Lab}{Lab} % the set of labels of Bohm trees

\newcommand{\NT}[1]{\mathsf{NT}(#1)} % Nakajima tree of an expression
\newcommand{\tNT}[2]{\mathsf{NT}_{#2}(#1)} % truncated Nakajima tree of an expression

\newcommand{\sub}[2]{\{#1/#2\}} % classical substitution of #1 for #2
\newcommand{\ab}[1]{\mathcal{A}(#1)} % abort of a term
\newcommand{\ctrl}[1]{\mathcal{C}(#1)} % control of a term
\newcommand{\cmd}[2]{\langle #1 \lVert #2\rangle} % command constructor for lambda mu-mu-tilde
\newcommand{\ap}{\star} % application symbol of a term to a process
\newcommand{\bd}{\kappa} % binder for stack variables
\newcommand{\lambdab}{\bar{\lambda}} % lambda bar of mu-mu tilde calculus
\newcommand{\mut}{\tilde{\mu}} % binder for mu tilde calculus
\newcommand{\tcbn}[1]{#1^{\circ}} % translation of the cbn lambdamumu expressions into stack expressions 
\newcommand{\tcbv}[1]{#1^{\bullet}} % translation of the cbv lambdamumu expressions into stack expressions 
\newcommand{\texp}[1]{#1^{\circ}} % translation of the lambdamu expressions into stack expressions
\newcommand{\ttyp}[1]{#1^{\circ}} % translation of the lambdamu types into stack types
\newcommand{\Tp}[1]{#1^{\circ}} % translation
\newcommand{\Te}[1]{#1^{\circ}} % translation of the lambdamu expressions into stack expressions
\newcommand{\Neg}[1]{#1^{-}} % negative translation of formulas
\newcommand{\Pos}[1]{#1^{+}} % positive translation of formulas
\newcommand{\Tt}[1]{#1^{-}} % translation of stack expressions into lambda calculus with pairing
\newcommand{\Ts}[1]{#1^{+}} % translation of the lambda mu calculus into the lambda calculus
\newcommand{\dev}[1]{#1^{\baro}} % inner-outer development of the redexes of #1
\newcommand{\AtForm}{\mathrm{AtFm}} % set of atomic formulas of first-order logic
\newcommand{\UqAtForm}{\mathrm{UqAtFm}} % the set of universally quantified atomic formulas of first-order logic
\newcommand{\UqBot}{\mathrm{UqBot}} % the set of universally quantified atomic formulas of first-order logic in which the atomic formula is $\bot$
\newcommand{\Form}{\mathrm{Fm}} % set of formulas of second-order logic
\newcommand{\cForm}{\mathrm{Fm}^\mathsf{o}} % set of closed formulas of second-order logic
\newcommand{\Val}[1]{\mathrm{Val}_{#1}} % set of valuations into the structure #1
\newcommand{\At}{\mathrm{At}} % atomic formulas
\newcommand{\cAt}{\mathrm{At}^\mathsf{o}} % closed atomic formulas
\newcommand{\Var}{\mathrm{Var}} % set of variables
\newcommand{\Nam}{\mathrm{Nam}} % set of names
\newcommand{\FV}{\mathrm{FV}} % free variables
\newcommand{\FN}{\mathrm{FN}} % free names
\newcommand{\LTer}[1]{\Lambda^{\mathsf{#1}}} % set of terms of the lambda-mu calculus
\newcommand{\LTyp}[1]{\cT_{\lambda\mu}^{\mathsf{#1}}} % set of types of the lambda-mu calculus
\newcommand{\ITer}[1]{\Sigma_{\mathsf{in}}^{\mathsf{#1}}} % set of intuitionistic terms of the stack calculus
\newcommand{\BTer}[1]{\Sigma_{\mathsf{b}}^{\mathsf{#1}}} % finite Bohm trees of the stack calculus
\newcommand{\KTer}[1]{\Sigma^{\mathsf{#1}}} % set of terms of the stack calculus
\newcommand{\KTyp}[1]{\cT_\bd^{\mathsf{#1}}} % set of types of the stack calculus

%\newcommand{\Kstate}[4]{\langle({#1},{#2}),({#3},{#4})\rangle} % a state of the Krivine Abstract Machine involving a term 
\newcommand{\transition}{\longrightarrow} % transition symbol from one state of the Krivine Abstract Machine to another
\newcommand{\Kstate}[3]{\mbox{\boldmath{$\langle$}} \ {#1},{#2},{#3} \ \mbox{\boldmath{$\rangle$}}} % a state of the Krivine Abstract Machine involving a term 

\newcommand{\Kproc}[2]{\langle{#1},{#2}\rangle} % a state of the Krivine Abstract Machine involving a process
\newcommand{\Kclos}[2]{({#1},{#2})} % a closure of the Krivine Abstract Machine

\newcommand{\SN}[1]{\mathrm{SN}^{\mathsf{#1}}} % set of strongly normalizing expressions of the stack calculus

%\newcommand{\deg}[2]{\mathsf{deg}_{#1}(#2)} % degree of a variable in a n expression

\newcommand{\dgr}[2]{\mathsf{deg}_{#1}(#2)} % degree of a variable in a n expression

\newcommand{\bbot}{
\mathrel{\vcenter{\offinterlineskip
\vskip-.130ex\hbox{\begin{turn}{90}$\models$\end{turn}}}}} % Krivine's double bottom

\newcommand{\ttop}{
\mathrel{\vcenter{\offinterlineskip
\vskip-.130ex\hbox{\begin{turn}{270}$\models$\end{turn}}}}} % double top

\newcommand{\sepa}{
\mathrel{\vcenter{\offinterlineskip
\vskip-.130ex\hbox{\begin{turn}{90}$\succ$\end{turn}}}}} % separability

\newcommand{\asm}{\! : \!} % separator for type assumptions in contexts
\newcommand{\tass}{:} % separator type assignment in judgements

\newcommand{\tval}[1]{\vert #1\vert} % truth value interpretation of types into sets of terms
\newcommand{\fval}[1]{\lVert #1 \rVert} % falsehood value interpretation of types into sets of stacks
\newcommand{\tInt}[1]{\vert #1\vert} % truth value-like interpretation of term types into set of terms
\newcommand{\sInt}[1]{\vert #1\vert} % falsehood value-like interpretation of stack types into set of stacks
\newcommand{\pInt}[1]{\vert #1 \vert} % interpretation of the process type into a set of processes
\newcommand{\eInt}[1]{\vert #1 \vert} % interpretation of the expression type into a set of expressions
\newcommand{\Int}[1]{\llbracket #1\rrbracket} % interpretation of expressions in a mathematical domain
\newcommand{\id}{\mathsf{id}} % identity morphsism in a category
\newcommand{\pr}[1]{\mathsf{pr}_{#1}} % i-th projection of a cartesian product
\newcommand{\ev}{\mathsf{ev}} % evaluation morphism of a ccc

%\newcommand{\list}[1]{\langle #1 \rangle} % list constructor write inside the arguments separated by commas
\newcommand{\lis}[1]{\prec #1 \succ} % list constructor
\newcommand{\copair}[2]{[ #1, #2 ]} % copair constructor

\newcommand{\cur}[1]{\Lambda(#1)} % currying natural isomorphism
\newcommand{\invcur}[1]{\Lambda^{-1}(#1)} % inverse of currying natural isomorphism
\newcommand{\adbmaL}{
\mathrel{\vcenter{\offinterlineskip
\vskip-.100ex\hbox{\begin{turn}{180}$\Lambda$\end{turn}}}}}

\newcommand{\ctrliso}[1]{\phi(#1)} % natural isomorphism proper to control categories
\newcommand{\invctrliso}[1]{\phi^{-1}(#1)} % inverse of the natural isomorphism proper to control categories

\newcommand{\cocur}[1]{\adbmaL\!\!(#1)} % co-currying natural isomorphism
\newcommand{\invcocur}[1]{\adbmaL^{-1}(#1)} % inverse of co-currying natural isomorphism

\newcommand{\cord}{\sqsubseteq_c} % computational order on Bohm trees
\newcommand{\lord}{\sqsubseteq_l} % logical order on Bohm trees

\newcommand{\coher}{\stackrel{\frown}{\smile}} % Girard's coherence relation
\newcommand{\scoher}{\frown} % Girard's strict coherence relation

\newcommand{\Cl}[1]{Cl(#1)} % the set of cliques of a set 

\newcommand{\ccl}{\ensuremath{CCL}} % name of classical combinatory logic
\newcommand{\lmuo}{\ensuremath{\lambda\mu\mathbf{1}}} % name of Andou's lambda-mu calculus 
\newcommand{\lmc}{\ensuremath{\lambda C}} % name of Herbelin-De Groote's lambda-C calculus 
\newcommand{\lamb}{\ensuremath{\lambda}} % name of Church's lambda calculus 
\newcommand{\lmu}{\ensuremath{\lambda\mu}} % name of Parigot's lambda-mu calculus 
\newcommand{\stk}{\ensuremath{\bd}} % name of the stack calculus
\newcommand{\stke}{\ensuremath{\bd\eta}} % name of the extensional stack calculus
\newcommand{\stkw}{\ensuremath{\bd w}} % name of the stack calculus + weta
\newcommand{\lsp}{\ensuremath{\lambda\mathsf{sp}}} % name of the lambda calculus with surjective pairing
\newcommand{\lesp}{\ensuremath{\lambda\eta\mathsf{sp}}} % name of the extensional lambda calculus with surjective pairing
\newcommand{\ort}[1]{#1^{\bot}} % orthogonal object

\newcommand{\wi}{\binampersand} % with connective
\newcommand{\pa}{\bindnasrepma} % par connective

\newcommand{\te}{\mathsf{ten}} % tensor morphism
\newcommand{\parm}{\mathsf{par}} % par morphism

\newcommand{\mon}{\mathsf{m}} % monoidality morphism
\newcommand{\see}{\mathsf{s}} % seely isomorphism
\newcommand{\ut}{\mathsf{t}} % terminal morphism in a Cartesian category

\newcommand{\assoc}{\mathsf{ass}} % generalized associativity morphism

\newcommand{\der}{\mathsf{der}} % dereliction morphism
\newcommand{\coder}{\mathsf{cod}} % codereliction morphism

\newcommand{\coa}{\mathsf{h}} % coalgebra for the functor \ort{(\cdot)} \xrightarrow{\cdot}

\newcommand{\con}{\mathsf{con}} % contraction morphism
\newcommand{\wkn}{\mathsf{wkn}} % weakening morphism
\newcommand{\cowkn}{\mathsf{cow}} % coweakening morphism

\newcommand{\nco}{\overline{\mathsf{con}}} % negative contraction morphism
\newcommand{\nwk}{\overline{\mathsf{wkn}}} % negative weakening morphism

\newcommand{\dig}{\mathsf{dig}} % digging morphism

%\newcommand{\codig}{\mathsf{cod}} % codigging morphism

\newcommand{\dual}{\partial} % dualizing morphism
\newcommand{\ddual}{\partial^{-1}} % inverse of the dualizing morphism

\newcommand{\teid}{\mathbf{1}} % identity of the tensor product
\newcommand{\bon}{\mathbf{1}} % identity of the tensor product 

\newcommand{\DEC}{\mathsf{DEC}} % the class of decidable languages
\newcommand{\SDEC}{\mathsf{SDEC}} % the class of semi-decidable languages
\newcommand{\REG}{\mathsf{REG}} % the class of regular languages
\newcommand{\CFL}{\mathsf{CFL}} % the class of context-free languages
\newcommand{\dCFL}{\mathsf{dCFL}} % the class of deterministically context-free languages

\newcommand{\sqb}[1]{[#1]} % square brackets

\newcommand{\cnt}[1]{#1^\bullet} % center of a control category
\newcommand{\foc}[1]{#1^\sharp} % focus of a control category

\newcommand{\com}{\mathsf{comp}} % composition proof
\newcommand{\exc}{\mathsf{exc}} % exchange proof

\newcommand{\ax}{\mathsf{ax}} % axiom rule
\newcommand{\dne}{\mathsf{dne}} % double negation elimination rule
\newcommand{\raa}{\mathsf{raa}} % reductio ad absurdum rule
\newcommand{\efq}{\mathsf{efq}} % ex flaso quodlibet rule
\newcommand{\cut}{\mathsf{cut}} % cut rule
\newcommand{\dni}{\mathsf{dni}} % double negation introduction proof
\newcommand{\mpo}{\mathsf{mp}} % modus ponens rule
\newcommand{\dt}{\mathsf{dt}} % deduction theorem
\newcommand{\idem}{\mathsf{id}} % identity proof
\newcommand{\contp}{\mathsf{contp}} % contraposition proof (positive)
\newcommand{\contn}{\mathsf{contn}} % contraposition proof (negative)
\newcommand{\sded}{\mathsf{sded}} % symmetric deduction proof

\newcommand{\varrule}{\mathsf{ax}} % inference rule for variables
\newcommand{\carrule}{\to e_r} % inference rule for \car
\newcommand{\cdrrule}{\to e_l} % inference rule for \cdr
\newcommand{\atrule}{\to i} % inference rule for \at
\newcommand{\aprule}{\mathsf{cut}} % inference rule for \app
\newcommand{\nilrule}{\bot i} % inference rule for \nil
\newcommand{\bdrule}[1]{\bd,{#1}} % inference rule for \bd with reference to the bound variable 
\newcommand{\orrule}{\vee i} % inference rule for or
\newcommand{\rsallrule}{2\forall r} % inference rule for right introduction of the second order universal quantifier
\newcommand{\rfallrule}{\forall r} % inference rule for right introduction of the first order universal quantifier
\newcommand{\lsallrule}{2\forall l} % inference rule for left introduction of the second order universal quantifier
\newcommand{\lfallrule}{\forall l} % inference rule for left introduction of the first order universal quantifier

\newcommand{\leng}[1]{\sharp #1} % length of a sequence

%\newcommand{\wid}[1]{\mathsf{w}(#1)} % width of a term
%\newcommand{\bwid}[2]{\mathsf{w}(#1,#2)} % bounded width of a term
\newcommand{\wei}[1]{\mathsf{w}(#1)} % weight of a term
\newcommand{\bwei}[2]{\mathsf{w}(#1,#2)} % bounded weight of a term
\newcommand{\brea}[1]{\mathsf{b}(#1)} % breadth of a term
\newcommand{\bbrea}[2]{\mathsf{b}(#1,#2)} % bounded breadth of a term
\newcommand{\gap}[1]{\mathsf{g}(#1)} % gap of a term
\newcommand{\bgap}[2]{\mathsf{g}(#1,#2)} % bounded gap of a term

\newcommand{\wnot}{?} % why not modality
\newcommand{\bang}{!} % bang modality
\newcommand{\bbang}{!!} % double bang functor
\newcommand{\app}{F} % morphism from $U \to U \Rightarrow U$
\newcommand{\lam}{G} % morphism from $U \Rightarrow U \to U$
%\newcommand{\cur}{\Lambda} % currying
\newcommand{\cld}{\downarrow\!} % down arrow closure operator
\newcommand{\clu}{\uparrow\!} % up arrow closure operator
\newcommand{\clo}[1]{\overline{#1}} % overline closure operator
\newcommand{\clde}{\downarrow_\eta\!} % closure operator for the extensionality preorder
\newcommand{\parcl}[1]{\uparrow_{#1}\!} % parameterized closure operator
\newcommand{\cldn}[2]{\downarrow_{#1}\!{#2}} % downwards closure operator
\newcommand{\clup}[2]{\uparrow_{#1}\!{#2}} % upwards closure operator
\newcommand{\opp}[1]{{#1}^{\mathsf{op}}} % opposite

%Macro for stack sequents. The forms of annotated sequents are 
%\tystk{s}{stack}{stack-type}{context}
%\tystk{t}{term}{term-type}{context}
%\tystk{p}{process}{process-type}{context}
%Sequents without annotations
%\tystk{s}{}{stack-type}{context}
%\tystk{t}{}{term-type}{context}
%\tystk{p}{}{}{context}
\newcommand{\tystk}[4]{%
\ifthenelse{\equal{#1}{s}\OR\equal{#1}{t}}{
	\ifthenelse{\equal{#1}{s}}{%\equal{#1}{s}
		\ifthenelse{\isempty{#2}}{#3 \vdash #4}{\textcolor{blue}{#2} \textcolor{blue}{:} #3 \vdash #4}
		}{%\equal{#1}{t}
		\ifthenelse{\isempty{#2}}{\vdash #3, #4}{\vdash \textcolor{blue}{#2} \textcolor{blue}{:} #3 \ \textcolor{blue}{\mid} \ #4}
		}
	}{%\equal{#1}{p}
	\ifthenelse{\isempty{#2}}{\vdash #4}{\vdash \textcolor{blue}{#2} \textcolor{blue}{\mid} #4}
	}
}
\newcommand{\ntystk}[4]{%
\ifthenelse{\equal{#1}{s}\OR\equal{#1}{t}}{
	\ifthenelse{\equal{#1}{s}}{%\equal{#1}{s}
		\ifthenelse{\isempty{#2}}{#3 \nvdash #4}{\textcolor{blue}{#2} \textcolor{blue}{:} #3 \nvdash #4}
		}{%\equal{#1}{t}
		\ifthenelse{\isempty{#2}}{\nvdash #3, #4}{\nvdash \textcolor{blue}{#2} \textcolor{blue}{:} #3 \ \textcolor{blue}{\mid} \ #4}
		}
	}{%\equal{#1}{p}
	\ifthenelse{\isempty{#2}}{\nvdash #4}{\nvdash \textcolor{blue}{#2} \textcolor{blue}{\mid} #4}
	}
}

%Macro for lambda-mu sequents. The form is 
%\tylmu{left_context}{expression}{expression-type}{right_context}
\newcommand{\tylmu}[4]{
	\ifthenelse{\isempty{#3}}
	{%if
	\ifthenelse{\isempty{#4}}
		{#1 \vdash_{\lmu} \textcolor{blue}{#2}}
		{#1 \vdash_{\lmu} \textcolor{blue}{#2} \mid #4}
	}
	{%else
        \ifthenelse{\isempty{#4}}
        	{#1 \vdash_{\lmu} \textcolor{blue}{#2} \textcolor{blue}{:} #3}
        	{#1 \vdash_{\lmu} \textcolor{blue}{#2} \textcolor{blue}{:} #3 \textcolor{blue}{\mid} #4}
	}
}

%Macros for lambda-mu-mu-tilde sequents.
 
%\tylmmcom{command}{left_context}{right_context}
\newcommand{\tylmmcom}[3]{\textcolor{blue}{#1}\ \textcolor{blue}{\triangleright} #2 \vdash #3}
%\tylmmter{left_context}{term}{right_active_formula}{right_context}
\newcommand{\tylmmter}[4]{#1 \vdash \textcolor{blue}{#2} \textcolor{blue}{:} #3 \mid #4}
%\tylmmenv{left_context}{environment}{left_active_formula}{right_context}
\newcommand{\tylmmenv}[4]{#1 \mid \textcolor{blue}{#2} \textcolor{blue}{:} #3 \vdash #4}

%Macro for lambda-mu-one sequents. The form is 
%\tylmuo{left_context}{expression}{expression-type}
%\newcommand{\tylmuo}[3]{#1 \vdash_{\lmuo} \textcolor{blue}{#2} \textcolor{blue}{:} #3}
\newcommand{\tylmuo}[3]{#1 \vdash \textcolor{blue}{#2} \textcolor{blue}{:} #3}

%Macro for lambda sequents, i.e. typed lambda terms. The form is 
%\tylamb{left_context}{expression}{expression-type}
\newcommand{\tylamb}[3]{#1 \vdash_{\lamb} \textcolor{blue}{#2} \textcolor{blue}{:} #3}

%Macro for lambda-c sequents. The form is 
%\tylmc{left_context}{expression}{expression-type}
\newcommand{\tylmc}[3]{#1 \vdash_{\lmc} \textcolor{blue}{#2} \textcolor{blue}{:} #3}

%Macro for ccl sequents. The form is 
%\tyccl{left_context}{expression}{expression-type}
\newcommand{\tyccl}[3]{#1 \vdash_{\ccl} \textcolor{blue}{#2} \textcolor{blue}{:} #3}

\newcommand{\prov}[2]{#1 \vdash #2} % provability symbol
\newcommand{\refu}[2]{#1 \dashv #2} % refutation symbol


%%%%%%%% MACRO PER LE NOTE DEL CORSO DI CALCOLABILITA'

\newcommand{\eclose}[1]{\mathsf{ecl}(#1)} % operatore di epsilon-chiusura
\newcommand{\zr}{\mathsf{Z}} % the constantly zero function
\newcommand{\suc}{\mathsf{S}} % the successor function
\newcommand{\pred}{\mathsf{P}} % the predecessor function
\newcommand{\prj}[2]{I_{#1}^{#2}} % the projection function
\newcommand{\ca}[1]{\mathsf{c}_{#1}} % the characteristic function of a predicate
\newcommand{\minus}{\stackrel{\centerdot}{-}} % the minus function on natural numbers
\newcommand{\conv}[1]{{#1}\!\downarrow} % convergence of a function
\newcommand{\dive}[1]{{#1}\!\uparrow} % divergence of a function
\newcommand{\PR}{\mathbf{PR}} % partial recursive functions
\newcommand{\REC}{\mathbf{REC}} % total recursive functions
\newcommand{\PRIMREC}{\mathbf{PrimREC}} % primitive recursive functions
\newcommand{\RESET}{\Sigma} % set of all recursively enumerable sets
\newcommand{\RECSET}{\Delta} % set of all recursive sets
\newcommand{\sse}{\iff}
\newcommand{\bforall}[2]{\forall{#1}\!<\!{#2}} % bounded universal quantification
\newcommand{\bexists}[2]{\exists{#1}\!<\!{#2}} % bounded existential quantification
\newcommand{\bmu}[2]{\mu{#1}\!<\!{#2}} % bounded mu-recursion
\newcommand{\fprim}[1]{\mathsf{p}(#1)} % function returning the n-th prime number
\newcommand{\pprim}[1]{\mathsf{prim}(#1)} % predicate testing primality of a number
\newcommand{\expn}[2]{\mathsf{exp}(#1,#2)} % function returning the exponent of #1 in the unique prime decomposition of #2

%\newcommand{\nat}{\mathbb{N}}
%\newcommand{\pair}[2]{\langle #1,#2 \rangle}
\newcommand{\fset}[1]{\sharp(#1)}
%\newcommand{\Pf}[1]{\mathcal{P}_{\mathrm{f}}(#1)}
%\newcommand{\st}{:}
% \newcommand{\seq}[1]{\vec{#1}}
\newcommand{\gramm}{\mathrel{::=}} % EBNF grammar definition
\newcommand{\ass}{\mathrel{:=}} % syntactical definition 
\newcommand{\nat}{\mathbb{N}} % set of natural numbers
\newcommand{\st}{:} % set constructor
% \newcommand{\ass}{:=} % assignment
\newcommand{\car}[1]{\mathsf{c}_{#1}} % characteristic function
\newcommand{\ran}[1]{\mathsf{ran}(#1)} % range of a function
\newcommand{\dom}[1]{\mathsf{dom}(#1)} % domain of a function
\newcommand{\secod}[1]{\prec\! #1 \!\succ} % code of a sequence
\newcommand{\seq}[1]{\prec\! #1 \!\succ} % code of a sequence
\newcommand{\sedecod}[2]{(#1)_{#2}} % extract the #2-th element of the sequence with code #1
\newcommand{\pair}[2]{\langle #1,#2 \rangle} % coding of pairs
\newcommand{\gph}[1]{\mathsf{gr}(#1)} % graph of a function

\maketitle
%\tableofcontents

%%%%%%%%%%%%%%%%%%%%%%%%%%%%%%%%%%%%%%%%%%%%
\section{Macchine di Turing}
%%%%%%%%%%%%%%%%%%%%%%%%%%%%%%%%%%%%%%%%%%%%

Le Macchine d Turing sono gli automi pi\`{u} potenti che tratteremo in tutto il corso. Tali automi possono essere considerati un'astrazione matematica dei nostri computer, proprio come le teorie fisiche sono un'astrazione del nostro mondo.

Come per le altre classi di automi, esistono le Macchine d Turing \emph{deterministiche} e quelle \emph{non-deterministiche}; noi vedremo solo quelle deterministiche e pertanto non menzioneremo mai pi\`{u} questa distinzione in seguito.

\begin{definition}[TM]\label{def:TM}
Una Macchina di Turing (TM, in breve) \`{e} una settupla $\cM = (Q,\Sigma,\Gamma,\delta,q_0,B,F)$ dove 
\begin{itemize}
\item $Q$ \`{e} l'insieme finito degli stati,
\item $\Sigma$ \`{e} l'alfabeto di input,
\item $\Gamma$ \`{e} l'alfabeto di nastro (ed abbiamo $\Sigma \subseteq \Gamma$),
\item $\delta: Q\times\Gamma \hookrightarrow Q\times\Gamma\times\{L,R\}$ \`{e} la funzione di transizione ed $\{L,R\}$ \`{e} l'insieme dei simboli di \emph{direzione},
\item $q_0 \in Q$ \`{e} lo stato iniziale,
\item $B \in \Gamma - \Sigma$ \`{e} il blank,
\item $F \subseteq Q$ \`{e} l'insieme degli stati finali.
\end{itemize}
Infine si richiede che $(F \times \Gamma) \cap \dom{\delta} = \emptyset$.
\end{definition}

La condizione $(F \times \Gamma) \cap \dom{\delta} = \emptyset$ vuol dire che non vi devono essere delle transizioni indicate da $\delta$ a partire dalle coppie $(q,X)$ dove $q$ \`{e} uno stato finale.

Si pu\`{o} notare una certa somiglianza con la definizione di PDA. Difatti una TM \`{e} in un certo senso una specie di automa a pila dove la pila \`{e} infinita e la si pu\`{o} scorrere avati e indietro. Si utilizzano le ultime lettere maiuscole dell'alfabeto inglese $X,Y,Z,\ldots$ per indicare generici elementi di $\Gamma$ mentre si utilizzano le prime lettere minuscole dell'alfabeto greco $\alpha,\beta,\gamma,\ldots$ per indicare generici elementi di $\Gamma^*$.

Il meccanismo di base di una Macchina di Turing \`{e} il seguente:
\begin{itemize}
\item vi \`{e} un nastro, diviso in celle di uguale dimensione, posto in posizione orizzontale che si estende all'infinito sia verso destra che verso sinistra;
\item vi \`{e} una testina che prima legge una cella del nastro poi, in base a ci\`{o} che c'\`{e} scritto, allo stato in cui si trova la macchina ed a ci\`{o} che dice la funzione $\delta$, scrive un simbolo nella cella, ed infine si sposta verso sinistra o destra sul nastro.
\end{itemize}

La prossima tappa \`{e} definire il linguaggio accettato da una TM. A tal fine dobbiamo precisare cosa significa dire che una stringa \`{e} accettata da una TM.

Al contrario di quanto fatto per gli automi finiti, ed analogamente al caso dei PDA, per le Macchine di Turing non si pu\`{o} definire una funzione di transizione estesa $\hat{\delta}: Q\times\Sigma^* \times \Gamma \to Q\times\Gamma\times\{L,R\}$. In luogo della funzione di transizione estesa si definisce una relazione binaria $\vdash$ su $\Gamma^* \times Q \times \Gamma^*$ che svolge un compito analogo a quello della funzione di transizione estesa nel caso degli automi finiti.

\begin{definition}\label{def:ID}
Una \emph{descrizione istantanea} (ID, in breve) \`{e} una tripla $(\ga,q,\gb) \in \Gamma^* \times Q \times \Gamma^*$.
\end{definition}

In una ID $(\ga,q,\gb)$ abbiamo che:
\begin{itemize}
\item $q$ indica lo stato attuale,
\item $\ga\gb$ \`{e} la sequenza di simboli contenuti nelle celle che appartengono alla porzione di nastro compresa tra il simbolo non-blank pi\`{u} a sinistra e il simbolo non-blank pi\`{u} a destra.
\item la testina si trova sopra la cella che contiene il primo simbolo della stringa $\gb$.
\end{itemize}

\begin{definition}[Relazione di transizione estesa]\label{def:trans-est3}
Data una TM $\cM = (Q,\Sigma,\Gamma,\delta,q_0,B,F)$ definiamo la sua \emph{relazione di transizione estesa} $\vdash$ su $\Gamma^* \times Q \times \Gamma^*$ come la pi\`{u} piccola che soddisfa le seguenti propriet\`{a}:
\begin{itemize}
\item se $\delta(q,X_i)=(p,Y,L)$, allora $(X_1\cdots X_{i-1},q,X_{i}X_{i+1}\cdots X_{n}) \vdash (X_1\cdots X_{i-2},p,X_{i-1}YX_{i+1}\cdots X_{n})$.\\ Si hanno due casi particolari:
\begin{itemize}
\item[(1)] se $i=1$, allora $(\epsilon,q,X_1\cdots X_{n}) \vdash (\epsilon,p,BYX_{2}\cdots X_{n})$
\item[(2)] se $i=n$ e $Y=B$, allora $(X_1\cdots X_{n-1},q,X_n) \vdash (X_{1}\cdots X_{n-2},p,X_{n-1})$
\end{itemize}
\item se $\delta(q,X_i)=(p,Y,R)$, allora $(X_1\cdots X_{i-1},q,X_{i}X_{i+1}\cdots X_{n}) \vdash (X_1\cdots X_{i-1}Y,p,X_{i+1}\cdots X_{n})$.\\ Si hanno due casi particolari:
\begin{itemize}
\item[(1)] se $i=1$ e $Y=B$, allora $(\epsilon,q,X_1\cdots X_n) \vdash (\epsilon,p,X_{2}\cdots X_{n})$
\item[(2)] se $i=n$, allora $(X_1\cdots X_{n-1},q,X_{n}) \vdash (X_1\cdots X_{n-1}Y,p,\epsilon)$
\end{itemize}
\end{itemize}
dove gli $X_j$ sono elementi di $\Gamma$ e $p,q \in Q$.
\end{definition}

La relazione $\vdash$ spiega completamente il funzionamento di una TM. Ad esempio $(\ga X,q,Y\gb) \vdash (\ga,p,XZ\gb)$ dice che se l'automa si trova nello stato $q$ e la testina legge il simbolo $Y$, allora la TM
\begin{itemize}
\item passa nello stato $q$,
\item sostituisce il simbolo $Y$ con il simbolo $Z$ nella cella che ha appena letto,
\item si sposta di una cella verso sinistra.
\end{itemize}
Chiaramente la sequenza di azioni descritta qui sopra si pu\`{o} verificare solo se $\delta(q,Y) = (p,Z,L)$.

Facciamo seguire un esempio di Macchina di Turing.

\begin{example}\label{ex:macchina-es}
Sia $\cM = (\{q_0,q_1,q_2,q_3,q_4\}, \{0,1\}, \{0,1,X,Y,B\}, \delta,q_0, B, \{q_4\})$ con la funzione di transizione definita come segue:

\begin{center}
\begin{tabular}{c | c c c c c |}
       & 0 & 1 & X & Y & B \\
\hline
$q_0$ & $(q_1,X,R)$ &  &  & $(q_3,Y,R)$ &  \\
$q_1$ & $(q_1,0,R)$ & $(q_2,Y,L)$ &  & $(q_1,Y,R)$ &  \\
$q_2$ & $(q_0,0,L)$ &  & $(q_0,X,R)$ & $(q_2,Y,L)$ &  \\
$q_3$ &  &  &  & $(q_3,Y,R)$ & $(q_4,B,R)$ \\
$q_4$ &  &  &  &  &  \\ 
\hline
\end{tabular}
\end{center}
In questo caso abbiamo che $\cL(\cM) = \{0^n1^n \st n \geq 1\}$.
\end{example}

%%%%%%%%%%%%%%%%%%%%%%%%%%%%%%%%%%%%%%%%%%%%
\subsection{Una rappresentazione grafica delle TM}
%%%%%%%%%%%%%%%%%%%%%%%%%%%%%%%%%%%%%%%%%%%%

Come nel caso dei PDA, la descrizione formale di una Macchina di Turing pu\`{o} essere complicata da cogliere immediatamente. Vediamo ora brevemente come si possono disegnare delle TM con una notazione che dia le stesse informazioni di quella formale, ma che sia pi\`{u} semplice da capire. Introdurremo questa notazione con un esempio, che speriamo dia l'idea di come applicare la notazione grafica ad ogni caso possibile.

\begin{example}\label{ex:macchina-es-grafica}
Ecco la rappresentazione grafica della macchina dell'esempio \ref{ex:macchina-es}.
\entrymodifiers={++[o][F-]}
$$
\xymatrix{
*\txt{ } \ar[r] & q_0 \ar[dd]_{Y/Y,R}\ar[r]^{0/Y,R} & q_1 \ar@(ur,ul)[]_{\txt{$Y/Y,R$ \\ $0/0,R$}} \ar[r]^{1/Y,L}   & q_2 \ar@(dr,ur)[]_{\txt{$Y/Y,L$ \\ $0/0,L$}} \ar@/^2pc/[ll]^{X/X,R} \\
*\txt{ }         & *\txt{ } &*\txt{ } &*\txt{ } \\
*\txt{ }         & q_3 \ar@(dr,dl)[]^{Y/Y,R} \ar[r]^{B/B,R}  & *++[o][F=]{q_4} \\
}
$$
\end{example}

%%%%%%%%%%%%%%%%%%%%%%%%%%%%%%%%%%%%%%%%%%%%
\subsection{Computazioni e terminazione}
%%%%%%%%%%%%%%%%%%%%%%%%%%%%%%%%%%%%%%%%%%%%

\begin{definition}\label{def:computation}
Sia $\cM = (Q,\Sigma,\Gamma,\delta,q_0,B,F)$ una TM. Una \emph{computazione} \`{e} sequenza
$$ \pi = (\ga_1,q_1,\gb_1),\ldots,(\ga_n,q_n,\gb_n),\ldots $$
di ID legate dalla relazione $\vdash$, cio\`{e} $(\ga_i,q_i,\gb_i) \vdash (\ga_{i+1},q_{i+1},\gb_{i+1})$ per ogni coppia di indici $i,i+1$ in $\pi$.
\end{definition}

Si noti che, siccome la TM \`{e} deterministica, ogni computazione \`{e} univocamente determinata dalla prima ID che vi compare.

In seguito noteremo $\pi_i = (\ga_i,q_i,\gb_i)$, cio\`{e} $\pi_i$ \`{e} l'$i$-esima ID di $\pi$. Si noti che vi possono essere delle computazioni infinite. Diciamo che una computazione $\pi$ 
\begin{itemize}
\item \emph{termina} se esiste una ID $(\ga_i,q_i,\gb_i)$ tale che $(\ga_i,q_i,\gb_i) \nvdash$, ovvero non esiste alcuna ID $(\ga_{i+1},q_{i+1},\gb_{i+1})$ tale che $(\ga_i,q_i,\gb_i) \vdash (\ga_{i+1},q_{i+1},\gb_{i+1})$.
\item \emph{non termina} altrimenti.
\end{itemize}

Si noti che il requisito $(F \times \Gamma) \cap \dom{\delta} = \emptyset$ implica che se $(\ga,q,\gb)$ \`{e} una ID tale che $q \in F$, allora $(\ga,q,\gb) \nvdash$. Pertanto tutte le computazioni che raggiungono uno stato finale terminano.

\begin{definition}[Decisore]\label{def:decisore}
Una TM $\cM = (Q,\Sigma,\Gamma,\delta,q_0,B,F)$ \`{e} un \emph{decisore} se per ogni  $w \in \Sigma^*$ la computazione che comincia in $(\epsilon,q_0,w)$ termina.
\end{definition}

La TM dell'esempio \ref{ex:macchina-es} \`{e} un decisore.

Indichiamo con $\vdash^*$ la chiusura riflessiva e transitiva della relazione $\vdash$, cio\`{e} la pi\`{u} piccola relazione riflessiva e transitiva contenente $\vdash$.

\begin{definition}[Accettazione]\label{def:lang-acc-TM}
Diciamo che una stringa $w \in \Sigma^*$ \`{e} \emph{accettata} da una TM $\cM$ sse esiste una configurazione $(\ga,p,\gb)$ tale che $p \in F$ e $(\epsilon,q_0,w) \vdash^* (\ga,p,\gb)$.
\end{definition}

In altre parole $w$ \`{e} accettata se la computazione eseguita da $\cM$ partire dalla ID iniziale $(\epsilon,q_0,w)$ termina raggiungendo uno stato finale. Si noti vi possono essere due diverse situazioni in cui $w$ NON \`{e} accettata:
\begin{enumerate}
\item la computazione eseguita da $\cM$ partire dalla ID iniziale $(\epsilon,q_0,w)$ termina raggiungendo uno stato NON finale;
\item la computazione eseguita da $\cM$ partire dalla ID iniziale $(\epsilon,q_0,w)$ NON termina affatto.
\end{enumerate} 

\begin{definition}[Linguaggio di una TM]\label{def:lang-acc-TM}
Sia $\cM = (Q,\Sigma,\Gamma,\delta,q_0,B,F)$ una TM. Il linguaggio \emph{accettato} da $\cM$, indicato con $\cL(\cM)$, \`{e} l'insieme delle stringhe accettate da $\cM$, ovvero $\cL(\cM) = \{w \in \Sigma^* \st \exists p \in F.\exists \ga,\gb \in \Gamma^*.\ (\epsilon,q_0,w) \vdash^* (\ga,p,\gb) \}$.
\end{definition}

Per cui tra le stringhe che non appartengono a $\cL(\cM)$ vi sono tutte quelle che inducono $\cM$ ad eseguire una computazione che non termina.

%%%%%%%%%%%%%%%%%%%%%%%%%%%%%%%%%%%%%%%%%%%%
\subsection{Linguaggi (semi-)decidibili}
%%%%%%%%%%%%%%%%%%%%%%%%%%%%%%%%%%%%%%%%%%%%

Andiamo ora a studiare la classe dei linguaggi accettati dalle TM.

\begin{definition}[Linguaggio semi-decidibile]\label{def:semi-dec-ling}
Un linguaggio $L$ \`{e} \emph{semi-decidibile} sse esiste una TM $\cM$ tale che $L = \cL(\cM)$.
\end{definition}

\begin{definition}[Linguaggio decidibile]\label{def:dec-ling}
Un linguaggio $L$ \`{e} \emph{decidibile} sse esiste un decisore $\cM$ tale che $L = \cL(\cM)$.
\end{definition}

Indichiamo con $\SDEC$ la classe dei linguaggi semi-decidibili ed indichiamo con $\DEC$ la classe dei linguaggi decidibili. Facciamo notare che nel libro di testo \cite{HMU} i linguaggi semi-decidibili vengono chiamati \emph{ricorsivamente enumerabili}, mentre in linguaggi decidibili vengono chiamati \emph{ricorsivi}. Si tratta di una pura questione di termini e la teminologia nell'area della calcolabilit\`{a} non \`{e} del tutto standard, ma ha una lunga storia (vedere \cite{Soare96}).

\begin{remark}\label{rem:semi-dec-dec}
Se un linguaggio \`{e} decidibile, allora \`{e} anche semi-decidibile.
\end{remark}

\begin{theorem}\label{thm:compl-dec}
Se un linguaggio \`{e} decidibile, allora anche il suo complementare \`{e} decidibile.
\end{theorem}

\begin{proof}
Sia $L$ un linguaggio decidibile. Allora esiste un decisore $\cM$ tale che $\cL(\cM) = L$. Costruiamo un decisore $\cM^c$ a partire da $\cM$ come segue:
\begin{itemize}
\item Gli stati di $\cM^c$ sono gli stessi di $\cM$, pi\`{u} un nuovo stato $r$;
\item Le transizioni di $\cM^c$ sono le stesse di $\cM$, pi\`{u} una nuova transizione $(q,X)\mapsto (r,X,R)$ per ogni coppia $(q,X)$ dove $q$ \`{e} un o stato non-finale in cui si pu\`{o} arrivare in $\cM$ e tale che non vi \`{e} una ransizione prevista per $(q,X)$ in $\cM$;
\item L'unico stato finale di $\cM^c$ \`{e} $r$.
\end{itemize}
Allora si vede subito che $\cM^c$ \`{e} un decisore tale che $\cL(\cM^c) = L^c$.
\qed\end{proof}

\begin{theorem}\label{thm:semi-compl-dec}
Se un linguaggio $L$ e il suo complementare sono entrambi semi-decidibili, allora $L$ \`{e} decidibile.
\end{theorem}

\begin{proof}
Siano $L$ ed $L^c$ entrambi semi-decidibili. Allora esistono due TM $\cM$ e $\cM'$ tali che $\cL(\cM) = L$ e $\cL(\cM') = L^c$. Sia $w$ una stringa qualsiasi. Siccome $w$ sta in almeno uno tra $L$ ed $L^c$, almeno una delle due macchine prendendo in input $w$ si arresta dopo una computazione finita raggiungendo uno stato finale. Possiamo costruire una terza TM $\cM''$ seguendo le seguenti indicazioni: data una stringa $w$ in input
\begin{itemize}
\item $\cM''$ simula la computazione in parallelo di $\cM$ e $\cM'$ con input $w$;
\item se la computazione in $\cM$ termina in uno stato finale, allora $\cM''$ termina in uno stato finale;
\item \item se la computazione in $\cM'$ termina in uno stato finale, allora $\cM''$ termina in uno stato non-finale.
\end{itemize}
\`{E} evidente che le due situazioni non possono presentarsi entrambe, ma almeno una di esse si presenter\`{a} di sicuro. Pertanto $\cM''$ \`{e} un decisore ed inotre $\cL(\cM'') = L$.
\qed\end{proof}

%%%%%%%%%%%%%%%%%%%%%%%%%%%%%%%%%%%%%%%%%%%%
\section{Robustezza del modello di Turing}
%%%%%%%%%%%%%%%%%%%%%%%%%%%%%%%%%%%%%%%%%%%%

Nelle lezioni passate abbiamo visto come i modelli di automi visti ammettesero modifiche che non modificavano la classe dei linguaggi ricoosciuti (per esempio determinismo/non-determinismo, $\varepsilon$-mosse, accettazione per pila vuota/stati finali). Anche nel caso delle Macchine di Turing discutiamo, per\`{o} in maniera breve ed informale, alcune modifiche ``senza conseguenze".

Il particolare modello di macchina di Turing introdotto nella Definizione \ref{def:TM} \`{e} in un certo senso irrilevante per il prosieguo del nostro corso, poich\'{e} noi ci occuperemo solo della potenza di calcolo e non dell'efficienza del modello (come invece fa la Teoria degli Automi e la Teoria della Complessit\`{a})). Citeremo i risultati ottenuti in letteratura; per alcune dimostrazioni e per ulteriori informazioni il lettore interessato pu\`{o} fare riferimento a \cite{HMU,Minsky67,Arbib69} per le loro prove e per ulteriori informazioni.

\textbf{Stati.} Anche se un solo stato non \`{e} in genere sufficienti per riconoscere ogni linguaggio semi-decidibile, due stati lo sono. Cos\`{i} non \`{e} rilevante se si limita il nostro modello di macchina con un numero fisso $n\geq 2$ di stati, si permette un qualsiasi numero di stati.

\textbf{Simboli.} Chiaramente dobbiamo avere almeno due simboli, dal momento che noi consideriamo il blank come un simbolo. Due simboli sono sufficienti per per riconoscere ogni linguaggio semi-decidibile, dal momento che \`{e} possibile economizzare sul numero di simboli, aumentando il numero di stati.

\textbf{Cancellazione.} Si pu\`{o} indifferentemente ammettere o meno che la testina possa cancellare simboli dal nastro\`{e}, nel senso che tutti i linguaggi semi-decidibili possono essere opportunamente riconosciuti da macchine che non cancellano mai. Il risultato dimostra che in linea di principio non abbiamo bisogno di materiale cancellabile, come nastri magnetici o dischi, per la memoria esterna del computer.

\textbf{Nastri e testine.} Qui la libert\`{a} \`{e} pressoch\'{e} assoluta. Lo sintetizziamo nel seguente risultato. Una Macchina di Turing con un numero finito di nastri, ognuno di dimensione infinita e con il suo numero finito di testine che lo scansionano contemporaneamente, pu\`{o} essere simulata da una Macchina di Turing con un solo nastro lineare, infinito in una sola direzione, e analizzato da una sola testina. Tuttavia, abbiamo bisogno delle due direzioni di movimento, dal momento che la limitazione ad una sarebbe compatibile solo con il comportamento finito o periodico sulle cellule al di fuori degli inputs.

\textbf{Determinismo.} Il nostro modello di una macchina di Turing \`{e} deterministico, nel senso che la funzione di transizione associa un solo risultato per ogni coppia stato-simbolo letto. Elementi di randomizzazione a nei dispositivi di calcolo sono stati introdotti da Shannon\textendash De Leeuw, Moore, Shannon\textendash Shapiro. Ci sono fondamentalmente due modelli. Le Macchine di Turing \emph{non-deterministiche} si comportano, in una situazione ambigua in cui pi\`{u} transizioni diverse sono applicabili, scegliendo casualmente una di loro: la loro potenza di calcolo, almeno per linguaggi sull'alfabeto $\{0,1\}$, non supera la potenza di quelle deterministiche. Le Macchine di Turing \emph{probabilistiche} differiscono da quelle non-deterministiche in quanto lo stato successivo ha una probabilit\`{a}, e quindi le istruzioni in conflitto non hanno la stessa probabilit\`{a} di essere scelte dalla macchina.

In conclusione, quando consideriamo una Macchina di Turing $\cM = (Q,\Sigma,\Gamma,\delta,q_0,B,F)$ \`{e} possibile assumere 


%%%%%%%%%%%%%%%%%%%%%%%%%%%%%%%%%%%%%%%%%%%%
\section{Modello di Turing e computers}
%%%%%%%%%%%%%%%%%%%%%%%%%%%%%%%%%%%%%%%%%%%%

Gran parte dell'interesse nella teoria della calcolabilit\`{a} ed in particolare nelle Mcchine di Turing \`{e} dovuto al fatto che esse sono considerate un modello matematico dei nostri moderni computers. Pertanto i risultati teorici che mostrano cosa le TM possono e non possono fare, ci fanno anche capire cosa i computer possono e non possono fare. Tutto ci\`{o} \`{e} legato alla \emph{fedelt\`{a}} del modello astratto alle caratteristiche dei reali calcolatori. Facciamo per\`{o} notare la differenza fondamentale tra i computer e le TM: ogni calcolatore sulla Terra ha una memoria \emph{finita} e pu\`{o} effettuare calcoli su numeri finiti compresi entro certi limiti, oltre i quali si ottengono degli errori di \emph{overflow}. Le TM invece, con il loro nastro infinito, non hanno questi problemi e possono infatti maneggiare qualsiasi numero naturale (oltre ogni limite imponibile) e simulare una memoria di qualsiasi dimensione. Considerato questo, non esiste alcun calcolatore sulla terra che possa fare tutto quello che fanno le TM, ma una cosa \`{e} sicuramente vera: i problemi che  dimostriamo irrisolvibili per tutte le TM sono irrisolvibili anche per i nostri calcolatori concreti! Per approfondire leggete la Sezione 8.6 di \cite{HMU} (escludendo la sezione 8.6.3).

%%%%%%%%%%%%%%%%%%%%%%%%%%%%%%%%%%%%%%%%%%%%
\bibliographystyle{abbrv}%splncs
\bibliography{bibliography}
%%%%%%%%%%%%%%%%%%%%%%%%%%%%%%%%%%%%%%%%%%%%
\end{document}

%%%%%%%%%%%%%%%%%%%%%%%%%%%%%%%%%%%%%%%%%%%%
%%%%%%%%%%%%%%%%%%%%%%%%%%%%%%%%%%%%%%%%%%%%
%%%%%%%%%%%%%%%%%%%%%%%%%%%%%%%%%%%%%%%%%%%%