\documentclass[runningheads,a4paper]{llncs}

\usepackage{amssymb}
\usepackage{amsmath}

\usepackage{mathrsfs}
\usepackage{stmaryrd}

\usepackage{enumitem}
% \usepackage{enumerate}

\usepackage{color}
\usepackage{graphicx}
\usepackage{rotating}
%\usepackage{xparse}
%\usepackage{latex8}
\usepackage{upgreek} 
\usepackage{cmll}
\usepackage{url}
\usepackage{xifthen}% provides \isempty test

\setcounter{tocdepth}{3}

\urldef{\mailsa}\path|{acarraro}@dsi.unive.it|
%\urldef{\mailsb}\path||
%\urldef{\mailsc}\path|
\newcommand{\keywords}[1]{\par\addvspace\baselineskip
\noindent\keywordname\enspace\ignorespaces#1}

\input prooftree.sty
\input xy
\xyoption{all}

\makeindex

\begin{document}

\mainmatter  % start of an individual contribution

% first the title is needed
\title{Note del corso di Calcolabilit\`{a} e Linguaggi Formali - Lezione 7}

% a short form should be given in case it is too long for the running head
\titlerunning{Note del corso di Calcolabilit\`{a} e Linguaggi Formali - Lezione 7}

% the name(s) of the author(s) follow(s) next
%
% NB: Chinese authors should write their first names(s) in front of
% their surnames. This ensures that the names appear correctly in
% the running heads and the author index.
%
\author{Alberto Carraro}
%
\authorrunning{A. Carraro}
% (feature abused for this document to repeat the title also on left hand pages)

% the affiliations are given next; don't give your e-mail address
% unless you accept that it will be published
\institute{DAIS, Universit\'{a} Ca' Foscari Venezia
%\mailsa\\
%\mailsb\\
%\mailsc\\
\url{http://www.dsi.unive.it/~acarraro}
}

%
% NB: a more complex sample for affiliations and the mapping to the
% corresponding authors can be found in the file "llncs.dem"
% (search for the string "\mainmatter" where a contribution starts).
% "llncs.dem" accompanies the document class "llncs.cls".
%

\toctitle{Note del corso di Calcolabilit\`{a} e Linguaggi Formali - Lezione 7}
\tocauthor{A. Carraro}

\newcommand{\scA}{\mathscr{A}}
\newcommand{\scB}{\mathscr{B}}
\newcommand{\scC}{\mathscr{C}}
\newcommand{\scD}{\mathscr{D}}
\newcommand{\scE}{\mathscr{E}}
\newcommand{\scF}{\mathscr{F}}
\newcommand{\scG}{\mathscr{G}}
\newcommand{\scH}{\mathscr{H}}
\newcommand{\scI}{\mathscr{I}}
\newcommand{\scJ}{\mathscr{J}}
\newcommand{\scK}{\mathscr{K}}
\newcommand{\scL}{\mathscr{L}}
\newcommand{\scM}{\mathscr{M}}
\newcommand{\scN}{\mathscr{N}}
\newcommand{\scO}{\mathscr{O}}
\newcommand{\scP}{\mathscr{P}}
\newcommand{\scQ}{\mathscr{Q}}
\newcommand{\scR}{\mathscr{R}}
\newcommand{\scS}{\mathscr{S}}
\newcommand{\scT}{\mathscr{T}}
\newcommand{\scU}{\mathscr{U}}
\newcommand{\scV}{\mathscr{V}}
\newcommand{\scW}{\mathscr{W}}
\newcommand{\scX}{\mathscr{X}}
\newcommand{\scY}{\mathscr{Y}}
\newcommand{\scZ}{\mathscr{Z}}

\newcommand{\fA}{\mathfrak{A}}
\newcommand{\fB}{\mathfrak{B}}
\newcommand{\fC}{\mathfrak{C}}
\newcommand{\fD}{\mathfrak{D}}
\newcommand{\fE}{\mathfrak{E}}
\newcommand{\fF}{\mathfrak{F}}
\newcommand{\fG}{\mathfrak{G}}
\newcommand{\fH}{\mathfrak{H}}
\newcommand{\fI}{\mathfrak{I}}
\newcommand{\fJ}{\mathfrak{J}}
\newcommand{\fK}{\mathfrak{K}}
\newcommand{\fL}{\mathfrak{L}}
\newcommand{\fM}{\mathfrak{M}}
\newcommand{\fN}{\mathfrak{N}}
\newcommand{\fO}{\mathfrak{O}}
\newcommand{\fP}{\mathfrak{P}}
\newcommand{\fQ}{\mathfrak{Q}}
\newcommand{\fR}{\mathfrak{R}}
\newcommand{\fS}{\mathfrak{S}}
\newcommand{\fT}{\mathfrak{T}}
\newcommand{\fU}{\mathfrak{U}}
\newcommand{\fV}{\mathfrak{V}}
\newcommand{\fW}{\mathfrak{W}}
\newcommand{\fX}{\mathfrak{X}}
\newcommand{\fY}{\mathfrak{Y}}
\newcommand{\fZ}{\mathfrak{Z}}

\newcommand\tA{{\mathsf{A}}}
\newcommand\tB{{\mathsf{B}}}
\newcommand\tC{{\mathsf{C}}}
\newcommand\tD{{\mathsf{D}}}
\newcommand\tE{{\mathsf{E}}}
\newcommand{\tF}{\mathsf{F}}
\newcommand\tG{{\mathsf{G}}}
\newcommand\tH{{\mathsf{H}}}
\newcommand\tI{{\mathsf{I}}}
\newcommand\tJ{{\mathsf{J}}}
\newcommand\tK{{\mathsf{K}}}
\newcommand\tL{{\mathsf{L}}}
\newcommand\tM{{\mathsf{M}}}
\newcommand\tN{{\mathsf{N}}}
\newcommand\tO{{\mathsf{O}}}
\newcommand\tP{{\mathsf{P}}}
\newcommand\tQ{{\mathsf{Q}}}
\newcommand\tR{{\mathsf{R}}}
\newcommand\tS{{\mathsf{S}}}
\newcommand\tT{{\mathsf{T}}}
\newcommand\tU{{\mathsf{U}}}
\newcommand\tV{{\mathsf{V}}}
\newcommand\tW{{\mathsf{W}}}
\newcommand\tX{{\mathsf{X}}}
\newcommand\tY{{\mathsf{Y}}}
\newcommand\tZ{{\mathsf{Z}}}

%Sums
\newcommand{\sM}{\mathbb{M}}
\newcommand{\sN}{\mathbb{N}}
\newcommand{\sL}{\mathbb{L}}
\newcommand{\sH}{\mathbb{H}}
\newcommand{\sP}{\mathbb{P}}
\newcommand{\sQ}{\mathbb{Q}}
\newcommand{\sR}{\mathbb{R}}
\newcommand{\sA}{\mathbb{A}}
\newcommand{\sB}{\mathbb{B}}
\newcommand{\sC}{\mathbb{C}}
\newcommand{\sD}{\mathbb{D}}

%overlined letters
\newcommand{\ova}{\bar{a}}
\newcommand{\ovb}{\bar{b}}
\newcommand{\ovc}{\bar{c}}
\newcommand{\ovd}{\bar{d}}
\newcommand{\ove}{\bar{e}}
\newcommand{\ovf}{\bar{f}}
\newcommand{\ovg}{\bar{g}}
\newcommand{\ovh}{\bar{h}}
\newcommand{\ovi}{\bar{i}}
\newcommand{\ovj}{\bar{j}}
\newcommand{\ovk}{\bar{k}}
\newcommand{\ovl}{\bar{l}}
\newcommand{\ovm}{\bar{m}}
\newcommand{\ovn}{\bar{n}}
\newcommand{\ovo}{\bar{o}}
\newcommand{\ovp}{\bar{p}}
\newcommand{\ovq}{\bar{q}}
\newcommand{\ovr}{\bar{r}}
\newcommand{\ovs}{\bar{s}}
\newcommand{\ovt}{\bar{t}}
\newcommand{\ovu}{\bar{u}}
\newcommand{\ovv}{\bar{v}}
\newcommand{\ovw}{\bar{w}}
\newcommand{\ovx}{\bar{x}}
\newcommand{\ovy}{\bar{y}}
\newcommand{\ovz}{\bar{z}}

%overlined capital letters
\newcommand{\ovA}{\overline{A}}
\newcommand{\ovB}{\overline{B}}
\newcommand{\ovC}{\overline{C}}
\newcommand{\ovD}{\overline{D}}
\newcommand{\ovE}{\overline{E}}
\newcommand{\ovF}{\overline{F}}
\newcommand{\ovG}{\overline{G}}
\newcommand{\ovH}{\overline{H}}
\newcommand{\ovI}{\overline{I}}
\newcommand{\ovJ}{\overline{J}}
\newcommand{\ovK}{\overline{K}}
\newcommand{\ovL}{\overline{L}}
\newcommand{\ovM}{\overline{M}}
\newcommand{\ovN}{\overline{N}}
\newcommand{\ovO}{\overline{O}}
\newcommand{\ovP}{\overline{P}}
\newcommand{\ovQ}{\overline{Q}}
\newcommand{\ovR}{\overline{R}}
\newcommand{\ovS}{\overline{S}}
\newcommand{\ovT}{\overline{T}}
\newcommand{\ovU}{\overline{U}}
\newcommand{\ovV}{\overline{V}}
\newcommand{\ovW}{\overline{W}}
\newcommand{\ovX}{\overline{X}}
\newcommand{\ovY}{\overline{Y}}
\newcommand{\ovZ}{\overline{Z}}

%vec capital letters
\newcommand{\veA}{\vec{A}}
\newcommand{\veB}{\vec{B}}
\newcommand{\veC}{\vec{C}}
\newcommand{\veD}{\vec{D}}
\newcommand{\veE}{\vec{E}}
\newcommand{\veF}{\vec{F}}
\newcommand{\veG}{\vec{G}}
\newcommand{\veH}{\vec{H}}
\newcommand{\veI}{\vec{I}}
\newcommand{\veJ}{\vec{J}}
\newcommand{\veK}{\vec{K}}
\newcommand{\veL}{\vec{L}}
\newcommand{\veM}{\vec{M}}
\newcommand{\veN}{\vec{N}}
\newcommand{\veO}{\vec{O}}
\newcommand{\veP}{\vec{P}}
\newcommand{\veQ}{\vec{Q}}
\newcommand{\veR}{\vec{R}}
\newcommand{\veS}{\vec{S}}
\newcommand{\veT}{\vec{T}}
\newcommand{\veU}{\vec{U}}
\newcommand{\veV}{\vec{V}}
\newcommand{\veW}{\vec{W}}
\newcommand{\veX}{\vec{X}}
\newcommand{\veY}{\vec{Y}}
\newcommand{\veZ}{\vec{Z}}

%bold capital letters
\newcommand{\bA}{\mathbf{A}}
\newcommand{\bB}{\mathbf{B}}
\newcommand{\bC}{\mathbf{C}}
\newcommand{\bD}{\mathbf{D}}
\newcommand{\bE}{\mathbf{E}}
\newcommand{\bF}{\mathbf{F}}
\newcommand{\bG}{\mathbf{G}}
\newcommand{\bH}{\mathbf{H}}
\newcommand{\bI}{\mathbf{I}}
\newcommand{\bJ}{\mathbf{J}}
\newcommand{\bK}{\mathbf{K}}
\newcommand{\bL}{\mathbf{L}}
\newcommand{\bM}{\mathbf{M}}
\newcommand{\bN}{\mathbf{N}}
\newcommand{\bO}{\mathbf{O}}
\newcommand{\bP}{\mathbf{P}}
\newcommand{\bQ}{\mathbf{Q}}
\newcommand{\bR}{\mathbf{R}}
\newcommand{\bS}{\mathbf{S}}
\newcommand{\bT}{\mathbf{T}}
\newcommand{\bU}{\mathbf{U}}
\newcommand{\bV}{\mathbf{V}}
\newcommand{\bW}{\mathbf{W}}
\newcommand{\bX}{\mathbf{X}}
\newcommand{\bY}{\mathbf{Y}}
\newcommand{\bZ}{\mathbf{Z}}

\newcommand{\mbbA}{\mathbb{A}}
\newcommand{\mbbB}{\mathbb{B}}
\newcommand{\mbbC}{\mathbb{C}}
\newcommand{\mbbD}{\mathbb{D}}
\newcommand{\mbbE}{\mathbb{E}}
\newcommand{\mbbF}{\mathbb{F}}
\newcommand{\mbbG}{\mathbb{G}}
\newcommand{\mbbH}{\mathbb{H}}
\newcommand{\mbbI}{\mathbb{I}}
\newcommand{\mbbL}{\mathbb{L}}
\newcommand{\mbbM}{\mathbb{M}}
\newcommand{\mbbN}{\mathbb{N}}
\newcommand{\mbbW}{\mathbb{W}}
\newcommand{\mbbY}{\mathbb{Y}}
\newcommand{\mbbX}{\mathbb{X}}
\newcommand{\mbbZ}{\mathbb{Z}}

%lower case greek letters
\newcommand{\ga}{\alpha}
\newcommand{\gb}{\beta}
\newcommand{\gc}{\gamma}
\newcommand{\gd}{\delta}
\newcommand{\gep}{\varepsilon}
\newcommand{\gz}{\zeta}
\newcommand{\geta}{\eta}
\newcommand{\gth}{\theta}
\newcommand{\gi}{\iota}
\newcommand{\gv}{\nu}
\newcommand{\gk}{\kappa}
\newcommand{\gl}{\lambda}
\newcommand{\gm}{\mu}
\newcommand{\gn}{\nu}
\newcommand{\gx}{\xi}
\newcommand{\gp}{\pi}
\newcommand{\gr}{\rho}
\newcommand{\gs}{\sigma}
\newcommand{\gt}{\ensuremath{\tau}}
\newcommand{\gu}{\upsilon}
% \newcommand{\gph}{\varphi}
\newcommand{\gch}{\chi}
\newcommand{\gps}{\psi}
\newcommand{\go}{\omega}
\newcommand{\gto}{\ensuremath{\bar\tau}}

%bold lower case greek letters
%\newcommand\ssn{\mbox{\boldmath $\eta$}}
\newcommand{\bga}{\mbox{\boldmath $\alpha$}}
\newcommand{\bgb}{\mbox{\boldmath $\beta$}}
\newcommand{\bgc}{\mbox{\boldmath $\gamma$}}
\newcommand{\bgp}{\mbox{\boldmath $\pi$}}
\newcommand{\bgd}{\mbox{\boldmath $\delta$}}
\newcommand{\bge}{\mbox{\boldmath $\epsilon$}}
\newcommand{\bgs}{\mbox{\boldmath $\sigma$}}
\newcommand{\bgt}{\mbox{\boldmath $\tau$}}
\newcommand{\bgr}{\mbox{\boldmath $\rho$}}
\newcommand{\bgch}{\mbox{\boldmath $\chi$}}
\newcommand{\bgo}{\mbox{\boldmath $\omega$}}

%upper case greek letters
\newcommand{\gG}{\Gamma}
\newcommand{\gF}{\Phi}
\newcommand{\gD}{\Delta}
\newcommand{\gT}{\Theta}
\newcommand{\gP}{\Pi}
\newcommand{\gX}{\Xi}
\newcommand{\gS}{\Sigma}
\newcommand{\gO}{\Omega}
\newcommand{\gL}{\Lambda}

\newcommand\rA{{\mathrm{A}}}
\newcommand\rB{{\mathrm{B}}}
\newcommand\rC{{\mathrm{C}}}
\newcommand\rD{{\mathrm{D}}}
\newcommand\rE{{\mathrm{E}}}
\newcommand{\rF}{\mathrm{F}}
\newcommand\rG{{\mathrm{G}}}
\newcommand\rH{{\mathrm{H}}}
\newcommand\rI{{\mathrm{I}}}
\newcommand\rL{{\mathrm{L}}}
 
%\newcommand\ra{{\mathrm{a}}}
\newcommand\rb{{\mathrm{b}}}
\newcommand\rc{{\mathrm{c}}}
\newcommand\rd{{\mathrm{d}}}
\newcommand\re{{\mathrm{e}}}
\newcommand{\rf}{\mathrm{f}}
\newcommand\rg{{\mathrm{g}}}
\newcommand\rh{{\mathrm{h}}}
\newcommand\ri{{\mathrm{i}}}
\newcommand\rl{{\mathrm{l}}}
\newcommand\mrm{{\mathrm{m}}}
\newcommand\rn{{\mathrm{n}}}
\newcommand\ro{{\mathrm{o}}}
\newcommand\rp{{\mathrm{p}}}
%\newcommand\rq{{\mathrm{q}}}
\newcommand\rr{{\mathrm{r}}}
\newcommand\rs{{\mathrm{s}}}
\newcommand\rt{{\mathrm{t}}}

\newcommand{\cA}{\mathcal{A}}
\newcommand{\cB}{\mathcal{B}}
\newcommand{\cC}{\mathcal{C}}
\newcommand{\cD}{\mathcal{D}}
\newcommand{\cE}{\mathcal{E}}
\newcommand{\cF}{\mathcal{F}}
\newcommand{\cG}{\mathcal{G}}
\newcommand{\cH}{\mathcal{H}}
\newcommand{\cI}{\mathcal{I}}
\newcommand{\cJ}{\mathcal{J}}
\newcommand{\cK}{\mathcal{K}}
\newcommand{\cL}{\mathcal{L}}
\newcommand{\cM}{\mathcal{M}}
\newcommand{\cN}{\mathcal{N}}
\newcommand{\cO}{\mathcal{O}}
\newcommand{\cP}{\mathcal{P}}
\newcommand{\cQ}{\mathcal{Q}}
\newcommand{\cR}{\mathcal{R}}
\newcommand{\cS}{\mathcal{S}}
\newcommand{\cT}{\mathcal{T}}
\newcommand{\cU}{\mathcal{U}}
\newcommand{\cV}{\mathcal{V}}
\newcommand{\cW}{\mathcal{W}}
\newcommand{\cX}{\mathcal{X}}
\newcommand{\cY}{\mathcal{Y}}
\newcommand{\cZ}{\mathcal{Z}}

\newenvironment{myitem}%
{\begin{list}%
       {-}%
       {\setlength{\itemsep}{0pt}
     \setlength{\parsep}{3pt}
     \setlength{\topsep}{3pt}
     \setlength{\partopsep}{0pt}
     \setlength{\leftmargin}{0.7em}
     \setlength{\labelwidth}{1em}
     \setlength{\labelsep}{0.3em}}}%
{\end{list}}

\newenvironment{myitemize}%
{\begin{list}%
       {-}%
       {\setlength{\itemsep}{0pt}
     \setlength{\parsep}{2pt}
     \setlength{\topsep}{2pt}
     \setlength{\partopsep}{0pt}
     \setlength{\leftmargin}{2em}
     \setlength{\labelwidth}{1em}
     \setlength{\labelsep}{0.3em}}}%
{\end{list}}

%Alberto's macros
\newcommand{\ls}[2]{\langle #2 / #1\rangle} % linear substitution
\newcommand{\cs}[2]{\{ #2 / #1\}} % classical substitution
%\newcommand{\ls}[2]{\langle #1:=#2\rangle} % linear substitution
%\newcommand{\cs}[2]{\{ #1:=#2\}} % classical substitution
\newcommand{\Bag}[1]{[#1]} % bag formation
\renewcommand{\smallsetminus}{-}
%Giulio's macros
%Sets:
%\newcommand{\nat}{\mathcal{N}}
\newcommand{\mbz}{\mathbf{0}}
\newcommand{\mbo}{\mathbf{1}}
\newcommand{\mbt}{\mathbf{2}}
\newcommand{\rea}[1]{\mathsf{rea}(#1)} % set of realizers of #1
\newcommand{\realize}{\Vdash} % realizability relation
\newcommand{\natp}{\nat^+}
\newcommand{\one}{\mathbf{1}}
\newcommand{\bool}{\mathbf{2}}
\newcommand{\perm}[1]{\fS_{#1}}
\newcommand{\card}[1]{\# #1} % cardinality of a set
%Boh
\newcommand{\Omegatuple}[1]{\Mfin{#1}^{(\omega)}}
\newcommand{\Pow}[1]{\cP(#1)}
\newcommand{\Powf}[1]{\cP_{\mathrm{f}}(#1)}
\newcommand{\Id}[1]{\mathrm{Id}_{#1}}
\newcommand{\comp}{\circ}
\newcommand{\With}[2]{{#1}\with{#2}}
\newcommand{\Termobj}{1}
\newcommand{\App}{\mathrm{Ap}}
\newcommand{\Abs}{\uplambda}
\newcommand{\Funint}[2]{[{#1}\!\!\imp\!\!{#2}]}

%Lambda calculus:
%\newcommand{\full}{\gto{\bang}}
%\newcommand{\dlam}{\ensuremath{\partial\lambda}}
%\newcommand{\dzlam}{\ensuremath{\partial_0\lambda}}
%\newcommand{\lam}{\ensuremath{\lambda}}
%\newcommand{\bang}{\oc}
%\newcommand{\hole}[1]{\llparenthesis #1\rrparenthesis}
\newcommand{\paral}{\vert}
\newcommand{\FSet}[1]{\Lambda^{#1}_{\bang}}
\newcommand{\supp}[1]{\mathsf{su}(#1)} % support of multises

%\newcommand{\tContSet}{\Set{\gt}\hole{\cdot}} % bang-free test contexts
%\newcommand{\tFContSet}{\FSet{\gt}\hole{\cdot}} % all test contexts

%\newcommand{\ContSet}{\Set{\gt}\hole{\cdot}} % bang-free term contexts
%\newcommand{\FContSet}{\FSet{\gt}\hole{\cdot}} % all term contexts

\newcommand{\sums}[1]{\bool\langle\Set{#1}\rangle}
\newcommand{\Fsums}[1]{\bool\langle\FSet{#1}\rangle}
\newcommand{\la}{\leftarrow}
\newcommand{\ot}{\leftarrow}
\newcommand{\labelot}[1]{\ _{#1}\!\leftarrow} % left arrow with label
\newcommand{\labelto}[1]{\rightarrow_{#1}} % right arrow with label
\newcommand{\mslabelot}[1]{\ _{#1}\!\twoheadleftarrow} % left two head arrow with label
\newcommand{\mslabelto}[1]{\twoheadrightarrow_{#1}} % right two head arrow with label
\newcommand{\msla}{\twoheadleftarrow} 
\newcommand{\msto}{\twoheadrightarrow}
\newcommand{\toh}{\to_{h}} % head reduction
\newcommand{\mstoh}{\msto_{h}} % transitive head reduction
\newcommand{\etoh}{\to_{h\eta}} % extensional head reduction
\newcommand{\msetoh}{\msto_{h\eta}} % extensional transitive head reduction
\newcommand{\too}{\to_{o}} % outer-reduction
\newcommand{\mstoo}{\msto_{o}} % transitive outer-reduction
\newcommand{\etoo}{\to_{o\eta}} % extensional outer-reduction
\newcommand{\msetoo}{\msto_{o\eta}} % extensional transitive outer-reduction
\newcommand{\eqt}{=_{\theta}} % weakly extensional conversion
\newcommand{\eqte}{=_{\theta\eta}} % extensional conversion
\newcommand{\eq}{=} % basic conversion

\newcommand{\dg}[2]{\mathrm{deg}_{#1}(#2)} % degree of a variable #1 in a term #2

\newcommand{\obsle}{\sqsubseteq_{\mathcal{O}}} % observational preorder
\newcommand{\obseq}{\approx_{\mathcal{O}}} % observational equivalence

\newcommand{\tesle}{\sqsubseteq_{\mathcal{C}}} % convergence preorder
\newcommand{\teseq}{\approx_{\mathcal{C}}} % convergence equivalence

\newcommand{\Fobsle}{\sqsubseteq^{\bang}_{\mathcal{O}}} % full observational preorder
\newcommand{\Fobseq}{\approx^{\bang}_{\mathcal{O}}} % full observational equivalence

\newcommand{\Ftesle}{\sqsubseteq^{\bang}_{\mathcal{C}}} % full convergence preorder
\newcommand{\Fteseq}{\approx^{\bang}_{\mathcal{C}}} % full convergence equivalence

%Semantics:
\newcommand{\rank}[1]{\mathsf{rk}(#1)} % rank of something
\newcommand{\rrank}[1]{\mathsf{rrk}(#1)} % right rank of an implicative formula
\newcommand{\lrank}[1]{\mathsf{lrk}(#1)} % left rank of an implicative formula
%\newcommand{\termin}[1]{\mathsf{t}(#1)} % set of terminals of a set of formulas
\newcommand{\termin}[3]{\mathsf{tmn}_{#1}^{#2}(#3)} % set of terminals of a set of formulas. The first argument is a tuple of terms to be substituted for the tuple of variables given in the second argument. The third argument is the formula of which we take the terminals 
\newcommand{\conc}[1]{\mathsf{cn}(#1)} % set of premisses of a set of formulas
\newcommand{\prem}[3]{\mathsf{pr}_{#1}^{#2}(#3)} % set of premisses of a set of formulas
\newcommand{\premp}[1]{\mathsf{pp}(#1)} % special premisses of premisses of a set of formulas
\newcommand{\premterm}[3]{\mathsf{prt}_{#1}^{#2}(#3)} % set of premisses having terminals in common with set of formulas #1
\newcommand{\spnex}[1]{\overline{#1}} % semi-prenex form of the formula #1
\newcommand{\ospnex}[1]{\overline{\overline{#1}}} % semi-prenex form of the formula #1 deprived of all universal quantifiers at the front
\newcommand{\forant}{\mathsf{uqa}} % one step semi-prenex form of the formula #1
\newcommand{\wrap}[1]{\bar{#1}} % wrapping of a term
\newcommand{\len}{\ell}
\newcommand{\trm}[1]{#1^{\textrm{--}}}
\newcommand{\cont}[2]{#1^{+}\hole{#2}}
\newcommand{\Mfin}[1]{\mathcal{M}_{\mathrm{f}}(#1)}
\newcommand{\mcup}{\uplus}
\newcommand{\mmcup}{\bar{\mcup}}
% \newcommand{\Pair}[2]{\langle{#1},{#2}\rangle}
\newcommand{\Rel}{\mathbf{REL}} %category of sets and relations
\newcommand{\MRel}{\mathbf{REL}_{\bang}} %Kleisli category of sets and relations
\newcommand{\Inf}{\mathbf{Inf}} %category of information system and approx rels
\newcommand{\SD}{\mathbf{SD}} %category of Scott domains and continuous functions
\newcommand{\CPO}{\mathbf{CPO}} %category of CPOs and continuous functions
\newcommand{\SL}{\mathbf{ScottL}} %category of preorders
\newcommand{\SLb}{\mathbf{ScottL}_{\bang}} %Kleisli category of \SL
\newcommand{\Coh}{\mathbf{Coh}} %category of coherent spaces
\newcommand{\Cohb}{\mathbf{Coh}_{\bang}} %Kleisli category of \Coh


\newcommand{\otspam}{
\mathrel{\vcenter{\offinterlineskip
\vskip-.130ex\hbox{\begin{turn}{180}$\mapsto$\end{turn}}}}} % reversed mapsto

\newcommand{\envup}[3]{#1[#2 \otspam #3]} % environment update

\newcommand{\try}[2]{\mathsf{try}_{#1}\{#2\}} % execute the second argument first argument until the second one is found
\newcommand{\catch}[2]{\mathsf{catch}_{#1}\{#2\}} % when the first argument is found, execute the second one

\newcommand{\Lamex}{\Lambda_{\mathsf{ex}}} % lambda calculus with try and catch

\renewcommand{\iff}{\Leftrightarrow}
\newcommand{\imp}{\Rightarrow}
\newcommand{\Apex}[1]{^{\: #1}}

\newcommand{\compl}[1]{{#1}^c} % complement of a set
\newcommand{\pts}{.\,.\,} % points abbreviated
%\newcommand{\conv}[1]{{#1}\!\downarrow} % covergence
\newcommand{\convh}[1]{{#1}\!\downarrow_h} % head covergence
\newcommand{\solv}[1]{#1\lightning} % solvance
\newcommand{\solvo}[1]{#1\lightning_o} % outer solvance
\newcommand{\module}[1]{\bool\langle #1 \rangle}

\newcommand{\Ide}[1]{Ide(#1)} % set of all ideals of a preorder

\newcommand{\Bstk}{\bB_{\mathsf{s}}} % quasi-boolean algebra of saturated sets of stacks
\newcommand{\fsubseteq}{\subseteq_\mathrm{f}} % finite subset
\newcommand{\Ps}[1]{\cP(#1)} % powerset
\newcommand{\Pss}[1]{\cP_\mathrm{s}(#1)} % set of all saturated subsets
\newcommand{\Psc}[1]{\cP_\mathrm{c}(#1)} % set of all closed subsets
\newcommand{\Psg}[1]{\cP_\mathrm{g}(#1)} % set of all good subsets
\newcommand{\Psf}[1]{\cP_\mathrm{f}(#1)} % set of all finite subsets
\newcommand{\Ms}[1]{\cM(#1)} % set of all multisets
\newcommand{\Msf}[1]{\cM_\mathrm{f}(#1)} % set of all finite multisets
\newcommand{\fst}{\mathsf{fst}} % reduction proper to the \Lambda\mu-calculus
\newcommand{\cons}{::} % stack constructor
\newcommand{\at}{\!\centerdot} % stack constructor (cons)
\newcommand{\ats}{\at\ldots\at} % stack constructor (cons) with lower suspension dots 
%\newcommand{\at}{\!::\!} % stack constructor

%\newcommand{\meet}{\} % inf operator
%\newcommand{\join}{\!\centerdot} % inf operator

\newcommand{\sps}[3]{\bgp^{(#1,#2,#3)}} % special stack defined as \overbrace{\cadr{#1}{0}\at\ldots\at\cadr{#1}{0}}^{#3 \mbox{ times}}\at #1
\newcommand{\spt}[1]{\bA^{(#1)}} % special term defined as \bd\epsilon.\cadr{\gd}{0}\ap(\cadr{\epsilon}{0}\at\ldots\at\cadr{\epsilon}{q-1}}\at\cddr{\epsilon}{q})

% \newcommand{\cdr}[1]{\mathsf{cdr}(#1)} % tail of stack
% \newcommand{\car}[1]{\mathsf{car}(#1)} % head of stack
% \newcommand{\itcdr}[2]{#1[#2)} % iterated tail of stack
% \newcommand{\cddr}[2]{#1[#2)} % iterated tail of stack
% \newcommand{\cadr}[2]{#1[#2]} % head of an iterated tail of stack

\newcommand{\op}{\mathsf{op}} % generic binary infix operator
\newcommand{\fun}[1]{\mathsf{f}(#1)} % generic unary function symbol
\newcommand{\nil}{\mathsf{nil}} % empty stack
\newcommand{\mcddr}[2]{\mathsf{cdr}^{#1}(#2)} % modified iterated tail of stack
\newcommand{\mitcar}[2]{\mathsf{car}^{#1}(#2)} % modified iterated head of stack
\newcommand{\mitcdr}[2]{\mathsf{cdr}^{#1}(#2)} % modified iterated tail of stack

\newcommand{\callcc}{\mathsf{cc}} % Felleisen's call/cc
\newcommand{\kpi}[1]{\mathsf{k}_{#1}} % Krivine's term that restores the stack
\newcommand{\nf}[1]{\mathsf{Nf}(#1)} % partial function returning the normal form 
\newcommand{\onf}[1]{\mathsf{Onf}(#1)} % partial function returning the outer normal form
\newcommand{\eonf}[1]{\eta\mathsf{Onf}(#1)} % partial function returning the extensional outer normal form
\newcommand{\hnf}[1]{\mathsf{Hnf}(#1)} % partial function returning the beta-head normal form
\newcommand{\ehnf}[1]{\eta\mathsf{Hnf}(#1)} % partial function returning the beta-eta head normal form
\newcommand{\Sol}{\mathsf{Sol}^{\mathsf{t}}} % set of all solvable terms
%\newcommand{\USol}{\mathsf{Sol}^{\mathsf{t}} % set of all solvable terms
\newcommand{\SetBT}{\mathfrak{B}} % set of all Bohm trees
\newcommand{\SetBTt}{\mathfrak{B}^{\mathsf{t}}} % set of all Bohm trees of \stk-terms
\newcommand{\BT}[1]{\mathsf{BT}(#1)} % Bohm tree of an expression
\newcommand{\tBT}[2]{\mathsf{BT}_{#2}(#1)} % truncated Bohm tree of an expression
\newcommand{\eBT}[1]{\eta\mathsf{BT}(#1)} % extensional Bohm tree of an expression
\newcommand{\teBT}[2]{\eta\mathsf{BT}_{#2}(#1)} % truncated extensional Bohm tree of an expression
\newcommand{\bdom}[2]{\mathsf{dom}(#1,#2)} % bounded domain of a term seen as a function over sequences of natural numbers
%\newcommand{\virt}[2]{\langle #1 \mid #2 \rangle} % virtual extension of the map corresponding to a term 
\newcommand{\bout}[3]{\mbox{\boldmath{$\langle$}} #1 \!\mid\! #2 \!\mid\! #3 \mbox{\boldmath{$\rangle$}}} % Bohm out term corresponding to a term #1, the sequence #2 , the bound #3 and the width #4
\newcommand{\vbout}[3]{\mbox{\boldmath{$\langle$}} #1 \!\mid\! #2 \!\mid\! #3 \mbox{\boldmath{$\rangle$}}} % virtual Bohm out term corresponding to a term #1, the sequence #2 and the bound #3 
\newcommand{\virt}[1]{\mathsf{vir}(#1)} % set of sequences that belong virtually to the map corresponding to a term
\newcommand{\bvirt}[2]{\mathsf{vir}(#1,#2)} % set of sequences that belong virtually to the map corresponding to a term, with a bound on their length
\newcommand{\extr}[1]{\mathsf{extr}(#1)} % extensionally reachable sequences
\newcommand{\uns}[1]{\mathsf{uns}(#1)} % unsolvable sequences
\newcommand{\unr}[1]{\mathsf{unr}(#1)} % unreachable sequences
\newcommand{\eqty}{\stackrel{\infty}{=}} % equality of Bohm trees up to infinite eta-expansion
\newcommand{\simty}{\stackrel{\infty}{\sim}} % similarity at all sequences of natural numbers
\newcommand{\pexp}[2]{\mbox{\boldmath{$\langle$}} #1 \lVert #2 \mbox{\boldmath{$\rangle$}}} % path expansion of a term
\newcommand{\Seq}{Seq} % the set of finite sequences of strictly positive natural numbers
\newcommand{\tSeq}[1]{Seq_{\leq #1}} % the set of finite sequences of length less or equal to a specified bound
\newcommand{\Lab}{Lab} % the set of labels of Bohm trees

\newcommand{\NT}[1]{\mathsf{NT}(#1)} % Nakajima tree of an expression
\newcommand{\tNT}[2]{\mathsf{NT}_{#2}(#1)} % truncated Nakajima tree of an expression

\newcommand{\sub}[2]{\{#1/#2\}} % classical substitution of #1 for #2
\newcommand{\ab}[1]{\mathcal{A}(#1)} % abort of a term
\newcommand{\ctrl}[1]{\mathcal{C}(#1)} % control of a term
\newcommand{\cmd}[2]{\langle #1 \lVert #2\rangle} % command constructor for lambda mu-mu-tilde
\newcommand{\ap}{\star} % application symbol of a term to a process
\newcommand{\bd}{\kappa} % binder for stack variables
\newcommand{\lambdab}{\bar{\lambda}} % lambda bar of mu-mu tilde calculus
\newcommand{\mut}{\tilde{\mu}} % binder for mu tilde calculus
\newcommand{\tcbn}[1]{#1^{\circ}} % translation of the cbn lambdamumu expressions into stack expressions 
\newcommand{\tcbv}[1]{#1^{\bullet}} % translation of the cbv lambdamumu expressions into stack expressions 
\newcommand{\texp}[1]{#1^{\circ}} % translation of the lambdamu expressions into stack expressions
\newcommand{\ttyp}[1]{#1^{\circ}} % translation of the lambdamu types into stack types
\newcommand{\Tp}[1]{#1^{\circ}} % translation
\newcommand{\Te}[1]{#1^{\circ}} % translation of the lambdamu expressions into stack expressions
\newcommand{\Neg}[1]{#1^{-}} % negative translation of formulas
\newcommand{\Pos}[1]{#1^{+}} % positive translation of formulas
\newcommand{\Tt}[1]{#1^{-}} % translation of stack expressions into lambda calculus with pairing
\newcommand{\Ts}[1]{#1^{+}} % translation of the lambda mu calculus into the lambda calculus
\newcommand{\dev}[1]{#1^{\baro}} % inner-outer development of the redexes of #1
\newcommand{\AtForm}{\mathrm{AtFm}} % set of atomic formulas of first-order logic
\newcommand{\UqAtForm}{\mathrm{UqAtFm}} % the set of universally quantified atomic formulas of first-order logic
\newcommand{\UqBot}{\mathrm{UqBot}} % the set of universally quantified atomic formulas of first-order logic in which the atomic formula is $\bot$
\newcommand{\Form}{\mathrm{Fm}} % set of formulas of second-order logic
\newcommand{\cForm}{\mathrm{Fm}^\mathsf{o}} % set of closed formulas of second-order logic
\newcommand{\Val}[1]{\mathrm{Val}_{#1}} % set of valuations into the structure #1
\newcommand{\At}{\mathrm{At}} % atomic formulas
\newcommand{\cAt}{\mathrm{At}^\mathsf{o}} % closed atomic formulas
\newcommand{\Var}{\mathrm{Var}} % set of variables
\newcommand{\Nam}{\mathrm{Nam}} % set of names
\newcommand{\FV}{\mathrm{FV}} % free variables
\newcommand{\FN}{\mathrm{FN}} % free names
\newcommand{\LTer}[1]{\Lambda^{\mathsf{#1}}} % set of terms of the lambda-mu calculus
\newcommand{\LTyp}[1]{\cT_{\lambda\mu}^{\mathsf{#1}}} % set of types of the lambda-mu calculus
\newcommand{\ITer}[1]{\Sigma_{\mathsf{in}}^{\mathsf{#1}}} % set of intuitionistic terms of the stack calculus
\newcommand{\BTer}[1]{\Sigma_{\mathsf{b}}^{\mathsf{#1}}} % finite Bohm trees of the stack calculus
\newcommand{\KTer}[1]{\Sigma^{\mathsf{#1}}} % set of terms of the stack calculus
\newcommand{\KTyp}[1]{\cT_\bd^{\mathsf{#1}}} % set of types of the stack calculus

%\newcommand{\Kstate}[4]{\langle({#1},{#2}),({#3},{#4})\rangle} % a state of the Krivine Abstract Machine involving a term 
\newcommand{\transition}{\longrightarrow} % transition symbol from one state of the Krivine Abstract Machine to another
\newcommand{\Kstate}[3]{\mbox{\boldmath{$\langle$}} \ {#1},{#2},{#3} \ \mbox{\boldmath{$\rangle$}}} % a state of the Krivine Abstract Machine involving a term 

\newcommand{\Kproc}[2]{\langle{#1},{#2}\rangle} % a state of the Krivine Abstract Machine involving a process
\newcommand{\Kclos}[2]{({#1},{#2})} % a closure of the Krivine Abstract Machine

\newcommand{\SN}[1]{\mathrm{SN}^{\mathsf{#1}}} % set of strongly normalizing expressions of the stack calculus

%\newcommand{\deg}[2]{\mathsf{deg}_{#1}(#2)} % degree of a variable in a n expression

\newcommand{\dgr}[2]{\mathsf{deg}_{#1}(#2)} % degree of a variable in a n expression

\newcommand{\bbot}{
\mathrel{\vcenter{\offinterlineskip
\vskip-.130ex\hbox{\begin{turn}{90}$\models$\end{turn}}}}} % Krivine's double bottom

\newcommand{\ttop}{
\mathrel{\vcenter{\offinterlineskip
\vskip-.130ex\hbox{\begin{turn}{270}$\models$\end{turn}}}}} % double top

\newcommand{\sepa}{
\mathrel{\vcenter{\offinterlineskip
\vskip-.130ex\hbox{\begin{turn}{90}$\succ$\end{turn}}}}} % separability

\newcommand{\asm}{\! : \!} % separator for type assumptions in contexts
\newcommand{\tass}{:} % separator type assignment in judgements

\newcommand{\tval}[1]{\vert #1\vert} % truth value interpretation of types into sets of terms
\newcommand{\fval}[1]{\lVert #1 \rVert} % falsehood value interpretation of types into sets of stacks
\newcommand{\tInt}[1]{\vert #1\vert} % truth value-like interpretation of term types into set of terms
\newcommand{\sInt}[1]{\vert #1\vert} % falsehood value-like interpretation of stack types into set of stacks
\newcommand{\pInt}[1]{\vert #1 \vert} % interpretation of the process type into a set of processes
\newcommand{\eInt}[1]{\vert #1 \vert} % interpretation of the expression type into a set of expressions
\newcommand{\Int}[1]{\llbracket #1\rrbracket} % interpretation of expressions in a mathematical domain
\newcommand{\id}{\mathsf{id}} % identity morphsism in a category
\newcommand{\pr}[1]{\mathsf{pr}_{#1}} % i-th projection of a cartesian product
\newcommand{\ev}{\mathsf{ev}} % evaluation morphism of a ccc

%\newcommand{\list}[1]{\langle #1 \rangle} % list constructor write inside the arguments separated by commas
\newcommand{\lis}[1]{\prec #1 \succ} % list constructor
\newcommand{\copair}[2]{[ #1, #2 ]} % copair constructor

\newcommand{\cur}[1]{\Lambda(#1)} % currying natural isomorphism
\newcommand{\invcur}[1]{\Lambda^{-1}(#1)} % inverse of currying natural isomorphism
\newcommand{\adbmaL}{
\mathrel{\vcenter{\offinterlineskip
\vskip-.100ex\hbox{\begin{turn}{180}$\Lambda$\end{turn}}}}}

\newcommand{\ctrliso}[1]{\phi(#1)} % natural isomorphism proper to control categories
\newcommand{\invctrliso}[1]{\phi^{-1}(#1)} % inverse of the natural isomorphism proper to control categories

\newcommand{\cocur}[1]{\adbmaL\!\!(#1)} % co-currying natural isomorphism
\newcommand{\invcocur}[1]{\adbmaL^{-1}(#1)} % inverse of co-currying natural isomorphism

\newcommand{\cord}{\sqsubseteq_c} % computational order on Bohm trees
\newcommand{\lord}{\sqsubseteq_l} % logical order on Bohm trees

\newcommand{\coher}{\stackrel{\frown}{\smile}} % Girard's coherence relation
\newcommand{\scoher}{\frown} % Girard's strict coherence relation

\newcommand{\Cl}[1]{Cl(#1)} % the set of cliques of a set 

\newcommand{\ccl}{\ensuremath{CCL}} % name of classical combinatory logic
\newcommand{\lmuo}{\ensuremath{\lambda\mu\mathbf{1}}} % name of Andou's lambda-mu calculus 
\newcommand{\lmc}{\ensuremath{\lambda C}} % name of Herbelin-De Groote's lambda-C calculus 
\newcommand{\lamb}{\ensuremath{\lambda}} % name of Church's lambda calculus 
\newcommand{\lmu}{\ensuremath{\lambda\mu}} % name of Parigot's lambda-mu calculus 
\newcommand{\stk}{\ensuremath{\bd}} % name of the stack calculus
\newcommand{\stke}{\ensuremath{\bd\eta}} % name of the extensional stack calculus
\newcommand{\stkw}{\ensuremath{\bd w}} % name of the stack calculus + weta
\newcommand{\lsp}{\ensuremath{\lambda\mathsf{sp}}} % name of the lambda calculus with surjective pairing
\newcommand{\lesp}{\ensuremath{\lambda\eta\mathsf{sp}}} % name of the extensional lambda calculus with surjective pairing
\newcommand{\ort}[1]{#1^{\bot}} % orthogonal object

\newcommand{\wi}{\binampersand} % with connective
\newcommand{\pa}{\bindnasrepma} % par connective

\newcommand{\te}{\mathsf{ten}} % tensor morphism
\newcommand{\parm}{\mathsf{par}} % par morphism

\newcommand{\mon}{\mathsf{m}} % monoidality morphism
\newcommand{\see}{\mathsf{s}} % seely isomorphism
\newcommand{\ut}{\mathsf{t}} % terminal morphism in a Cartesian category

\newcommand{\assoc}{\mathsf{ass}} % generalized associativity morphism

\newcommand{\der}{\mathsf{der}} % dereliction morphism
\newcommand{\coder}{\mathsf{cod}} % codereliction morphism

\newcommand{\coa}{\mathsf{h}} % coalgebra for the functor \ort{(\cdot)} \xrightarrow{\cdot}

\newcommand{\con}{\mathsf{con}} % contraction morphism
\newcommand{\wkn}{\mathsf{wkn}} % weakening morphism
\newcommand{\cowkn}{\mathsf{cow}} % coweakening morphism

\newcommand{\nco}{\overline{\mathsf{con}}} % negative contraction morphism
\newcommand{\nwk}{\overline{\mathsf{wkn}}} % negative weakening morphism

\newcommand{\dig}{\mathsf{dig}} % digging morphism

%\newcommand{\codig}{\mathsf{cod}} % codigging morphism

\newcommand{\dual}{\partial} % dualizing morphism
\newcommand{\ddual}{\partial^{-1}} % inverse of the dualizing morphism

\newcommand{\teid}{\mathbf{1}} % identity of the tensor product
\newcommand{\bon}{\mathbf{1}} % identity of the tensor product 

\newcommand{\DEC}{\mathsf{DEC}} % the class of decidable languages
\newcommand{\SDEC}{\mathsf{SDEC}} % the class of semi-decidable languages
\newcommand{\REG}{\mathsf{REG}} % the class of regular languages
\newcommand{\CFL}{\mathsf{CFL}} % the class of context-free languages
\newcommand{\dCFL}{\mathsf{dCFL}} % the class of deterministically context-free languages

\newcommand{\sqb}[1]{[#1]} % square brackets

\newcommand{\cnt}[1]{#1^\bullet} % center of a control category
\newcommand{\foc}[1]{#1^\sharp} % focus of a control category

\newcommand{\com}{\mathsf{comp}} % composition proof
\newcommand{\exc}{\mathsf{exc}} % exchange proof

\newcommand{\ax}{\mathsf{ax}} % axiom rule
\newcommand{\dne}{\mathsf{dne}} % double negation elimination rule
\newcommand{\raa}{\mathsf{raa}} % reductio ad absurdum rule
\newcommand{\efq}{\mathsf{efq}} % ex flaso quodlibet rule
\newcommand{\cut}{\mathsf{cut}} % cut rule
\newcommand{\dni}{\mathsf{dni}} % double negation introduction proof
\newcommand{\mpo}{\mathsf{mp}} % modus ponens rule
\newcommand{\dt}{\mathsf{dt}} % deduction theorem
\newcommand{\idem}{\mathsf{id}} % identity proof
\newcommand{\contp}{\mathsf{contp}} % contraposition proof (positive)
\newcommand{\contn}{\mathsf{contn}} % contraposition proof (negative)
\newcommand{\sded}{\mathsf{sded}} % symmetric deduction proof

\newcommand{\varrule}{\mathsf{ax}} % inference rule for variables
\newcommand{\carrule}{\to e_r} % inference rule for \car
\newcommand{\cdrrule}{\to e_l} % inference rule for \cdr
\newcommand{\atrule}{\to i} % inference rule for \at
\newcommand{\aprule}{\mathsf{cut}} % inference rule for \app
\newcommand{\nilrule}{\bot i} % inference rule for \nil
\newcommand{\bdrule}[1]{\bd,{#1}} % inference rule for \bd with reference to the bound variable 
\newcommand{\orrule}{\vee i} % inference rule for or
\newcommand{\rsallrule}{2\forall r} % inference rule for right introduction of the second order universal quantifier
\newcommand{\rfallrule}{\forall r} % inference rule for right introduction of the first order universal quantifier
\newcommand{\lsallrule}{2\forall l} % inference rule for left introduction of the second order universal quantifier
\newcommand{\lfallrule}{\forall l} % inference rule for left introduction of the first order universal quantifier

\newcommand{\leng}[1]{\sharp #1} % length of a sequence

%\newcommand{\wid}[1]{\mathsf{w}(#1)} % width of a term
%\newcommand{\bwid}[2]{\mathsf{w}(#1,#2)} % bounded width of a term
\newcommand{\wei}[1]{\mathsf{w}(#1)} % weight of a term
\newcommand{\bwei}[2]{\mathsf{w}(#1,#2)} % bounded weight of a term
\newcommand{\brea}[1]{\mathsf{b}(#1)} % breadth of a term
\newcommand{\bbrea}[2]{\mathsf{b}(#1,#2)} % bounded breadth of a term
\newcommand{\gap}[1]{\mathsf{g}(#1)} % gap of a term
\newcommand{\bgap}[2]{\mathsf{g}(#1,#2)} % bounded gap of a term

\newcommand{\wnot}{?} % why not modality
\newcommand{\bang}{!} % bang modality
\newcommand{\bbang}{!!} % double bang functor
\newcommand{\app}{F} % morphism from $U \to U \Rightarrow U$
\newcommand{\lam}{G} % morphism from $U \Rightarrow U \to U$
%\newcommand{\cur}{\Lambda} % currying
\newcommand{\cld}{\downarrow\!} % down arrow closure operator
\newcommand{\clu}{\uparrow\!} % up arrow closure operator
\newcommand{\clo}[1]{\overline{#1}} % overline closure operator
\newcommand{\clde}{\downarrow_\eta\!} % closure operator for the extensionality preorder
\newcommand{\parcl}[1]{\uparrow_{#1}\!} % parameterized closure operator
\newcommand{\cldn}[2]{\downarrow_{#1}\!{#2}} % downwards closure operator
\newcommand{\clup}[2]{\uparrow_{#1}\!{#2}} % upwards closure operator
\newcommand{\opp}[1]{{#1}^{\mathsf{op}}} % opposite

%Macro for stack sequents. The forms of annotated sequents are 
%\tystk{s}{stack}{stack-type}{context}
%\tystk{t}{term}{term-type}{context}
%\tystk{p}{process}{process-type}{context}
%Sequents without annotations
%\tystk{s}{}{stack-type}{context}
%\tystk{t}{}{term-type}{context}
%\tystk{p}{}{}{context}
\newcommand{\tystk}[4]{%
\ifthenelse{\equal{#1}{s}\OR\equal{#1}{t}}{
	\ifthenelse{\equal{#1}{s}}{%\equal{#1}{s}
		\ifthenelse{\isempty{#2}}{#3 \vdash #4}{\textcolor{blue}{#2} \textcolor{blue}{:} #3 \vdash #4}
		}{%\equal{#1}{t}
		\ifthenelse{\isempty{#2}}{\vdash #3, #4}{\vdash \textcolor{blue}{#2} \textcolor{blue}{:} #3 \ \textcolor{blue}{\mid} \ #4}
		}
	}{%\equal{#1}{p}
	\ifthenelse{\isempty{#2}}{\vdash #4}{\vdash \textcolor{blue}{#2} \textcolor{blue}{\mid} #4}
	}
}
\newcommand{\ntystk}[4]{%
\ifthenelse{\equal{#1}{s}\OR\equal{#1}{t}}{
	\ifthenelse{\equal{#1}{s}}{%\equal{#1}{s}
		\ifthenelse{\isempty{#2}}{#3 \nvdash #4}{\textcolor{blue}{#2} \textcolor{blue}{:} #3 \nvdash #4}
		}{%\equal{#1}{t}
		\ifthenelse{\isempty{#2}}{\nvdash #3, #4}{\nvdash \textcolor{blue}{#2} \textcolor{blue}{:} #3 \ \textcolor{blue}{\mid} \ #4}
		}
	}{%\equal{#1}{p}
	\ifthenelse{\isempty{#2}}{\nvdash #4}{\nvdash \textcolor{blue}{#2} \textcolor{blue}{\mid} #4}
	}
}

%Macro for lambda-mu sequents. The form is 
%\tylmu{left_context}{expression}{expression-type}{right_context}
\newcommand{\tylmu}[4]{
	\ifthenelse{\isempty{#3}}
	{%if
	\ifthenelse{\isempty{#4}}
		{#1 \vdash_{\lmu} \textcolor{blue}{#2}}
		{#1 \vdash_{\lmu} \textcolor{blue}{#2} \mid #4}
	}
	{%else
        \ifthenelse{\isempty{#4}}
        	{#1 \vdash_{\lmu} \textcolor{blue}{#2} \textcolor{blue}{:} #3}
        	{#1 \vdash_{\lmu} \textcolor{blue}{#2} \textcolor{blue}{:} #3 \textcolor{blue}{\mid} #4}
	}
}

%Macros for lambda-mu-mu-tilde sequents.
 
%\tylmmcom{command}{left_context}{right_context}
\newcommand{\tylmmcom}[3]{\textcolor{blue}{#1}\ \textcolor{blue}{\triangleright} #2 \vdash #3}
%\tylmmter{left_context}{term}{right_active_formula}{right_context}
\newcommand{\tylmmter}[4]{#1 \vdash \textcolor{blue}{#2} \textcolor{blue}{:} #3 \mid #4}
%\tylmmenv{left_context}{environment}{left_active_formula}{right_context}
\newcommand{\tylmmenv}[4]{#1 \mid \textcolor{blue}{#2} \textcolor{blue}{:} #3 \vdash #4}

%Macro for lambda-mu-one sequents. The form is 
%\tylmuo{left_context}{expression}{expression-type}
%\newcommand{\tylmuo}[3]{#1 \vdash_{\lmuo} \textcolor{blue}{#2} \textcolor{blue}{:} #3}
\newcommand{\tylmuo}[3]{#1 \vdash \textcolor{blue}{#2} \textcolor{blue}{:} #3}

%Macro for lambda sequents, i.e. typed lambda terms. The form is 
%\tylamb{left_context}{expression}{expression-type}
\newcommand{\tylamb}[3]{#1 \vdash_{\lamb} \textcolor{blue}{#2} \textcolor{blue}{:} #3}

%Macro for lambda-c sequents. The form is 
%\tylmc{left_context}{expression}{expression-type}
\newcommand{\tylmc}[3]{#1 \vdash_{\lmc} \textcolor{blue}{#2} \textcolor{blue}{:} #3}

%Macro for ccl sequents. The form is 
%\tyccl{left_context}{expression}{expression-type}
\newcommand{\tyccl}[3]{#1 \vdash_{\ccl} \textcolor{blue}{#2} \textcolor{blue}{:} #3}

\newcommand{\prov}[2]{#1 \vdash #2} % provability symbol
\newcommand{\refu}[2]{#1 \dashv #2} % refutation symbol


%%%%%%%% MACRO PER LE NOTE DEL CORSO DI CALCOLABILITA'

\newcommand{\eclose}[1]{\mathsf{ecl}(#1)} % operatore di epsilon-chiusura
\newcommand{\zr}{\mathsf{Z}} % the constantly zero function
\newcommand{\suc}{\mathsf{S}} % the successor function
\newcommand{\pred}{\mathsf{P}} % the predecessor function
\newcommand{\prj}[2]{I_{#1}^{#2}} % the projection function
\newcommand{\ca}[1]{\mathsf{c}_{#1}} % the characteristic function of a predicate
\newcommand{\minus}{\stackrel{\centerdot}{-}} % the minus function on natural numbers
\newcommand{\conv}[1]{{#1}\!\downarrow} % convergence of a function
\newcommand{\dive}[1]{{#1}\!\uparrow} % divergence of a function
\newcommand{\PR}{\mathbf{PR}} % partial recursive functions
\newcommand{\REC}{\mathbf{REC}} % total recursive functions
\newcommand{\PRIMREC}{\mathbf{PrimREC}} % primitive recursive functions
\newcommand{\RESET}{\Sigma} % set of all recursively enumerable sets
\newcommand{\RECSET}{\Delta} % set of all recursive sets
\newcommand{\sse}{\iff}
\newcommand{\bforall}[2]{\forall{#1}\!<\!{#2}} % bounded universal quantification
\newcommand{\bexists}[2]{\exists{#1}\!<\!{#2}} % bounded existential quantification
\newcommand{\bmu}[2]{\mu{#1}\!<\!{#2}} % bounded mu-recursion
\newcommand{\fprim}[1]{\mathsf{p}(#1)} % function returning the n-th prime number
\newcommand{\pprim}[1]{\mathsf{prim}(#1)} % predicate testing primality of a number
\newcommand{\expn}[2]{\mathsf{exp}(#1,#2)} % function returning the exponent of #1 in the unique prime decomposition of #2

%\newcommand{\nat}{\mathbb{N}}
%\newcommand{\pair}[2]{\langle #1,#2 \rangle}
\newcommand{\fset}[1]{\sharp(#1)}
%\newcommand{\Pf}[1]{\mathcal{P}_{\mathrm{f}}(#1)}
%\newcommand{\st}{:}
% \newcommand{\seq}[1]{\vec{#1}}
\newcommand{\gramm}{\mathrel{::=}} % EBNF grammar definition
\newcommand{\ass}{\mathrel{:=}} % syntactical definition 
\newcommand{\nat}{\mathbb{N}} % set of natural numbers
\newcommand{\st}{:} % set constructor
% \newcommand{\ass}{:=} % assignment
\newcommand{\car}[1]{\mathsf{c}_{#1}} % characteristic function
\newcommand{\ran}[1]{\mathsf{ran}(#1)} % range of a function
\newcommand{\dom}[1]{\mathsf{dom}(#1)} % domain of a function
\newcommand{\secod}[1]{\prec\! #1 \!\succ} % code of a sequence
\newcommand{\seq}[1]{\prec\! #1 \!\succ} % code of a sequence
\newcommand{\sedecod}[2]{(#1)_{#2}} % extract the #2-th element of the sequence with code #1
\newcommand{\pair}[2]{\langle #1,#2 \rangle} % coding of pairs
\newcommand{\gph}[1]{\mathsf{gr}(#1)} % graph of a function

\maketitle
%\tableofcontents

%%%%%%%%%%%%%%%%%%%%%%%%%%%%%%%%%%%%%%%%%%%%
\section{Introduzione}
%%%%%%%%%%%%%%%%%%%%%%%%%%%%%%%%%%%%%%%%%%%%

Nelle precedenti lezioni sono stati introdotti vari modelli astratti di automi che implementano  dispositivi di calcolo. Per le Macchine di Turing inoltre sono state definite codifiche e ``protocolli" per il loro utilizzo allo scopo di calcolare funzioni sui numeri naturali. In questa lezione faremo una cosa diversa: descriveremo una classe di funzioni sui numeri naturali in termini puramente matematici, senza riferimento ad alcun modello di macchina di calcolo e solo successivamente scopriremo che tali funzioni sono esattamente quelle calcolabili dalle Macchine di Turing.

%%%%%%%%%%%%%%%%%%%%%%%%%%%%%%%%%%%%%%%%%%%%
\section{Le funzioni primitive ricorsive}
%%%%%%%%%%%%%%%%%%%%%%%%%%%%%%%%%%%%%%%%%%%%

Un tipo di funzioni molto importanti sono le cosiddette \emph{funzioni caratteristiche}, che sono funzioni a valori in $\{0,1\}$ da intendersi come valori booleani che indicano l'appartenenza (o non-appartenenza) di $n$-uple di numeri a sottoinsiemi di $\nat^n$.

\begin{definition}[Funzione caratteristica]\label{def:fun-car}
Sia $R \subseteq \nat^n$ una relazione $n$-aria. La \emph{funzione caratteristica} di $R$, indicata con $\ca{R}: \nat^n \to \nat$, \`{e} definita come segue
$$
\ca{R}(\vec x) = 
\begin{cases}
1 & \mbox{se } R(\vec x) \\
0 & \mbox{altrimenti} \\
\end{cases}
$$
\end{definition}

\begin{definition}[Funzioni di base]\label{def:bas-fun}
Chiamiamo \emph{funzioni di base} quelle qui di seguito elencate:
\begin{itemize}
\item[(1)] la funzione $\ca{=}$;
\item[(2)] la funzione \emph{zero} $\zr: \nat \to \nat$ data da $\zr(x) = 0$, per ogni $x \in \nat$;
\item[(3)] la funzione \emph{successore} $\suc: \nat \to \nat$ data da $\suc(x) = x+1$, per ogni $x \in \nat$;
\item[(4)] per ogni coppia di numeri $j,n$ con $1\leq j \leq n$ la funzione \emph{proiezione $j$-esima} $\prj{j}{n}: \nat^n \to \nat$ data da $\prj{j}{n}(x_1,\ldots,x_n) = x_j$, per ogni $n$-upla $(x_1,\ldots,x_n) \in \nat^n$.
\end{itemize}
\end{definition}

\begin{definition}[Funzioni primitive ricorsive]\label{def:ric-prim}
L'insieme $\PRIMREC$ delle \emph{funzioni ricorsive primitive} \`{e}  il pi\`{u} piccolo insieme di funzioni sui numeri naturali che soddisfa le seguenti propriet\`{a}:
\begin{itemize}
\item[$(\mathbf{r1})$] Contiene tutte le funzioni di base della Definizione \ref{def:bas-fun}.
\item[$(\mathbf{r2})$] \`{E} chiuso per composizione, ovvero se contiene $\psi,\gamma_1,\ldots,\gamma_m$ allora contiene anche la funzione $\varphi$ data da $\varphi(\vec x) = \psi(\gamma_1(\vec x),\ldots,\gamma_m(\vec x))$.
\item[$(\mathbf{r3})$] \`{E} chiuso per ricorsione primitiva, ovvero se contiene $\psi,\gamma$ allora contiene anche la funzione $\varphi$ data da 
$$
\varphi(\vec x,y) = 
\begin{cases}
\psi(\vec x) & \mbox{se } y = 0 \\
\gamma(\vec x,y-1,\varphi(\vec x,y-1)) & \mbox{altrimenti} \\
\end{cases}
$$
\end{itemize}
\end{definition}

Le propriet\`{a} di chiusura $(\mathbf{r2})$ ed $(\mathbf{r3})$ della Definizione \ref{def:ric-prim}, sono detti, rispettivamente, \emph{schema di composizione} e \emph{schema di ricorsione primitiva}. Infatti, avendo definito $\PRIMREC$ come il pi\`{u} piccolo insieme di funzioni chiuse rispetto alle costruzioni elencate, esse rappresentano di fatto i soli schemi di definizione, da applicare un numero finito di volte alle funzioni di base, per creare nuove funzioni primitive ricorsive a partire da altre esistenti. In altre parole l'insieme $\PRIMREC$ \`{e} definito per induzione, e si possono pertanto fare dimostrazioni e costruzioni per induzione sulla definizione delle funzioni che vi appartengono.

%%%%%%%%%%%%%%%%%%%%%%%%%%%%%%%%%%%%%%%%%%%%
\subsection{Schemi alternativi}
%%%%%%%%%%%%%%%%%%%%%%%%%%%%%%%%%%%%%%%%%%%%

Grazie alle funzioni di proiezione e alle propriet\`{a} elementari dell'aritmetica, si possono creare nuovi e pi\`{u} flessibili schemi derivati da quello di composizione e di ricorsione primitiva che rendono pi\`{u} comode le definizioni, pur non aggiungendo nuove funzioni a $\PRIMREC$.

\begin{lemma}\label{def:alt-ric-prim1}
\begin{enumerate}[label=(\roman*)]
\item La funzione identit\`{a} \`{e} in $\PRIMREC$.
\item Sia $\psi: \nat^m \to \nat$ una funzione in $\PRIMREC$, sia $n \geq m$ e siano ${j_1},\ldots,{j_m}$ indici presi in $\{1,\ldots,n\}$. Allora la funzione $\varphi: \nat^n \to \nat$ data da\\
$\varphi(x_1,\ldots,x_n) = \psi(x_{j_1},\ldots,x_{j_m})$ \`{e} anch'essa in $\PRIMREC$.
\item Ogni funzione costante \`{e} in $\PRIMREC$.
\item Se $\psi,\gamma$ sono in $\PRIMREC$, anche la funzione $\varphi$ data da 
$$
\varphi(\vec x,y) = 
\begin{cases}
\psi(\vec x) & \mbox{se } y = 0 \\
\gamma(\varphi(\vec x,y-1)) & \mbox{altrimenti} \\
\end{cases}
$$
\`{e} in $\PRIMREC$.
\item Se $\psi,\gamma$ sono in $\PRIMREC$, anche la funzione $\varphi$ data da 
$$
\varphi(\vec x,y) = 
\begin{cases}
\psi(\vec x) & \mbox{se } y = 0 \\
\gamma(\vec x,\varphi(\vec x,y-1)) & \mbox{altrimenti} \\
\end{cases}
$$
\`{e} in $\PRIMREC$.
\item La funzione predecessore \`{e} in $\PRIMREC$.
\end{enumerate}
\end{lemma}

\begin{proof}
\begin{enumerate}[label=(\roman*)]
\item L'identit\`{a} \`{e} la funzione $\prj{1}{1}(x)$.
\item Chiaramente $\varphi(\vec x) = \psi(\prj{j_1}{n}(\vec x),\ldots,\prj{j_m}{n}(\vec x))$, quindi \`{e} in $\PRIMREC$ per lo schema di composizione.
\item Sia $\psi$ data da $\psi(\vec x) = n$, per un certo $n \in \nat$. Allora 
$$\psi(\vec x) = \underbrace{\suc(\cdots\suc(}_{n\mbox{ volte}}\zr(x_1))\cdots)$$
quindi $\psi \in \PRIMREC$ per il punto (2) e lo schema di composizione.
\item Si ponga $\gamma'(\vec x,y,z) = \gamma(z)$. Allora $\gamma'$ \`{e} in $\PRIMREC$ per il punto (2). Ora la funzione $\varphi$ \`{e} data da 
$$
\varphi(\vec x,y) = 
\begin{cases}
\psi(\vec x) & \mbox{se } y = 0 \\
\gamma'(\vec x,y-1,\varphi(\vec x,y-1)) & \mbox{altrimenti} \\
\end{cases}
$$
e quindi, per lo schema di ricorsione primitiva, \`{e} in $\PRIMREC$.
\item Simile al punto precedente.
\item La funzione predecessore sui numeri naturali \`{e} data da 
$$
\pred(y) = 
\begin{cases}
0 & \mbox{se } y = 0 \\
\prj{1}{2}(y-1,\pred(y-1)) & \mbox{altrimenti} \\
\end{cases}
$$
e quindi \`{e} in $\PRIMREC$ per lo schema di ricorsione primitiva.
\end{enumerate}
\qed\end{proof}

Notiamo che se $\varphi(\vec x,y): \nat^{n+1} \to \nat$ \`{e} una funzione ricorsiva, allora per ogni numero naturale $k$ fissato la funzione $\varphi(\vec x,k): \nat^n \to \nat$ \`{e} una funzione ricorsiva, per via dello schema di composizione.

Facciamo seguire alcuni esempi di funzioni a noi familiari che sono primitive ricorsive. Per ciascuna di esse cercate di capire quali schemi sono stati utilizzati nella definizione. 

\[
\begin{array}{ll}
x\minus y = 
\begin{cases}
x & \mbox{se } y = 0 \\
\pred(x\minus(y-1)) & \mbox{altrimenti} \\
\end{cases}
\qquad
&
x+y = 
\begin{cases}
x & \mbox{se } y = 0 \\
\suc(x+(y-1)) & \mbox{altrimenti} \\
\end{cases}
\\
x \cdot y = 
\begin{cases}
0 & \mbox{se } y = 0 \\
x + (x\cdot (y-1)) & \mbox{altrimenti} \\
\end{cases}
&
x^y = 
\begin{cases}
1 & \mbox{se } y = 0 \\
x \cdot (x^{y-1}) & \mbox{altrimenti} \\
\end{cases}
\\
!y = 
\begin{cases}
1 & \mbox{se } y = 0 \\
y \cdot !(y-1) & \mbox{altrimenti} \\
\end{cases}
&
\end{array}
\]

Chiaramente, utilizzando il solo schema di rimpiazzamento, otteniamo che se $\psi_1(\vec x),\ldots,\psi_k(\vec x)$ sono funzioni primitive ricorsive, anche le funzioni $\varphi(\vec x) = \sum_{i=1}^{k} \psi_i(\vec x)$ e $\gamma(\vec x) = \prod_{i=1}^{k} \psi_i(\vec x)$ sono primitive ricorsive. Altre importantissime funzioni primitive ricorsive sono le cosiddette \emph{somme e prodotti limitati}, che generalizzano il caso delle somme appena visto, in quanto si vuole permettere al numero degli addendi di variare in funzione dell'argomento passato alla funzione. Data una funzione primitiva ricorsiva $n+1$-aria $\varphi(\vec x,y)$, le funzioni $\sum_{y < z} \varphi(\vec x,y)$ e $\prod_{y < z} \varphi(\vec x,y)$ (anch'esse da $\nat^{n+1}$ in $\nat$) sono definite come segue:

\[
\begin{array}{l}
\sum_{y < z} \varphi(\vec x,y) = 
\begin{cases}
0 & \mbox{se } z = 0 \\
\varphi(\vec x,z) + \sum_{y < z-1} \varphi(\vec x,y) & \mbox{altrimenti} \\
\end{cases}
\\
\qquad
\\
\prod_{y < z} \varphi(\vec x,y) = 
\begin{cases}
1 & \mbox{se } z = 0 \\
\varphi(\vec x,z) \cdot \prod_{y < z-1} \varphi(\vec x,y) & \mbox{altrimenti} \\
\end{cases}
\\
\end{array}
\]

Inoltre per lo schema di rimpiazzamento anche le funzioni $\sum_{y < \psi(\vec x,z)} \varphi(\vec x,y)$ e $\prod_{y < \psi(\vec x,z)} \varphi(\vec x,y)$ sono primitive ricorsive, qualora $\psi(\vec x,z)$ e $\varphi(\vec x,y)$ lo siano.

%%%%%%%%%%%%%%%%%%%%%%%%%%%%%%%%%%%%%%%%%%%%
\section{Insiemi e predicati primitivi ricorsivi}
%%%%%%%%%%%%%%%%%%%%%%%%%%%%%%%%%%%%%%%%%%%%

\begin{definition}\label{def:prim-ric-pred}
Sia $R \subseteq \nat^n$ una relazione $n$-aria. Diciamo che $R$ \`{e} \emph{primitiva ricorsiva} se la sua funzione caratteristica $\ca{R}$ \`{e} una funzione primitiva ricorsiva.
\end{definition}

Ad esempio l'uguglianza \`{e} una relazione primitiva ricorsiva. Vediamo come si comportano i predicati primitivi ricorsivi rispetto alle operazioni logiche.

\begin{lemma}\label{lem:log-clos}
Siano $R,P \subseteq \nat^n$ due predicati primitivi ricorsivi. Allora $P \vee R$, $P \wedge R$, $P \Rightarrow R$ e $\neg P$ sono tutti predicati primitivi ricorsivi.
\end{lemma}

\begin{proof}
Abbiamo che $\ca{P \wedge R}(\vec x) = \ca{P}(\vec x) \cdot \ca{P}(\vec x)$ e $\ca{\neg P}(\vec x) = 1 - \ca{P}(\vec x)$. Infine $\ca{P \vee R}(\vec x) = \ca{\neg(\neg P \wedge \neg R)}(\vec x)$ e $\ca{P \Rightarrow R}(\vec x) = \ca{\neg P \vee R}(\vec x)$.
\qed\end{proof}

\begin{lemma}\label{lem:confronto}
Le relazioni $<,>,\leq,\geq,\neq$ sono primitive ricorsive.
\end{lemma}

\begin{proof}
Abbiamo che $x < y \sse (x \minus y = 0) \wedge (y \minus x \neq 0)$. Quindi $\ca{<}(x,y)$ \`{e} una funzione primitiva ricorsiva dato che $\ca{<}(x,y)=\ca{=}(x \minus y,0) \cdot (1- \ca{=}(y \minus x,0))$. Usando le propriet\`{a} degli operatori logici ora si possono definire le funzioni caratteristiche delle altre relazioni.
\qed\end{proof}

\begin{lemma}
Sia $\varphi(\vec x)$ in $\PRIMREC$. Allora il predicato $P(\vec x,y) \ass \varphi(\vec x) = y$ (oppure $<,>$, ecc...) \`{e} primitivo ricorsivo.
\end{lemma}

\begin{proof}
Immediato, poich\'{e} $\ca{P}(\vec x,y) = \ca{=}(\varphi(\vec x),y)$ che \`{e} in $\PRIMREC$ per lo schema di rimpiazzamento.
\qed\end{proof}


\begin{lemma}[Schema di definizione per casi]\label{lem:def-casi}
Siano $\psi_1,\ldots,\psi_k$ funzioni primitive ricorsive e siano $R_1,\ldots,R_k$ predicati primitivi ricorsivi $n$-ari a due a due disgiunti e tali che $\bigcup_{i=1}^{k} R_i=\nat^k$. Allora la funzione $\varphi$ definita dallo schema
$$
\varphi(\vec x) =
\begin{cases}
\psi_1(\vec x) & \mbox{se } R_1(\vec x) \\
\vdots & \vdots \\
\psi_k(\vec x) & \mbox{se } R_k(\vec x) \\
\end{cases}
$$
\`{e} primitiva ricorsiva.
\end{lemma}

\begin{proof}
Abbiamo $\varphi(\vec x) = \sum_{i=1}^{k} \ca{R_i}(\vec x)\cdot\psi_i(\vec x)$.
\qed\end{proof}

Come conseguenza dei Lemmi \ref{lem:def-casi} e \ref{lem:confronto} la funzione \emph{massimo} tra due numeri \`{e} primitiva ricorsiva poich\'{e} abbiamo 
$$
max(x,y) =
\begin{cases}
x & \mbox{se } x \geq y \\
y & \mbox{se } x < y \\
\end{cases}
$$

Quindi anche il massimo di una lista di valori di funzione primitiva ricorsiva $\varphi$ \`{e} una funzione primitiva ricorsiva in quanto
$$
max_{y < z} \varphi(\vec x,y) = 
\begin{cases}
0 & \mbox{se } z = 0 \\
max(\varphi(\vec x,z), max_{y < z-1} \varphi(\vec x,y)) & \mbox{altrimenti} \\
\end{cases}
$$

Con un ragionamento analogo si dimostra che la funzione $min_{y < z} \varphi(\vec x,y)$ \`{e} anch'essa primitiva ricorsiva se $\varphi$ lo \`{e}.

Vediamo come si comportano i predicati primitivi ricorsivi rispetto alla quantificazione. Anticipando che essi non sono chiusi rispetto a quantificazioni universali ed esistenziali arbitrarie, ci interessiamo ad una particolare forma di quantificazione, detta \emph{quantificazione limitata}. Sia $P(\vec x,y)$ una relazione $n+1$-aria. Definiamo due relazioni $n+1$-arie
$$ \bforall{y}{z}.P(\vec x,y) \ass \forall y.(y \geq z \vee P(\vec x,y)) \qquad\qquad \bexists{y}{z}.P(\vec x,y) \ass \exists y.(y < z \wedge P(\vec x,y)) $$

\begin{lemma}[Quantificazione limitata]\label{lem:bound-quant}
Siano $R\subseteq \nat^{n+1}$ un predicato primitivo ricorsivo. Allora $\bforall{y}{z}.P(\vec x,y)$ e $\bexists{y}{z}.P(\vec x,y)$ sono predicati primitivi ricorsivi.
\end{lemma}

\begin{proof}
Siano $R \ass \bforall{y}{z}.P(\vec x,y)$ e $Q \ass \bexists{y}{z}.P(\vec x,y)$. Abbiamo che $\ca{R}(\vec x,z) = min_{y < z} \ca{P}(\vec x,z)$ e similmente $\ca{Q}(\vec x,z) = max_{y < z} \ca{Q}(\vec x,y)$.
\qed\end{proof}

Non \`{e} difficile immaginare come estendere il Lemma \ref{lem:bound-quant} al caso di predicati come $\forall y\!\leq\!z.P(\vec x,y)$ e $\bforall{y}{\psi(\vec x,y)}.P(\vec x,y)$ (con $\psi(\vec x,y)$ primitiva ricorsiva).

\begin{definition}[$\mu$-ricorsione limitata]\label{def:mu-rec-lim}
Sia $P(\vec x,y)$ un predicato $n+1$-ario primitivo ricorsivo. Allora definiamo la funzione $\bmu{y}{z}.P(\vec x,y): \nat^{n+1} \to \nat$ come segue
$$
\bmu{y}{z}.P(\vec x,y) = 
\begin{cases}
\mbox{il minimo $y$ tale che $P(\vec x,y)$} & \mbox{se } \bexists{y}{z}.P(\vec x,y) \\
z & \mbox{altrimenti} \\
\end{cases}
$$
\end{definition}

\begin{lemma}[Schema di definizione per $\mu$-ricorsione limitata]\label{lem:def-mu-lim}
Sia $R$ un predicato primitivo ricorsivo $n+1$-ario e sia $\psi(\vec x,z)$ una funzione primitiva ricorsiva $n+1$-aria. Allora
\begin{enumerate}[label=(\roman*)]
\item la funzione $\varphi$ definita dallo schema $\varphi(\vec x,z) = \bmu{y}{z}.P(\vec x,y)$ \`{e} primitiva ricorsiva.
\item la funzione $\varphi$ definita dallo schema $\varphi(\vec x,z) = \bmu{y}{\psi(\vec x,z)}.P(\vec x,y)$ \`{e} primitiva ricorsiva.
\end{enumerate}
\end{lemma}

\begin{proof}
\begin{enumerate}[label=(\roman*)]
\item Per i Lemmi \ref{lem:log-clos} e \ref{lem:bound-quant} il predicato $R(\vec x,y) \ass P(\vec x,y) \wedge (\bforall{z}{y}.\neg P(\vec x,z))$ \`{e} primitivo ricorsivo. Infine osserviamo che $\varphi(\vec x,z) = \sum_{y < z+1} y \cdot \ca{R}(\vec x,y)$.
\item Semplicemente usando il punto (i) e lo schema di rimpiazzamento.
\end{enumerate}
\qed\end{proof}

Analizziamo ora un importante caso particolare del Lemma \ref{lem:def-mu-lim}. 

\begin{lemma}[Schema di definizione per $\mu$-ricorsione limitata autorefente]\label{lem:def-mu-lim-auto}
Siano $\gamma(\vec x,z,y)$ e $\delta(\vec x,z)$ due funzioni primitive ricorsive. Allora
\begin{enumerate}[label=(\roman*)]
\item la funzione $\varphi$ definita dallo schema $\varphi(\vec x,z) = \bmu{y}{z}.[\gamma(\vec x,\varphi(\vec x,z-1),y) = 0]$ \`{e} primitiva ricorsiva.
\item la funzione $\varphi$ definita dallo schema $\varphi(\vec x,z) = \bmu{y}{\delta(\vec x,\varphi(\vec x,z-1))}.[\gamma(\vec x,\varphi(\vec x,z-1),y) = 0]$ \`{e} primitiva ricorsiva.
\end{enumerate}
\end{lemma}

\begin{proof}
\begin{enumerate}[label=(\roman*)]
\item Sia $R_\gamma(\vec x,z,y) \ass (\gamma(\vec x,z,y) = 0 \wedge (\bforall{z}{y}.\gamma(\vec x,z,y) \neq 0))$. La funzione $\ca{R}(\vec x,z,y)$ \`{e} primitiva ricorsiva. Infine osserviamo che
$$
\varphi(\vec x,z) = 
\begin{cases}
0 & \mbox{se } z = 0 \\
\sum_{y < z} y \cdot \ca{R_\gamma}(\vec x,\varphi(\vec x,z-1),y) & \mbox{altrimenti} \\
\end{cases}
$$
e quindi $\varphi(\vec x,z) \in \PRIMREC$ per lo schema di ricorsione primitiva e di rimpiazzamento.
\item Si tratta di una semplice estensione del punto (i).
\end{enumerate}
\qed\end{proof}

In sostanza il Lemma \ref{lem:def-mu-lim-auto} dice che quando definiamo una funzione per $\mu$-ricorsione limitata, sia il predicato di controllo che il bound possono essere dati in funzione del valore di $\varphi(\vec x,z-1)$. 

Prendiamo ad esempio la definizione $\varphi(\vec x,z) = \bmu{y}{z+1}.[y > \varphi(\vec x,z-1)]$. Tale funzione si pu\`{o} definire come segue:
$$
\varphi(\vec x,z) = 
\begin{cases}
0 & \mbox{se } z = 0 \\
z+1 & \mbox{altrimenti} \\
\end{cases}
$$

%%%%%%%%%%%%%%%%%%%%%%%%%%%%%%%%%%%%%%%%%%%%
\subsection{Codaggio delle sequenze e lo schema ``course-of-values"}
%%%%%%%%%%%%%%%%%%%%%%%%%%%%%%%%%%%%%%%%%%%%

A questo punto ci si pu\`{o} domandare se una funzione famosa come quella di Fibonacci, data da
$$ 
F(x) =
\begin{cases}
0 & \mbox{se } x = 0 \\
1 & \mbox{se } x = 1 \\
F(x-1) + F(x-2) & \mbox{se } x \geq 2 \\
\end{cases}
$$
sia o meno primitiva ricorsiva. A prima vista diremmo che non vi \`{e} uno schema che permette di definirla poich\`{e} nella sua definizione induttiva il calcolo pu\`{o} richiedere pi\`{u} di un valore della funzione stessa su argomenti pi\`{u} piccoli. Tuttavia stiamo per vedere che la funzione di Fibonacci \`{e} primitiva ricorsiva, ma per dimostrarlo abbiamo bisogno di un meccanismo di codifica (primitivo ricorsivo!) per codificare le sequenze di numeri naturali nei numeri naturali stessi.

Prima di procedere abbiamo bisogno di dimostrare che alcune funzioni sono primitive ricorsive. Ricordiamo che per il Teorema Fondamentale dell'Aritmetica ogni numero naturale $n$ si scrive in maniera unica come prodotto di potenze di numeri primi: tale prodotto \`{e} chiamato \emph{decomposizione prima} del numero $n$.

\begin{lemma}
I seguenti predicati e funzioni sono primitivi ricorsivi:
\begin{enumerate}[label=(\roman*)]
\item il predicato $x \mid y \ass$ ``$x$ divide $y$"
\item il predicato $\pprim{x} \ass$ ``$x$ \`{e} primo"
\item la funzione $\fprim{x} =$ ``l'$x$-esimo numero primo"
\item la funzione $\expn{x}{y} =$ ``il pi\`{u} grande numero $k$ tale che $x^k$ divide $y$"
\end{enumerate}
\end{lemma}

\begin{proof}
\begin{enumerate}[label=(\roman*)]
\item Abbiamo $x \mid y \iff \bexists{z}{y+1}.x\cdot z = y$.
\item Abbiamo $\pprim{x} \iff x \geq 2 \wedge \bforall{y}{x+1}.y \mid x \Rightarrow (y = 1 \vee y = x)$.
\item Osserviamo che, per la famosa dimostrazione di Euclide sull'infinit\`{a} dei numeri primi, l'$x$-esimo numero primo \`{e} minore o uguale al prodotto dei primi $x-1$ numeri primi  aumentato di uno. Pertanto abbiamo $\fprim{x} = \bmu{y}{!\fprim{x-1}+2}.[y > \fprim{x-1} \wedge \pprim{y}]$.
\item Abbiamo $\expn{x}{y} = \bmu{z}{y+1}.[x^z\mid y \wedge \neg (x^{z+1}\mid y)]$.
\end{enumerate}
\qed\end{proof}

Utilizzando questi strumenti possiamo progettare funzioni primitive ricorsive di codifica e decodifica delle sequenze di numeri naturali. 

Indicheremo nel seguito con $p_i$ l'$i+1$-esimo numero primo (pertanto $p_0 = 2$).
 Definiamo la codifica della sequenza $(x_1,\ldots,x_n)$ come il numero
 $\seq{x_1,\ldots,x_n} = p_0^n \cdot \prod_{i=1}^{n} p_i^{x_i}$. Per la decodifica, dato un numero $x$ poniamo:
\begin{itemize}
\item $Seq(x) \iff \forall n \leq x.[(n > 0 \wedge (x)_n \neq 0) \Rightarrow n \leq (x)_0]$
\item $(x)_n = \expn{p_n}{x}$
\item $ln(x) = (x)_0$
\end{itemize}

Dunque $Seq(x)$ \`{e} vero sse $x$ \`{e} la codifica di una qualche sequenza; se $Seq(x)$ vale,
 allora $x = \seq{(x)_1,\ldots,(x)_{ln(x)}}$. I numeri naturali $n$ tali che $Seq(n)$ sono detti \emph{numeri di sequenza}.

Si pu\`{o} anche definire l'operazione di \emph{concatenazione di numeri di sequenza} come segue:
$$
x \ast y =
\begin{cases}
p_0^{ln(x)+ln(y)} \cdot \prod_{i < ln(x)} p_{i+1}^{(x)_{i+1}} \cdot \prod_{j < ln(y)} p_{ln(x) + j+1}^{(y)_{j+1}} & \mbox{se $Seq(x) \wedge Seq(y)$} \\
0 & \mbox{altrimenti} \\
\end{cases}
$$
Qiondi $x \ast y$ \`{e} il numero di sequenza della concatenazione delle sequenze di cui $x$ ed $y$ sono le codifiche. In altre parole $\langle x_1,\ldots,x_n\rangle \ast \langle y_1,\ldots,y_m\rangle = \langle x_1,\ldots,x_n, y_1,\ldots,y_m\rangle$.

Con il prossimo teorema, che fa uso della codifica delle sequenze, dimostriamo che la classe delle funzioni primitive ricorsive \`{e} chiusa per definizioni ricorsive in cui il calcolo di  $f(\vec x,y+1)$ possa coinvolgere non solo il valore $f(\vec x,y)$, ma anche possibilmente tutti i valori nell'insieme finito $\{f(\vec x,z) \st z < y \}$.

Allo scopo del teorema definiamo la \emph{funzione storia} $\hat{f}$ di $f$ come seque:
$$ \hat{f}(\vec x,y) = \langle f(\vec x,0),\ldots,f(\vec x,y)\rangle $$

\begin{theorem}[Course-of-values recursion]\label{thm:course-of-vals}
Siano $g,h$ funzioni primitive ricorsive e sia $f$ definita dallo schema:
\begin{itemize}
  \item[ ] $f(\vec x,0) = g(\vec x)$ 
  \item[ ] $f(\vec x,y+1) = h(\vec x,y,\hat{f}(\vec x,y))$
\end{itemize}
Allora $f$ e $\hat{f}$ sono entrambe primitive ricorsive.
\end{theorem}

\begin{proof}
\`{E} sufficiente dimostrare che $\hat{f}$ \`{e} primitiva ricorsiva, poich\'{e} $f$ lo sar\`{a} di conseguenza. Ora notiamo che 

\begin{eqnarray*}
\hat{f}(\vec x,0) & = & \seq{f(\vec x,0)} \\
 & = & \seq{g(\vec x)}
\end{eqnarray*}
\begin{eqnarray*}
\hat{f}(\vec x,y+1) & = & \seq{f(\vec x,0),\ldots,f(\vec x,y),f(\vec x,y+1)} \\
 & = & \hat{f}(\vec x,y) \ast \seq{f(\vec x,y+1)} \\
 & = & \hat{f}(\vec x,y) \ast \seq{h(\vec x,y,\hat{f}(\vec x,y))}
\end{eqnarray*}

Pertanto il calcolo di $\hat{f}(\vec x,y+1)$ utilizza solo il valore $\hat{f}(\vec x,y)$ e le funzioni primitive ricorsive $\ast$ e $h$. Quindi $\hat{f}$ \`{e} primitiva ricorsiva dato che \`{e} definita con lo schema [$(\mathbf{r3})$. Infine $f$ \`{e} primitiva ricorsiva perch\'{e} $f(\vec x,y) = (\hat{f}(\vec x,y))_{y+1}$.
\end{proof}

Come conseguenza del Teorema \ref{thm:course-of-vals} la funzione di Fibonacci (e tante altre) \`{e} primitiva ricorsiva.

%%%%%%%%%%%%%%%%%%%%%%%%%%%%%%%%%%%%%%%%%%%%
\section{Le funzioni ricorsive totali e parziali}
%%%%%%%%%%%%%%%%%%%%%%%%%%%%%%%%%%%%%%%%%%%%

Un'altra famosa funzione \`{e} la \emph{funzione di Ackermann}, definita come segue:
$$
A(x,y) =
\begin{cases}
y+1 & \mbox{se $x = 0$} \\
A(x-1,1)        & \mbox{se $x > 0$ e $y = 0$} \\
A(x-1,A(x,y-1)) & \mbox{se $x >0$ e $y>0$} \\
\end{cases}
$$
\`{E} abbastanza immediato intuire che si possa scrivere un programma per calcolatore che computi la funzione di Ackermann. Molto meno immediato \`{e} intuire che questa funzione non \`{e} primitiva ricorsiva, eppure ci\`{o} si pu\`{o} dimostrare. Questo vuol dire che le funzioni ricorsive primitive non sono sufficienti a catturare tutte quelle calcolabili dalle macchine di Turing. Un'altra osservazione importante \`{e} che alcune funzioni calcolate da macchine di Turing sono \emph{parziali}, ovvero vi sono degli input su cui esse non danno alcun risulato: questo perch\`{e} alcune macchine di Turing non si arrestano mai su certi input.

Per questi motivi andiamo ad esaminare una classe pi\`{u} ampia di funzioni, che riesca a comprendere ad esempio anche la funzione di Ackermann ed in generale anche le funzioni parziali calcolate dalle macchine di Turing. Per procedere su questa strada abbiamo bisogno di un nuovo fondamentale strumento, la \emph{$\mu$-ricorsione} che andiamo a definire qui di seguito.

\begin{definition}[$\mu$-ricorsione]\label{def:mu-rec}
Sia $P(\vec x,y)$ un predicato (o sottoinsieme di $\nat^{n+1}$). Allora definiamo la funzione $\mu y.P(\vec x,y): \nat^n \to \nat$ come segue
$$
\mu y.P(\vec x,y) = 
\begin{cases}
\mbox{il minimo $y$ tale che $P(\vec x,y)$}& \mbox{se un tale $y$ esiste} \\
\uparrow & \mbox{altrimenti} \\
\end{cases}
$$
\end{definition}

Appare immediatamente evidente che lo schema della Definizione \ref{def:mu-rec} pu\`{o} produrre funzioni parziali. Tale tipo di $\mu$-ricorsione \`{e} anche detta \emph{illimitata} perch\`{e} esistono alcuni importanti casi paticolari della $\mu$-ricorsione, ottenuti ponendo qualche vincolo, che producono esclusivamente funzioni totali:
\begin{itemize}
\item quando il predicato $P$ \`{e} tale che $\forall \vec x.\exists y.P(\vec x,y)$, allora lo schema diventa la \emph{$\mu$-ricorsione totale},
\item quando il predicato $P$ \`{e} della forma $P(\vec x,z,y) \ass P'(\vec x,y) \vee y = z$, allora lo schema diventa la \emph{$\mu$-ricorsione limitata}. 
\end{itemize}
Nei due casi appena elencati lo schema di $\mu$-ricorsione produce solo funzioni totali. 

Gli schemi di $\mu$-ricorsione qui sopra descritti non producono per\`{o} necessariamente delle funzioni adatte ai nostri scopi. Difatti \`{e} necessario per noi che il predicato $P(\vec x,y)$ usato nella definizione di $\mu y.P(\vec x,y)$ abbia una particolare forma.

\begin{definition}[Funzioni ricorsive parziali]\label{def:ric-par}
L'insieme $\PR$ delle \emph{funzioni ricorsive parziali} \`{e} il pi\`{u} piccolo insieme di funzioni sui numeri naturali che soddisfa le propriet\`{a} ($\mathbf{r1}$)-($\mathbf{r3}$) della Definizione \ref{def:ric-prim} ed in pi\`{u} la seguente propriet\`{a}:
\begin{itemize}
\item[$(\mathbf{r4})$] Se $\psi \in \PR$ e $P(\vec x,y) \ass (\forall z \leq y.\ \conv{\psi(\vec x,z)}) \wedge \psi(\vec x,y) = 0$, allora la funzione $\varphi$ data da $\varphi(\vec x) = \mu y.P(\vec x,y)$ appartiene a $\PR$.
\end{itemize}
\end{definition}

Chiamiamo $\REC$ l'insieme delle funzioni ricorsive parziali che sono totali. Abbiamo evidentemente $\REC \subset \PR$.

\begin{lemma}\label{lem:ric-tot}
L'insieme $\REC$ delle funzioni ricorsive totali \`{e} il pi\`{u} piccolo insieme di funzioni sui naturali che soddisfa le propriet\`{a} ($\mathbf{r1}$)-($\mathbf{r3}$) della Definizione \ref{def:ric-prim} e la seguente restrizione della propriet\`{a} ($\mathbf{r4}$):
\begin{itemize}
\item[$(\mathbf{r4}')$] Se $\psi \in \REC$ e $\forall \vec x.\exists y.\psi(\vec x,y) = 0$, allora la funzione $\varphi$ data da $\varphi(\vec x) = \mu y.[\psi(\vec x,y) = 0]$ appartiene a $\REC$.
\end{itemize}
\end{lemma}

Si noti che l'unica causa possibile di parzialit\`{a} per una funzione in $\PR$ \`{e} proprio lo schema di $\mu$-ricorsione illimitata, per cui con la $\mu$-ricorsione totale si catturano tutte le funzioni ricorsive totali.

Riportiamo, senza dimostrarlo, il fatto che la funzione di Ackermann appartiene a $\REC$. Siccome essa non appartiene a $\PRIMREC$, abbiamo $\PRIMREC \subset \REC$.

%%%%%%%%%%%%%%%%%%%%%%%%%%%%%%%%%%%%%%%%%%%%
%\bibliographystyle{abbrv}%splncs
%\bibliography{bibliography}
%%%%%%%%%%%%%%%%%%%%%%%%%%%%%%%%%%%%%%%%%%%%
\end{document}

%%%%%%%%%%%%%%%%%%%%%%%%%%%%%%%%%%%%%%%%%%%%
%%%%%%%%%%%%%%%%%%%%%%%%%%%%%%%%%%%%%%%%%%%%
%%%%%%%%%%%%%%%%%%%%%%%%%%%%%%%%%%%%%%%%%%%%